
\usepackage[paperwidth=150mm, paperheight=230mm]{geometry}
\usepackage{fontspec}
%\usepackage[latin1]{inputenc}
\usepackage[french]{babel}
\usepackage[strict]{changepage}
\usepackage{fancyhdr}
\usepackage{paracol}
\usepackage{tableof}
\usepackage{setspace}
\usepackage{alltt}
\usepackage{titlesec}
\usepackage{xcolor}
\usepackage{xstring}
\usepackage{parskip}
\usepackage{enumitem}
\usepackage{etoolbox}
\usepackage{needspace}

%%%%%%%%%%%%%%%%%%%%%%%%%%%%%%%%%%%%%%%%%%%%%%%%%%% Mise en page %%%%%%%%%%%%%%%%%%%%%%%%%%%%%%
% on numérote les nbp par page et non globalement
\usepackage[perpage]{footmisc}

% définition des en-têtes et pieds de page
\pagestyle{empty}
\fancyhead{}
\fancyfoot{}
\renewcommand{\headrulewidth}{0pt}
\setlength{\headheight}{0pt}

% la commande titres permet de changer les titres de gauche et de droite.
\newcommand{\titres}[2]{
	\renewcommand{\rightmark}{\textcolor{red}{\sc #2}}
	\renewcommand{\leftmark}{\textcolor{red}{\sc #1}}
}
\titres{}{}

% pas d'indentation
\setlength{\parindent}{0mm}

\geometry{
inner=20mm,
outer=20mm,
top=10mm,
bottom=25mm,
headsep=0mm,
}

\twosided[p]

%%%%%%%%%%%%%%%%%%%%%%%%%%%%%%%%%%%%%%%%%%%%%%%%% Options gregorio %%%%%%%%%%%%%%%%%%%%%%%%%

\usepackage[autocompile]{gregoriotex}
%\usepackage{gregoriotex}

\definecolor{gregoriocolor}{RGB}{215,65,29}

%% style général de gregorio :
% lignes rouges, commenter pour du noir
%\gresetlinecolor{gregoriocolor}

% texte <alt> (au-dessus de la portée) en rouge et en petit, avec réglage de sa position verticale
\grechangestyle{abovelinestext}{\color{gregoriocolor}\footnotesize}
\newcommand{\altraise}{-2.4mm}
\grechangedim{abovelinestextraise}{\altraise}{scalable}

% taille des initiales
\newcommand{\initialsize}[1]{
    \grechangestyle{initial}{\fontspec{ZallmanCaps}\fontsize{#1}{#1}\selectfont}
}
\newcommand{\defaultinitialsize}{32}
\initialsize{\defaultinitialsize}
% espace avant et après les initiales
\newcommand{\initialspace}[1]{
  \grechangedim{afterinitialshift}{#1}{scalable}
  \grechangedim{beforeinitialshift}{#1}{scalable}
}
\newcommand{\defaultinitialspace}{0cm}
\initialspace{\defaultinitialspace}


% on définit le système qui capture des headers pour générer des annotations
% cette commande sera appelée pour définir des abréviations ou autres substitutions
\newcommand{\resultat}{}
\newcommand{\abbrev}[3]{
  \IfSubStr{#1}{#2}{ \renewcommand{\resultat}{#3} }{}
}
\newcommand{\officepartannotation}[1]{
  \renewcommand{\resultat}{#1}
  \abbrev{#1}{ntro}{ {Intr.} }
  \abbrev{#1}{espo}{Resp.}
  \abbrev{#1}{adu}{Gr.}
  \abbrev{#1}{ll}{All.}
  \abbrev{#1}{act}{Tract.}
  \abbrev{#1}{equen}{Seq.}
  \abbrev{#1}{ffert}{Off.}
  \abbrev{#1}{ommun}{Co.}
  \abbrev{#1}{ntip}{Ant.}
  \abbrev{#1}{ntic}{Cant.}
  \abbrev{#1}{ymn}{Hy.}
  \abbrev{#1}{salm}{}
  \abbrev{#1}{Toni Communes}{}
  \abbrev{#1}{yrial}{}
  \greannotation{\resultat}
}
\newcommand{\modeannotation}[1]{
  \renewcommand{\resultat}{#1}
  \abbrev{#1}{1}{ {\sc i} }
  \abbrev{#1}{2}{ {\sc ii} }
  \abbrev{#1}{3}{ {\sc iii} }
  \abbrev{#1}{4}{ {\sc iv} }
  \abbrev{#1}{5}{ {\sc v} }
  \abbrev{#1}{6}{ {\sc vi} }
  \abbrev{#1}{7}{ {\sc vii} }
  \abbrev{#1}{8}{ {\sc viii} }
  \greannotation{\resultat}
}
\gresetheadercapture{office-part}{officepartannotation}{}
\gresetheadercapture{mode}{modeannotation}{string}

%%%%%%%%%%%%%%%%%%%%%%%%%%%%%%%%%%%%%%%%%%%%%% Graphisme %%%%%%%%%%%%%%%%%%%%%%%%%%%
% on définit l'échelle générale

\newcommand{\echelle}{1}

% on centre les titres et on ne les numérote pas
\titleformat{\section}[block]{\Large\filcenter\sc}{}{}{}
\titleformat{\subsection}[block]{\large\filcenter\sc}{}{}{}
\titleformat{\paragraph}[block]{\filcenter\sc}{}{}{}
\setcounter{secnumdepth}{0}
% on diminue l'espace avant les titres
\titlespacing*{\paragraph}{0pt}{1.8ex plus .4ex minus .4ex}{1.2ex plus .2ex minus .2ex}

% commandes versets, repons et croix
\newcommand{\vv}{\textcolor{gregoriocolor}{\fontspec[Scale=\echelle]{Charis SIL}℣.\hspace{3mm}}}
\newcommand{\rr}{\textcolor{gregoriocolor}{\fontspec[Scale=\echelle]{Charis SIL}℟.\hspace{3mm}}}
\newcommand{\cc}{\textcolor{gregoriocolor}{\fontspec[Scale=\echelle]{FreeSerif}\symbol{"2720}~}}
\renewcommand{\aa}{\textcolor{gregoriocolor}{\fontspec[Scale=\echelle]{Charis SIL}\Abar.\hspace{3mm}}}

% commandes diverses
\newcommand{\antiphona}{\textcolor{gregoriocolor}{\noindent Antiphona.\hspace{4mm}}}
\newcommand{\antienne}{\textcolor{gregoriocolor}{\noindent Antienne.\hspace{4mm}}}
% rubrique
\newcommand{\rubrique}[1]{\textcolor{gregoriocolor}{\emph{#1}}}
% pour afficher du texte noir roman au milieu d'une rubrique
\newcommand{\normaltext}[1]{{\normalfont\normalcolor #1}}

\newcommand{\saut}{\hspace{1cm}}
\newcommand{\capsaut}{\hspace{3mm}}
% pour affichier 1 en rouge et un peu d'espace
\newcommand{\un}{{\color{gregoriocolor} 1~~~}}


% abréviations
\newcommand{\tpalleluia}{\rubrique{(T.P.} \mbox{Allelúia.\rubrique{)}}}
\newcommand{\tpalleluiafr}{\rubrique{(T.P.} \mbox{Alléluia.\rubrique{)}}}

\newcommand{\tqomittitur}{{\small \rubrique{(In Tempore Quadragesimæ ommittitur} Allelúia.\rubrique{)}}}
\newcommand{\careme}{{\small \rubrique{(Pendant le Carême on omet l'}Alléluia.\rubrique{)}}}

% environnement hymne : alltt + normalfont + marges custom
\newenvironment{hymne}
  {
  \begin{adjustwidth}{1.6cm}{1mm}
  \begin{alltt}\normalfont
  }
  {
  \end{alltt}
  \end{adjustwidth}
  }
  
% la commande \u permet de souligner les inflexions
\let\u\textbf

% on définit la police par défaut
\setmainfont[Ligatures=TeX, Scale=\echelle]{Charis SIL}
%renderer=ICU a l'air de ne plus marcher...
%\setmainfont[Renderer=ICU, Ligatures=TeX, Scale=\echelle]{Charis SIL}
\setstretch{0.9}


%%%%%%%%%%%%%%%% Commandes de mise en forme %%%%%%%%%%%%%%%%

\newcommand{\lectioresponsorium}[8]{
	\needspace{4\baselineskip}
	\paragraph{#1}

	\vv Jube, domne, benedícere.\\
	\vv #3.\\
	\rr Amen.

	#5

	\vv Tu autem, Dómine, miserére nobis.\\
	\rr Deo grátias.

	\ifblank{#7}{}{\gregorioscore{gabc/#7}}

	\switchcolumn
	\needspace{4\baselineskip}
	\paragraph{#2}

	\vv Veuillez, maître, bénir.\\
	\vv #4.\\
	\rr Amen.

	#6

	\vv Et toi Seigneur, prends pitié de nous.\\
	\rr Nous rendons grâces à Dieu.

	\ifblank{#8}{}{#8}

	\switchcolumn*

}

\newcommand{\versiculus}[2]{
	\vv #1.\\
	\rr #2.\\
}

\newcommand{\versiculusabsolutio}[6]{

	\paragraph{Versiculus et Absolutio}

	\versiculus{#1}{#2}
	\vv Pater noster... \rubrique{(secrètement)} Et ne nos indúcas in tentatiónem. \\
	\rr Sed líbera nos a malo. \\
	\vv #3. \rr Amen.

	\switchcolumn

	\paragraph{Versicule et Absolution}

	\versiculus{#4}{#5}
	\vv Notre Père... Et ne nous laisse pas entrer en tentation. \\
	\rr Mais délivre-nous du mal. \\
	\vv #6. \rr Amen.

	\switchcolumn*

}

\newcommand{\versetGloireAuPere}{
	\vv Gloire au Père, au Fils, et au Saint-Esprit.
}

\newcommand{\gscore}[1]{
	\gregorioscore{gabc/#1}
}

\newcommand{\smallscore}[1]{
	\gresetinitiallines{0}
	\gscore{#1}
	\gresetinitiallines{1}
}