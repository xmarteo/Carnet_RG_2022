\documentclass[twoside]{article}


\usepackage[paperwidth=150mm, paperheight=230mm]{geometry}
\usepackage{fontspec}
%\usepackage[latin1]{inputenc}
\usepackage[french]{babel}
\usepackage[strict]{changepage}
\usepackage{fancyhdr}
\usepackage{paracol}
\usepackage{tableof}
\usepackage{setspace}
\usepackage{alltt}
\usepackage{titlesec}
\usepackage{xcolor}
\usepackage{xstring}
\usepackage{parskip}
\usepackage{enumitem}
\usepackage{etoolbox}
\usepackage{needspace}

%%%%%%%%%%%%%%%%%%%%%%%%%%%%%%%%%%%%%%%%%%%%%%%%%%% Mise en page %%%%%%%%%%%%%%%%%%%%%%%%%%%%%%
% on numérote les nbp par page et non globalement
\usepackage[perpage]{footmisc}

% définition des en-têtes et pieds de page
\pagestyle{empty}
\fancyhead{}
\fancyfoot{}
\renewcommand{\headrulewidth}{0pt}
\setlength{\headheight}{0pt}

% la commande titres permet de changer les titres de gauche et de droite.
\newcommand{\titres}[2]{
	\renewcommand{\rightmark}{\textcolor{red}{\sc #2}}
	\renewcommand{\leftmark}{\textcolor{red}{\sc #1}}
}
\titres{}{}

% pas d'indentation
\setlength{\parindent}{0mm}

\geometry{
inner=20mm,
outer=20mm,
top=10mm,
bottom=25mm,
headsep=0mm,
}

\twosided[p]

%%%%%%%%%%%%%%%%%%%%%%%%%%%%%%%%%%%%%%%%%%%%%%%%% Options gregorio %%%%%%%%%%%%%%%%%%%%%%%%%

\usepackage[autocompile]{gregoriotex}
%\usepackage{gregoriotex}

\definecolor{gregoriocolor}{RGB}{215,65,29}

%% style général de gregorio :
% lignes rouges, commenter pour du noir
%\gresetlinecolor{gregoriocolor}

% texte <alt> (au-dessus de la portée) en rouge et en petit, avec réglage de sa position verticale
\grechangestyle{abovelinestext}{\color{gregoriocolor}\footnotesize}
\newcommand{\altraise}{-1mm}
\grechangedim{abovelinestextraise}{\altraise}{scalable}

% taille des initiales
\newcommand{\initialsize}[1]{
    \grechangestyle{initial}{\fontspec{ZallmanCaps}\fontsize{#1}{#1}\selectfont}
}
\newcommand{\defaultinitialsize}{32}
\initialsize{\defaultinitialsize}
% espace avant et après les initiales
\newcommand{\initialspace}[1]{
  \grechangedim{afterinitialshift}{#1}{scalable}
  \grechangedim{beforeinitialshift}{#1}{scalable}
}
\newcommand{\defaultinitialspace}{0cm}
\initialspace{\defaultinitialspace}


% on définit le système qui capture des headers pour générer des annotations
% cette commande sera appelée pour définir des abréviations ou autres substitutions
\newcommand{\resultat}{}
\newcommand{\abbrev}[3]{
  \IfSubStr{#1}{#2}{ \renewcommand{\resultat}{#3} }{}
}
\newcommand{\officepartannotation}[1]{
  \renewcommand{\resultat}{#1}
  \abbrev{#1}{ntro}{Intr.}
  \abbrev{#1}{re}{Resp.}
  \abbrev{#1}{espo}{Resp.}
  \abbrev{#1}{adu}{Gr.}
  \abbrev{#1}{ll}{All.}
  \abbrev{#1}{act}{Tract.}
  \abbrev{#1}{equen}{Seq.}
  \abbrev{#1}{ffert}{Off.}
  \abbrev{#1}{ommun}{Co.}
  \abbrev{#1}{an}{Ant.}
  \abbrev{#1}{ntiph}{Ant.}
  \abbrev{#1}{ntic}{Cant.}
  \abbrev{#1}{ymn}{Hy.}
  \abbrev{#1}{salm}{}
  \abbrev{#1}{Toni Communes}{}
  \abbrev{#1}{yrial}{}
  \greannotation{\resultat}
}
\newcommand{\modeannotation}[1]{
  \renewcommand{\resultat}{#1}
  \abbrev{#1}{1}{ {\sc i} }
  \abbrev{#1}{2}{ {\sc ii} }
  \abbrev{#1}{3}{ {\sc iii} }
  \abbrev{#1}{4}{ {\sc iv} }
  \abbrev{#1}{5}{ {\sc v} }
  \abbrev{#1}{6}{ {\sc vi} }
  \abbrev{#1}{7}{ {\sc vii} }
  \abbrev{#1}{8}{ {\sc viii} }
  \greannotation{\resultat}
}
\gresetheadercapture{office-part}{officepartannotation}{}
\gresetheadercapture{mode}{modeannotation}{string}

%%%%%%%%%%%%%%%%%%%%%%%%%%%%%%%%%%%%%%%%%%%%%% Graphisme %%%%%%%%%%%%%%%%%%%%%%%%%%%
% on définit l'échelle générale

\newcommand{\echelle}{1}

% on centre les titres et on ne les numérote pas
\titleformat{\section}[block]{\Large\filcenter\sc}{}{}{}
\titleformat{\subsection}[block]{\large\filcenter\sc}{}{}{}
\titleformat{\paragraph}[block]{\filcenter\sc}{}{}{}
\setcounter{secnumdepth}{0}
% on diminue l'espace avant les titres
\titlespacing*{\paragraph}{0pt}{1.8ex plus .4ex minus .4ex}{1.2ex plus .2ex minus .2ex}

% commandes versets, repons et croix
\newcommand{\vv}{\textcolor{gregoriocolor}{\fontspec[Scale=\echelle]{Charis SIL}℣.\hspace{3mm}}}
\newcommand{\rr}{\textcolor{gregoriocolor}{\fontspec[Scale=\echelle]{Charis SIL}℟.\hspace{3mm}}}
\newcommand{\cc}{\textcolor{gregoriocolor}{\fontspec[Scale=\echelle]{FreeSerif}\symbol{"2720}~}}
\renewcommand{\aa}{\textcolor{gregoriocolor}{\fontspec[Scale=\echelle]{Charis SIL}\Abar.\hspace{3mm}}}

% commandes diverses
\newcommand{\antiphona}{\textcolor{gregoriocolor}{\noindent Antiphona.\hspace{4mm}}}
\newcommand{\antienne}{\textcolor{gregoriocolor}{\noindent Antienne.\hspace{4mm}}}
% rubrique
\newcommand{\rubrique}[1]{\textcolor{gregoriocolor}{\emph{#1}}}
% pour afficher du texte noir roman au milieu d'une rubrique
\newcommand{\normaltext}[1]{{\normalfont\normalcolor #1}}

\newcommand{\saut}{\hspace{1cm}}
\newcommand{\capsaut}{\hspace{3mm}}
% pour affichier 1 en rouge et un peu d'espace
\newcommand{\un}{{\color{gregoriocolor} 1~~~}}


% abréviations
\newcommand{\tpalleluia}{\rubrique{(T.P.} \mbox{Allelúia.\rubrique{)}}}
\newcommand{\tpalleluiafr}{\rubrique{(T.P.} \mbox{Alléluia.\rubrique{)}}}

\newcommand{\tqomittitur}{{\small \rubrique{(In Tempore Quadragesimæ ommittitur} Allelúia.\rubrique{)}}}
\newcommand{\careme}{{\small \rubrique{(Pendant le Carême on omet l'}Alléluia.\rubrique{)}}}

% environnement hymne : alltt + normalfont + marges custom
\newenvironment{hymne}
  {
  \begin{adjustwidth}{1.6cm}{1mm}
  \begin{alltt}\normalfont
  }
  {
  \end{alltt}
  \end{adjustwidth}
  }
  
% la commande \u permet de souligner les inflexions
\let\u\textbf

% on définit la police par défaut
\setmainfont[Ligatures=TeX, Scale=\echelle]{Charis SIL}
%renderer=ICU a l'air de ne plus marcher...
%\setmainfont[Renderer=ICU, Ligatures=TeX, Scale=\echelle]{Charis SIL}
\setstretch{0.9}


%%%%%%%%%%%%%%%% Commandes de mise en forme %%%%%%%%%%%%%%%%

\newcommand{\lectioresponsorium}[8]{
	\needspace{3\baselineskip}
	\paragraph{#1}

	\vv Jube, domne, benedícere.\\
	\vv #3.\\
	\rr Amen.

	#5

	\vv Tu autem, Dómine, miserére nobis.\\
	\rr Deo grátias.

	\ifblank{#7}{}{\gregorioscore{gabc/#7}}

	\switchcolumn
	%\needspace{3\baselineskip}
	\paragraph{#2}

	\vv Veuillez, maître, bénir.\\
	\vv #4.\\
	\rr Amen.

	#6

	\vv Et toi Seigneur, prends pitié de nous.\\
	\rr Nous rendons grâces à Dieu.

	\ifblank{#8}{}{#8}

	\switchcolumn*

}

\newcommand{\versiculus}[2]{
	\vv #1.\\
	\rr #2.\\
}

\newcommand{\versiculusabsolutio}[6]{

	\paragraph{Versiculus et Absolutio}

	\versiculus{#1}{#2}
	\vv Pater noster... \rubrique{(secrètement)} Et ne nos indúcas in tentatiónem. \\
	\rr Sed líbera nos a malo. \\
	\vv #3. \rr Amen.

	\switchcolumn

	\paragraph{Versicule et Absolution}

	\versiculus{#4}{#5}
	\vv Notre Père... Et ne nous laisse pas entrer en tentation. \\
	\rr Mais délivre-nous du mal. \\
	\vv #6. \rr Amen.

	\switchcolumn*

}

\newcommand{\versetGloireAuPere}{
	\vv Gloire au Père, au Fils, et au Saint-Esprit.
}

\newcommand{\gscore}[1]{
	\gregorioscore{gabc/#1}
}

\newcommand{\smallscore}[1]{
	\gresetinitiallines{0}
	\gscore{#1}
	\gresetinitiallines{1}
}

\begin{document}

\null \newpage

\sloppy

\begin{paracol}[1]{2}

\begin{center}\begin{doublespace}

{\fontspec[Scale=\echelle]{Futura Book}
\MakeUppercase{\Large Dominica IV Octobris \\ ad Matutinum}\\
juxta usum antiquiorem ritus romani}
\end{doublespace}\end{center}
\selectlanguage{french}

\smallscore{domine_labia_mea}

~~

\switchcolumn

\begin{center}\begin{doublespace}
{\fontspec[Scale=\echelle]{Futura Book}
\MakeUppercase{\Large Matines du 20\ieme~dimanche \\ après la Pentecôte, 4\ieme~d'octobre}\\
selon l'usage ancien du rite romain
}
\end{doublespace}\end{center}

~~

~~

\vv Seigneur, ouvre mes lèvres.

\rr Et ma bouche annoncera ta louange.

\vv Dieu, viens à mon aide.

\rr Seigneur, viens vite à mon secours.

\vv Gloire au Père, au Fils, et au Saint-Esprit.

\rr Comme il était au commencement, maintenant et toujours, et dans les siècles des siècles. Amen. Alléluia.

\switchcolumn*

\paragraph{Invitatorium}

\gscore{i}
\gscore{ip}

\switchcolumn

\paragraph{Invitatoire}

\aa Adorons le Seigneur, \GreSpecial{*} Car c'est lui qui nous a faits.\\
\aa Adorons le Seigneur, \GreSpecial{*} Car c'est lui qui nous a faits.\\
\vv Venez, chantons avec allégresse au Seigneur, jubilons pour Dieu, notre salut. Hâtons-nous de nous présenter devant lui avec des louanges et, dans des psaumes célébrons sa gloire.\\
\newpage
\aa Adorons le Seigneur, \GreSpecial{*} Car c'est lui qui nous a faits.\\
\vv Parce que le Seigneur est le grand Dieu; le grand Roi au dessus de tous les dieux; parce que le Seigneur ne repoussera pas son peuple; parce que dans sa main sont tous les confins de la terre et que son regard domine les cimes des montagnes.\\
\GreSpecial{*} Car c'est lui qui nous a faits.\\
\vv Parce qu'à lui est la mer, et que c'est lui-même qui l'a faite, et que ses mains ont formé le continent. \rubrique{(À genoux)} Venez, adorons, prosternons-nous devant Dieu, et pleurons devant le Seigneur qui nous a faits, \rubrique{(Debout)} parce que lui-même est le Seigneur notre Dieu, et que nous sommes son peuple et les brebis de son pâturage.\\
\newpage
\aa Adorons le Seigneur, \GreSpecial{*} Car c'est lui qui nous a faits.\\
\vv Aujourd'hui, si vous entendez sa voix, n'endurcissez pas vos cœurs, comme il arriva à vos pères dans l'exaspération au jour de la tentation dans le désert, alors qu'ils me tentèrent, m'éprouvèrent et virent mes œuvres.\\
\GreSpecial{*} Car c'est lui qui nous a faits.\\
\vv Pendant quarante ans, j'ai été proche de cette génération et j'ai dit : toujours ils errent de cœur ; et eux, ils n'ont point connu mes voies : et Je leur ai juré dans ma colère, s'ils entreront dans mon repos.\\
\aa Adorons le Seigneur, \GreSpecial{*} Car c'est lui qui nous a faits.\\
\vv Gloire au Père, au Fils, et au Saint-Esprit, comme il était au commencement, maintenant et toujours, et dans les siècles des siècles. Amen.\\
\GreSpecial{*} Car c'est lui qui nous a faits.\\
\aa Adorons le Seigneur, \GreSpecial{*} Car c'est lui qui nous a faits.\\

\switchcolumn*

\paragraph{Hymnus}

\gscore{h}

\switchcolumn

\paragraph{Hymne}

En ce premier de tous les jours \\
Où paraît le monde créé,\\
Le Créateur ressuscité\\
Vainqueur de la mort nous libère.\\
\\
Bannissons loin de nous la tiédeur,\\
Levons-nous tous, levons-nous sans retard,\\
Du sein de la nuit, cherchons le Seigneur,\\
Qu'il nous enseigne comme pour le Prophète (David).\\
\\
Dieu entendra notre prière,\\
Il nous tendra une main secourable,\\
Purifiera notre âme des souillures\\
Et nous rendra nos droits au Paradis.\\
\\
Nous qui venons,\\
En cette très sainte partie du jour,\\
Chanter nos cantiques, durant les heures du repos,\\
Nous aurons part aux récompenses éternelles.\\
\newpage
\null\vfill
Ô Jésus, splendeur du Père,\\
Nous t'en supplions instamment,\\
Éteins en nous la flamme des passions,\\
Et garde-nous de toute action coupable.\\
\\
Garde nos corps et nos âmes\\
Du souffle impur de la concupiscence,\\
C'est à cause de ses feux,\\
Que les feux de l'enfer brûlent avec plus d'ardeur.\\
\\
O Rédempteur du monde, nous t'en supplions\\
Purifie-nous, lave-nous de nos crimes,\\
Et dans ta miséricorde,\\
Accorde-nous les biens de l'éternelle vie.\\
\\
Exauce-nous, Père très miséricordieux,\\
Fils unique égal au Père,\\
Et toi, Esprit Paraclet,\\
Qui règnes dans tous les siècles.
\vfill
\newpage

\switchcolumn*

\subsection{In I Nocturno}

\switchcolumn

\subsection{Premier Nocturne}

\switchcolumn*

\paragraph{Psalmus 1}

\gscore{a1_8g}

\begin{enumerate}[wide, itemsep=0mm, labelwidth=!, labelindent=0pt, label=\color{gregoriocolor}\theenumi]
\item Beátus vir, qui non ábiit in consílio impiórum,~† et in via peccatórum non \textbf{ste}tit,~* et in cáthedra pestilénti\textit{æ} \textit{non} \textbf{se}dit:
\item Sed in lege Dómini volúntas \textbf{e}jus,~* et in lege ejus meditábitur di\textit{e} \textit{ac} \textbf{noc}te.
\item Et erit tamquam lignum, quod plantátum est secus decúrsus a\textbf{quá}rum,~* quod fructum suum dabit in tém\textit{po}\textit{re} \textbf{su}o:
\item Et fólium ejus non \textbf{dé}fluet:~* et ómnia quæcúmque fáciet, pro\textit{spe}\textit{ra}\textbf{bún}tur.
\item Non sic ímpii, \textbf{non} sic:~* sed tamquam pulvis, quem prójicit ventus a fá\textit{ci}\textit{e} \textbf{ter}ræ.
\item Ideo non resúrgent ímpii in ju\textbf{dí}cio:~* neque peccatóres in concíli\textit{o} \textit{jus}\textbf{tó}rum.
\item Quóniam novit Dóminus viam jus\textbf{tó}rum:~* et iter impió\textit{rum} \textit{per}\textbf{í}bit.
\item Glória Patri, et \textbf{Fí}lio,~* et Spirí\textit{tu}\textit{i} \textbf{Sanc}to.
\item Sicut erat in princípio, et nunc, et \textbf{sem}per,~* et in sǽcula sæcu\textit{ló}\textit{rum}. \textbf{A}men.
\end{enumerate}

\switchcolumn

\paragraph{Psaume 1}
\aa Bienheureux l'homme qui médite la loi du Seigneur.

\gscore{8_G}

\begin{enumerate}[wide, itemsep=0mm, labelwidth=!, labelindent=0pt, label=\color{gregoriocolor}\theenumi]
\item Heureux est l'homme qui n'entre pas au conseil des méchants, + qui ne suit pas le chemin des pécheurs, * ne siège pas avec ceux qui ricanent,
\item mais se plaît dans la loi du Seigneur * et murmure sa loi jour et nuit !
\item Il est comme un arbre planté près d'un ruisseau, * qui donne du fruit en son temps,
\item et jamais son feuillage ne meurt ; * tout ce qu'il entreprend réussira,
\item tel n'est pas le sort des méchants. * Mais ils sont comme la paille balayée par le vent :
\item au jugement, les méchants ne se lèveront pas, * ni les pécheurs au rassemblement des justes.
\item Le Seigneur connaît le chemin des justes, * mais le chemin des méchants se perdra.
\end{enumerate}

\switchcolumn*
\paragraph{Psalmus 2}

\gscore{a2_7a}

\begin{enumerate}[wide, itemsep=0mm, labelwidth=!, labelindent=0pt, label=\color{gregoriocolor}\theenumi]
\item Quare fremu\textbf{é}runt \textbf{Gen}tes:~* et pópuli meditáti \textbf{sunt} in\textbf{á}nia?
\item Astitérunt reges terræ, et príncipes conve\textbf{né}runt in \textbf{u}num~* advérsus Dóminum, et advérsus \textbf{Chris}tum \textbf{e}jus.
\item Dirumpámus víncu\textbf{la} e\textbf{ó}rum:~* et projiciámus a nobis \textbf{ju}gum ip\textbf{só}rum.
\item Qui hábitat in cælis, irri\textbf{dé}bit \textbf{e}os:~* et Dóminus subsan\textbf{ná}bit \textbf{e}os.
\item Tunc loquétur ad eos in \textbf{i}ra \textbf{su}a,~* et in furóre suo contur\textbf{bá}bit \textbf{e}os.
\item Ego autem constitútus sum Rex ab eo super Sion montem \textbf{sanc}tum \textbf{e}jus,~* prǽdicans præ\textbf{cép}tum \textbf{e}jus.
\item Dóminus \textbf{di}xit \textbf{ad} me:~* Fílius meus es tu, ego hódie \textbf{gé}nu\textbf{i} te.
\item Póstula a me, et dabo tibi Gentes heredi\textbf{tá}tem \textbf{tu}am,~* et possessiónem tuam \textbf{tér}minos \textbf{ter}ræ.
\item Reges eos in \textbf{vir}ga \textbf{fér}rea,~* et tamquam vas fíguli con\textbf{frín}ges \textbf{e}os.
\item Et nunc, reges, \textbf{in}tel\textbf{lí}gite:~* erudímini, qui judi\textbf{cá}tis \textbf{ter}ram.
\item Servíte Dómino \textbf{in} ti\textbf{mó}re:~* et exsultáte ei \textbf{cum} tre\textbf{mó}re.
\item Apprehéndite disciplínam, nequándo iras\textbf{cá}tur \textbf{Dó}minus,~* et pereátis de \textbf{vi}a \textbf{jus}ta.
\item Cum exárserit in brevi \textbf{i}ra \textbf{e}jus:~* beáti omnes qui con\textbf{fí}dunt in \textbf{e}o.
\item Glória \textbf{Pa}tri, et \textbf{Fí}lio,~* et Spi\textbf{rí}tui \textbf{Sanc}to.
\item Sicut erat in princípio, et \textbf{nunc}, et \textbf{sem}per,~* et in sǽcula sæcu\textbf{ló}rum. \textbf{A}men.
\end{enumerate}

\switchcolumn

\paragraph{Psaume 2}
\aa Servez le Seigneur dans la crainte, et exultez devant lui avec tremblement.

\gscore{7_a}


\begin{enumerate}[wide, itemsep=0mm, labelwidth=!, labelindent=0pt, label=\color{gregoriocolor}\theenumi]
\item Pourquoi ce tumulte des nations, ce vain murmure des peuples ?
\item Les rois de la terre se dressent, les grands se liguent entre eux contre le Seigneur et son messie :
\item « Faisons sauter nos chaînes, rejetons ces entraves ! »
\item Celui qui règne dans les cieux s'en amuse, le Seigneur les tourne en dérision ;
\item puis il leur parle avec fureur , et sa colère les épouvante :
\item « Moi, j'ai sacré mon roi sur Sion, ma sainte montagne. »
\item Je proclame le décret du Seigneur ! + Il m'a dit : « Tu es mon fils ; moi, aujourd'hui, je t'ai engendré.
\item Demande, et je te donne en héritage les nations, pour domaine la terre tout entière.
\item Tu les détruiras de ton sceptre de fer, tu les briseras comme un vase de potier. »
\item Maintenant, rois, comprenez, reprenez-vous, juges de la terre.
\item Servez le Seigneur avec crainte, rendez-lui votre hommage en tremblant.
\item Qu'il s'irrite et vous êtes perdus : soudain sa colère éclatera. Heureux qui trouve en lui son refuge !
\end{enumerate}

\switchcolumn*

\paragraph{Psalmus 3}

\gscore{a3_6f}

\begin{enumerate}[wide, itemsep=0mm, labelwidth=!, labelindent=0pt, label=\color{gregoriocolor}\theenumi]
\item Dómine quid multiplicáti sunt qui \textbf{trí}bu\textbf{lant} me?~* multi insúr\textit{gunt} \textit{ad}\textbf{vér}sum me.
\item Multi dicunt \textbf{á}nimæ \textbf{me}æ:~* Non est salus ipsi in \textit{De}\textit{o} \textbf{e}jus.
\item Tu autem, Dómine, su\textbf{scép}tor \textbf{me}us es,~* glória mea, et exáltans \textit{ca}\textit{put} \textbf{me}um.
\item Voce mea ad Dómi\textbf{num} cla\textbf{má}vi:~* et exaudívit me de monte \textit{sanc}\textit{to} \textbf{su}o.
\item Ego dormívi, et \textbf{so}po\textbf{rá}tus sum:~* et exsurréxi, quia Dómi\textit{nus} \textit{su}\textbf{scé}pit me.
\item Non timébo míllia pópuli \textbf{cir}cum\textbf{dán}tis me:~* exsúrge, Dómine, salvum me fac, \textit{De}\textit{us} \textbf{me}us.
\item Quóniam tu percussísti omnes adversántes mihi \textbf{si}ne \textbf{cau}sa:~* dentes peccatórum \textit{con}\textit{tri}\textbf{vís}ti.
\item Dómi\textbf{ni} est \textbf{sa}lus:~* et super pópulum tuum benedíc\textit{ti}\textit{o} \textbf{tu}a.
\item Glória \textbf{Pa}tri, et \textbf{Fí}lio,~* et Spirí\textit{tu}\textit{i} \textbf{Sanc}to.
\item Sicut erat in princípio, et \textbf{nunc}, et \textbf{sem}per,~* et in sǽcula sæcu\textit{ló}\textit{rum}. \textbf{A}men.
\end{enumerate}

\switchcolumn

\paragraph{Psaume 3}
\aa Lève-toi, Seigneur, sauve-moi, mon Dieu.

\gscore{6_lib}

\begin{enumerate}[wide, itemsep=0mm, labelwidth=!, labelindent=0pt, label=\color{gregoriocolor}\theenumi]
\item Seigneur, qu'ils sont nombreux mes adversaires, nombreux à se lever contre moi,
\item nombreux à déclarer à mon sujet : « Pour lui, pas de salut auprès de Dieu ! »
\item Mais toi, Seigneur, mon bouclier, ma gloire, tu tiens haute ma tête.
\item À pleine voix je crie vers le Seigneur ; il me répond de sa montagne sainte.
\item Et moi, je me couche et je dors ; je m'éveille : le Seigneur est mon soutien.
\item Je ne crains pas ce peuple nombreux qui me cerne et s'avance contre moi.
\item Lève-toi, Seigneur ! Sauve-moi, mon Dieu ! Tous mes ennemis, tu les frappes à la mâchoire ; les méchants, tu leur brises les dents.
\item Du Seigneur vient le salut ; vienne ta bénédiction sur ton peuple !
\end{enumerate}

\switchcolumn*

\versiculusabsolutio
	{Ipse liberávit me de láqueo venántium}
	{Et a verbo áspero}
	{Exáudi, Dómine Iesu Christe, preces servórum tuórum, et miserére nobis: Qui cum Patre et Spíritu Sancto vivis et regnas in sǽcula sæculórum}
	{C’est lui qui m’a délivré du lacet des chasseurs}
	{Et de l’affaire de ruine}
	{Exaucez, Seigneur Jésus-Christ, les prières de vos serviteurs, et ayez pitié de nous, vous qui vivez et régnez avec le Père et le Saint-Esprit, dans les siècles des siècles}

\lectioresponsorium
	{Lectio \textsc{i}}
	{Première lecture}
	{Benedictióne perpétua benedícat nos Pater ætérnus}
	{Que le Père éternel nous bénisse d'une bénédiction perpétuelle}
	{
		Incipit liber secúndus Machabæórum

		Frátribus qui sunt per Ægýptum Judǽis, salútem dicunt fratres qui sunt in Jerosólymis Judǽi, et qui in regióne Judǽæ, et pacem bonam.
		Benefáciat vobis Deus, et memínerit testaménti sui, quod locútus est ad Abraham, et Isaac, et Jacob servórum suórum fidélium:
		Et det vobis cor ómnibus ut colátis eum, et faciátis ejus voluntátem, corde magno et ánimo volénti.
		Adapériat cor vestrum in lege sua, et in præcéptis suis, et fáciat pacem;
		Exáudiat oratiónes vestras, et reconciliétur vobis, nec vos déserat in témpore malo.
		Et nunc hic sumus orántes pro vobis.
	}
	{	\rubrique{2 Mac 1 : 1-6}
	
		Commencement du Deuxième Livre des Martyrs d'Israël
		
		Aux frères juifs qui sont en Égypte, salut ! Leurs frères juifs qui sont à Jérusalem et dans le pays de Judée leur souhaitent paix et prospérité.
		Que Dieu vous comble de bienfaits ; qu’il se souvienne de son alliance en faveur d’Abraham, d’Isaac et de Jacob, ses fidèles serviteurs !
		Qu’il vous donne à tous un cœur pour l’adorer, pour accomplir ses volontés généreusement et de plein gré !
		Qu’il ouvre votre cœur à sa Loi et à ses décrets ; qu’il établisse la paix !
		Qu’il exauce vos demandes, se réconcilie avec vous et ne vous délaisse pas au temps du malheur !
		Telle est la prière que nous formulons pour vous ici, en ce moment.
	}
	{r1}
	{\vfill
	\rr Que le Seigneur ouvre votre cœur à sa Loi et à ses décrets ; qu’il établisse la paix en vos jours.\\
	\GreSpecial{*} Qu'il vous donne le salut et vous rachète de vos péchés.\\
	\vv Qu’il exauce vos demandes, se réconcilie avec vous et ne vous délaisse pas au temps du malheur.\\
	\GreSpecial{*} Qu'il vous donne le salut et vous rachète de vos péchés.
	\vfill}

\lectioresponsorium
	{Lectio \textsc{ii}}
	{Deuxième lecture}
	{Unigénitus Dei Fílius nos benedícere et adjuváre dignétur}
	{Que le Père éternel nous bénisse d'une bénédiction perpétuelle}
	{
		Factúri ígitur quinta et vigésima die mensis Cásleu, purificatiónem templi, necessárium dúximus significáre vobis: ut et vos quoque agátis diem scenopégiæ, et diem ignis, qui datus est quando Nehemías, ædificáto templo et altári, óbtulit sacrifícia.
		Nam, cum in Pérsidem duceréntur patres nostri, sacerdótes qui tunc cultóres Dei erant, accéptum ignem de altári occúlte abscondérunt in valle, ubi erat púteus altus et siccus, et in eo contutáti sunt eum, ita ut ómnibus ignótus esset locus.
	}
	{	\rubrique{2 Mac 1 : 18-19}
	
		Comme nous allons bientôt célébrer la purification du Temple, le vingt-cinq du mois de Kisléou, nous avons estimé devoir vous en informer, 
		afin que vous la célébriez, vous aussi, à la manière de la fête des Tentes, et en souvenir du feu qui se manifesta quand Néhémie, 
		après avoir rebâti le Temple et l’autel, offrit des sacrifices.
		En effet, lorsque nos pères furent emmenés en Perse, les prêtres d’alors, remplis de piété,
		prirent du feu de l’autel et le cachèrent secrètement dans la cavité d’un puits qui se trouvait à sec.
		Ils l’y mirent en sécurité de manière à ce que l’endroit demeure ignoré de tous.
	}
	{r2}
	{\vfill
	\rr Que le Seigneur exauce vos demandes, se réconcilie avec vous et ne vous délaisse pas au temps du malheur,\\
	\GreSpecial{*} C'est le Seigneur notre Dieu.\\
	\vv Qu’il vous donne à tous un cœur pour l’adorer, pour accomplir ses volontés.\\
	\GreSpecial{*} C'est le Seigneur notre Dieu.}

\lectioresponsorium
	{Lectio \textsc{iii}}
	{Troisième lecture}
	{Spíritus Sancti grátia illúminet sensus et corda nostra}
	{Que la grâce du Saint-Esprit illumine nos esprits et nos cœurs}
	{
		Cum autem præteríssent anni multi, et plácuit Deo ut mitterétur Nehemías a rege Pérsidis, 
		nepótes sacerdótum illórum, qui abscónderant, misit ad requiréndum ignem: et, sicut narravérunt nobis, non invenérunt ignem, sed aquam crassam.
		Et jussit eos hauríre, et afférre sibi: et sacrifícia, quæ impósita erant, jussit sacérdos Nehemías aspérgi ipsa aqua: et ligna et quæ erant superpósita.
		Utque hoc factum est, et tempus áffuit quo sol refúlsit, quo prius erat in núbilo, accénsus est ignis magnus, ita ut omnes miraréntur.
	}
	{	\rubrique{2 Mac 1 : 20-22}
	
		Bien des années plus tard, au moment choisi par Dieu, Néhémie, envoyé par le roi de Perse, fit rechercher ce feu par les descendants des prêtres qui l’avaient caché. 
		Ceux-ci informèrent Néhémie qu’ils n’avaient pas trouvé de feu, mais plutôt un liquide épais, et Néhémie leur ordonna d’en puiser et d’en rapporter.
		Quand on eut tout préparé pour les sacrifices, Néhémie ordonna aux prêtres de répandre ce liquide sur le bois et sur ce que l’on y avait déposé.
		Après cela, il se passa un peu de temps. Le soleil, d’abord caché par les nuages, se mit à briller. Alors, un grand brasier s’alluma, à la stupéfaction de tous.
	}
	{r3}
	{\vfill
	\rr Nos ennemis se sont ligués et ils se glorifient de leur force ; anéantis leur courage, Seigneur, et disperse-les.\\
	\GreSpecial{*} Car il n'est autre qui combatte pour nous, sinon toi, notre Dieu.\\
	\vv Disperse-les dans ta puissance, et détruis-les, Seigneur notre protecteur.\\
	\GreSpecial{*} Car il n'est autre qui combatte pour nous, sinon toi, notre Dieu.\\
	\versetGloireAuPere{}\\
	\GreSpecial{*} Car il n'est autre qui combatte pour nous, sinon toi, notre Dieu.}

\subsection{In II Nocturno}

\switchcolumn

\subsection{Deuxième Nocturne}

\switchcolumn*

\paragraph{Psalmus 8}

\gscore{a4_1g}

\begin{enumerate}[wide, itemsep=0mm, labelwidth=!, labelindent=0pt, label=\color{gregoriocolor}\theenumi]
\item Dómine, \textit{Dó}\textit{mi}\textit{nus} \textbf{nos}ter,~* quam admirábile est nomen tuum in uni\textit{vér}\textit{sa} \textbf{ter}ra!
\item Quóniam eleváta est magnifi\textit{cén}\textit{ti}\textit{a} \textbf{tu}a,~* \textit{su}\textit{per} \textbf{cæ}los.
\item Ex ore infántium et lacténtium perfecísti laudem propter in\textit{i}\textit{mí}\textit{cos} \textbf{tu}os,~* ut déstruas inimícum \textit{et} \textit{ul}\textbf{tó}rem.
\item Quóniam vidébo cælos tuos, ópera digi\textit{tó}\textit{rum} \textit{tu}\textbf{ó}rum:~* lunam et stellas, quæ \textit{tu} \textit{fun}\textbf{dás}ti.
\item Quid est homo quod \textit{me}\textit{mor} \textit{es} \textbf{e}jus?~* aut fílius hóminis, quóniam ví\textit{si}\textit{tas} \textbf{e}um?
\item Minuísti eum paulo minus ab Angelis,~† glória et honóre co\textit{ro}\textit{nás}\textit{ti} \textbf{e}um:~* et constituísti eum super ópera mánu\textit{um} \textit{tu}\textbf{á}rum.
\item Omnia subjecísti sub \textit{pé}\textit{di}\textit{bus} \textbf{e}jus,~* oves et boves univérsas: ínsuper et pé\textit{co}\textit{ra} \textbf{cam}pi.
\item Vólucres cæli, \textit{et} \textit{pi}\textit{sces} \textbf{ma}ris,~* qui perámbulant sé\textit{mi}\textit{tas} \textbf{ma}ris.
\item Dómine, \textit{Dó}\textit{mi}\textit{nus} \textbf{nos}ter,~* quam admirábile est nomen tuum in uni\textit{vér}\textit{sa} \textbf{ter}ra!
\item Glória \textit{Pa}\textit{tri}, \textit{et} \textbf{Fí}\textbf{li}o,~* et Spirí\textit{tu}\textit{i} \textbf{Sanc}to.
\item Sicut erat in princípio, \textit{et} \textit{nunc}, \textit{et} \textbf{sem}per,~* et in sǽcula sæcu\textit{ló}\textit{rum}. \textbf{A}men.
\end{enumerate}

\switchcolumn

\paragraph{Psaume 8}
\aa Qu'il est admirable ton nom, Seigneur, par toute la terre !

\gscore{1_g}

\begin{enumerate}[wide, itemsep=0mm, labelwidth=!, labelindent=0pt, label=\color{gregoriocolor}\theenumi]
\item Ô Seigneur, notre Dieu, qu'il est grand ton nom par toute la terre ! Jusqu'aux cieux, ta splendeur est chantée
\item par la bouche des enfants, des tout-petits : rempart que tu opposes à l'adversaire, où l'ennemi se brise en sa révolte.
\item A voir ton ciel, ouvrage de tes doigts, la lune et les étoiles que tu fixas,
\item qu'est-ce que l'homme pour que tu penses à lui, le fils d'un homme, que tu en prennes souci ?
\item Tu l'as voulu un peu moindre qu'un dieu, le couronnant de gloire et d'honneur ;
\item tu l'établis sur les oeuvres de tes mains, tu mets toute chose à ses pieds :
\item les troupeaux de boeufs et de brebis, et même les bêtes sauvages,
\item les oiseaux du ciel et les poissons de la mer, tout ce qui va son chemin dans les eaux.
\item O Seigneur, notre Dieu, qu'il est grand ton nom par toute la terre !
\end{enumerate}

\switchcolumn*

\paragraph{Psalmus 9 - I}

\gscore{a5_8g}

\begin{enumerate}[wide, itemsep=0mm, labelwidth=!, labelindent=0pt, label=\color{gregoriocolor}\theenumi]
\item Confitébor tibi, Dómine, in toto corde \textbf{me}o:~* narrábo ómnia mirabí\textit{li}\textit{a} \textbf{tu}a.
\item Lætábor et exsultábo \textbf{in} te:~* psallam nómini tu\textit{o}, \textit{Al}\textbf{tís}sime.
\item In converténdo inimícum meum re\textbf{trór}sum:~* infirmabúntur, et períbunt a fá\textit{ci}\textit{e} \textbf{tu}a.
\item Quóniam fecísti judícium meum et causam \textbf{me}am:~* sedísti super thronum, qui júdi\textit{cas} \textit{jus}\textbf{tí}tiam.
\item Increpásti Gentes, et périit \textbf{ím}pius:~* nomen eórum delésti in ætérnum, et in sǽ\textit{cu}\textit{lum} \textbf{sǽ}culi.
\item Inimíci defecérunt frámeæ in \textbf{fi}nem:~* et civitátes eórum \textit{de}\textit{stru}\textbf{xís}ti.
\item Périit memória eórum cum \textbf{só}nitu:~* et Dóminus in æ\textit{tér}\textit{num} \textbf{pér}manet.
\item Parávit in judício thronum \textbf{su}um:~* et ipse judicábit orbem terræ in æquitáte, judicábit pópulos \textit{in} \textit{jus}\textbf{tí}tia.
\item Et factus est Dóminus refúgium \textbf{páu}peri:~* adjútor in opportunitátibus, in tribu\textit{la}\textit{ti}\textbf{ó}ne.
\item Et sperent in te qui novérunt nomen \textbf{tu}um:~* quóniam non dereliquísti quærén\textit{tes} \textit{te}, \textbf{Dó}mine.
\item Glória Patri, et \textbf{Fí}lio,~* et Spirí\textit{tu}\textit{i} \textbf{Sanc}to.
\item Sicut erat in princípio, et nunc, et \textbf{sem}per,~* et in sǽcula sæcu\textit{ló}\textit{rum}. \textbf{A}men.
\end{enumerate}

\switchcolumn

\paragraph{Psaume 9 - I}
\aa Tu siégeras sur un trône, toi qui juges la justice.

\gscore{8_G}

\begin{enumerate}[wide, itemsep=0mm, labelwidth=!, labelindent=0pt, label=\color{gregoriocolor}\theenumi]
\item De tout mon coeur, Seigneur, je rendrai grâce, je dirai tes innombrables merveilles ;
\item pour toi, j'exulterai, je danserai, je fêterai ton nom, Dieu Très-Haut.
\item Mes ennemis ont battu en retraite, devant ta face, ils s'écroulent et périssent.
\item Tu as plaidé mon droit et ma cause, tu as siégé, tu as jugé avec justice.
\item Tu menaces les nations, tu fais périr les méchants, à tout jamais tu effaces leur nom.
\item L'ennemi est achevé, ruiné pour toujours, tu as rasé des villes, leur souvenir a péri.
\item Mais il siège, le Seigneur, à jamais : pour juger, il affermit son trône ;
\item il juge le monde avec justice et gouverne les peuples avec droiture.
\item Qu'il soit la forteresse de l'opprimé, sa forteresse aux heures d'angoisse :
\item ils s'appuieront sur toi, ceux qui connaissent ton nom ; jamais tu n'abandonnes, Seigneur, ceux qui te cherchent.
\end{enumerate}

\newpage
\switchcolumn*

\paragraph{Psalmus 9 - II}

\gscore{a6_1g2}

\begin{enumerate}[wide, itemsep=0mm, labelwidth=!, labelindent=0pt, label=\color{gregoriocolor}\theenumi]
\item Psállite Dómino, qui hábi\textbf{tat} in \textbf{Si}on:~* annuntiáte inter Gentes stú\textit{di}\textit{a} \textbf{e}jus:
\item Quóniam requírens sánguinem eórum \textbf{re}cor\textbf{dá}tus est:~* non est oblítus cla\textit{mó}\textit{rem} \textbf{páu}perum.
\item Miserére \textbf{me}i, \textbf{Dó}mine:~* vide humilitátem meam de ini\textit{mí}\textit{cis} \textbf{me}is.
\item Qui exáltas me de \textbf{por}tis \textbf{mor}tis,~* ut annúntiem omnes laudatiónes tuas in portis fí\textit{li}\textit{æ} \textbf{Si}on.
\item Exsultábo in salu\textbf{tá}ri \textbf{tu}o:~* infíxæ sunt Gentes in intéritu, \textit{quem} \textit{fe}\textbf{cé}runt.
\item In láqueo isto, quem \textbf{abs}con\textbf{dé}runt,~* comprehénsus est \textit{pes} \textit{e}\textbf{ó}rum.
\item Cognoscétur Dóminus ju\textbf{dí}cia \textbf{fá}ciens:~* in opéribus mánuum suárum comprehénsus \textit{est} \textit{pec}\textbf{cá}tor.
\item Convertántur peccatóres \textbf{in} in\textbf{fér}num,~* omnes Gentes quæ oblivis\textit{cún}\textit{tur} \textbf{De}um.
\item Quóniam non in finem oblívio \textbf{e}rit \textbf{páu}peris:~* patiéntia páuperum non perí\textit{bit} \textit{in} \textbf{fi}nem.
\item Glória \textbf{Pa}tri, et \textbf{Fí}lio,~* et Spirí\textit{tu}\textit{i} \textbf{Sanc}to.
\item Sicut erat in princípio, et \textbf{nunc}, et \textbf{sem}per,~* et in sǽcula sæcu\textit{ló}\textit{rum}. \textbf{A}men.
\end{enumerate}

\switchcolumn

\paragraph{Psaume 9 - II}
\aa Lève-toi, Seigneur, que l'homme ne triomphe pas.

\gscore{1_g2}

\begin{enumerate}[wide, itemsep=0mm, labelwidth=!, labelindent=0pt, label=\color{gregoriocolor}\theenumi]
\item Fêtez le Seigneur qui siège dans Sion, annoncez parmi les peuples ses exploits !
\item Attentif au sang versé, il se rappelle, il n'oublie pas le cri des malheureux.
\item Pitié pour moi, Seigneur, vois le mal que m'ont fait mes adversaires, * toi qui m'arraches aux portes de la mort ;
\item et je dirai tes innombrables louanges aux portes de Sion, * je danserai de joie pour ta victoire.
\item Ils sont tombés, les païens, dans la fosse qu'ils creusaient ; aux filets qu'ils ont tendus, leurs pieds se sont pris.
\item Le Seigneur s'est fait connaître : il a rendu le jugement, il prend les méchants à leur piège.
\item Que les méchants retournent chez les morts, toutes les nations qui oublient le vrai Dieu !
\item Mais le pauvre n'est pas oublié pour toujours : jamais ne périt l'espoir des malheureux.
\item Lève-toi, Seigneur : qu'un mortel ne soit pas le plus fort, que les nations soient jugées devant ta face !
\item Frappe-les d'épouvante, Seigneur : que les nations se reconnaissent mortelles !
\end{enumerate}

\switchcolumn*

\versiculusabsolutio
	{Scápulis suis obumbrábit tibi}
	{Et sub pennis ejus sperábis}
	{Ipsíus píetas et misericórdia nos ádjuvet, qui cum Patre et Spíritu Sancto vivit et regnat in sǽcula sæculórum}
	{Sous ses épaules, il t’abritera}
	{Et sous ses ailes, tu auras confiance}
	{Qu'il nous secoure par sa bonté et sa miséricorde, celui qui, avec le Père et le Saint-Esprit, vit et règne dans les siècles des siècles}

\lectioresponsorium
	{Lectio \textsc{iv}}
	{Quatrième lecture}
	{Deus Pater omnípotens sit nobis propítius et clemens}
	{Que Dieu le Père tout-puissant soit pour nous propice et plein de clémence}
	{
		~\\Ex Tractátu sancti Joánnis Chrysóstomi super Psalmum quadragésimum tértium.

		Deus, áuribus nostris audívimus, patres nostri annuntiavérunt nobis opus quod operátus es in diébus eórum. Hunc Psalmum dicit quidem prophéta, dicit autem non ex persóna própria, sed ex persóna Machabæórum, narrans et prædícens quæ futúra erant illo témpore. Tales enim sunt prophétæ: ómnia témpora percúrrunt, præséntia, prætérita, futúra. Quinam sint autem hi Machabǽi, quidque passi sint et quid fécerint, necessárium est primum dícere, ut sint apertióra quæ in arguménto dicúntur. Ii enim, cum invasísset Judǽam Antíochus qui dictus est Epíphanes, et ómnia devastásset, et multos qui tunc erant, a pátriis institútis resilíre coëgísset, permansérunt illǽsi ab illis tentatiónibus.
	}
	{
		~\\Du traité de saint Jean Chrysostome sur le psaume 43.
	
		«~Ô Dieu, nous les avons entendues de nos oreilles, nos pères nous les ont racontées, les oeuvres que Tu as accomplies de leur temps aux jours anciens.~» Le Roi-prophète, dans ce psaume, parle non pas en son nom mais au nom des Macchabées, et il raconte et prédit les événements qui devaient avoir lieu de leur temps. Tels sont en effet les prophètes. Ils parcourent tous les temps présents, passés et à venir. Mais il est nécessaire de faire connaître tout d'abord ce qu'étaient ces Macchabées, aussi bien que leurs travaux et leurs épreuves, pour jeter un plus grand jour sur le sujet de ce psaume. Lorsque Antiochus-Épiphane fut entré dans la Judée, en semant la dévastation sous ses pas, et qu'il eut forcé un grand nombre de Juifs à transgresser les lois et la religion de leurs pères, les Macchabées demeurèrent invulnérables au milieu de ces rudes épreuves.}
	{r4}
	{\vfill
	\rr Ne craignez pas l’assaut des ennemis : rappelez-vous comment ont été sauvés nos pères ;\\
	\GreSpecial{*} Et maintenant lançons notre appel vers le ciel, et notre Dieu aura pitié de nous.\\
	\vv Rappelez-vous ses merveilles, qu’il a accomplies contre Pharaon et son armée dans la mer Rouge.\\
	\GreSpecial{*} Et maintenant lançons notre appel vers le ciel, et notre Dieu aura pitié de nous.
	\vfill}

\lectioresponsorium
	{Lectio \textsc{v}}
	{Cinquième lecture}
	{Christus perpétuæ det nobis gáudia vitæ}
	{Que le Christ nous donne les joies de l'éternelle vie}
	{
		Et quando grave quidem bellum ingruébat, nec quidquam possent fácere quod prodésset, se abscondébant; nam hoc quoque fecérunt Apóstoli. Non enim semper apparéntes in média irruébant perícula, sed nonnúmquam et fugiéntes, et laténtes secedébant. Postquam autem parum respirárunt, tamquam generósi quidam cátuli ex antris exsiliéntes et e látebris emergéntes, statuérunt non se ámplius solos serváre, sed étiam álios quoscúmque possent: et civitátem et omnem regiónem obeúntes, collegérunt quotquot invenérunt adhuc sanos et íntegros; et multos étiam qui laborábant et corrúpti erant, in statum prístinum redegérunt, eis persuadéntes redíre ad legem pátriam.
	}
	{Et quand la guerre devenait si accablante que toute résistance était impossible, les Macchabées se cachaient. C'est ce que firent plus tard les apôtres eux-mêmes. Ils ne se jetaient pas continuellement au milieu des dangers, mais ils s'y dérobaient quelquefois en se retirant dans des lieux sûrs et ignorés. Puis, lorsque les Macchabées avaient tant soit peu repris courage, ils sortaient de leurs retraites; comme de jeunes lions vigoureux, ils s'élançaient de leurs cavernes, résolus à sauver avec eux tous ceux qu'ils pourraient. Ils parcouraient les villes, la contrée tout entière, ils réunissaient autour d'eux tous ceux qui étaient demeurés fidèles, et relevaient le courage de ceux qui s'étaient laissé abattre et corrompre, en les exhortant à revenir à la religion de leurs pères.}
	{r5}
	{\vfill
	\rr Les nations se sont liguées contre nous afin de nous anéantir, et nous ignorons ce qui nous devons faire.\\
	\GreSpecial{*} Seigneur, Dieu, nos regards sont tournés vers toi afin de ne pas périr.\\
	\vv Toi, tu connais leurs intentions à notre égard. Comment pourrons-nous leur résister, si tu ne nous aides pas ?\\
	\GreSpecial{*} Seigneur, Dieu, nos regards sont tournés vers toi afin de ne pas périr.
	\vfill}
	
\lectioresponsorium
	{Lectio \textsc{vi}}
	{Sixième lecture}
	{Ignem sui amóris accéndat Deus in córdibus nostris}
	{Que Dieu daigne allumer dans nos cœurs le feu de son amour}
	{
		Deum enim dicébant esse benígnum et cleméntem, nec umquam adímere salútem, quæ proficíscitur ex pœniténtia. Hæc autem dicéntes, habuérunt deléctum fortissimórum virórum. Non enim pro uxóribus, líberis, et ancíllis, patriǽque eversióne et captivitáte, sed pro lege et pátria república pugnábant. Eórum autem dux erat Deus. Cum ergo áciem dirígerent, et suas ánimas prodígerent, fundébant adversários, non armis fidéntes, sed loco omnis armatúræ, pugnæ causam suffícere ducéntes. Ad bellum autem eúntes non tragœ́dias excitábant, non pæána canébant, sicut nonnúlli fáciunt: non ascivérunt tibícines, ut fit in áliis castris: sed Dei supérne auxílium invocábant, ut adésset, opem ferret et manum præbéret, propter quem bellum gerébant, pro cujus glória decertábant.
	}
	{Ils leur représentaient la grande Bonté de Dieu qui ne refuse jamais le salut au repentir. C'est ainsi qu'ils se formèrent une armée composée d'hommes d'un courage à toute épreuve. Car ce n'était ni pour leurs épouses, ni pour leurs enfants et leurs serviteurs, ni même pour sauver leur patrie de la destruction et de la captivité, mais pour les lois et les institutions religieuses de leurs pères qu'ils combattaient, et Dieu Lui-même était leur chef. Lors donc qu'ils marchaient au combat et qu'ils exposaient leur vie, ils triomphaient de leurs ennemis, par la confiance qu'ils avaient non dans leurs armes, mais dans la cause même pour laquelle ils combattaient et qui était pour eux comme une armure invincible. Aussi, avant de combattre, ils ne poussaient point de cris effrayants; ils ne chantaient pas, comme quelques autres peuples, d'hymnes guerriers; ils ne menaient pas avec eux de joueurs d'instruments, comme dans les autres armées; mais ils invoquaient le secours d'en-haut, et priaient Dieu de prendre leur défense en main, puisque c'était pour Lui qu'ils livraient bataille et pour sa Gloire qu'ils combattaient.}
	{r6}
	{\newpage \rr À toi la puissance, à toi la royauté, Seigneur ; c’est toi qui es au-dessus de toutes les nations :\\
	\GreSpecial{*} Donne la paix, Seigneur, en nos jours.\\
	\vv Dieu, Créateur de toutes choses, terrible et fort, juste et miséricordieux.\\
	\GreSpecial{*} Donne la paix, Seigneur, en nos jours.\\
	\versetGloireAuPere{}\\
	\GreSpecial{*} Donne la paix, Seigneur, en nos jours.}

\subsection{In III Nocturno}

\switchcolumn

\subsection{Troisième Nocturne}

\switchcolumn*

\paragraph{Psalmus 9 - III}

\gscore{a7_2d}

\begin{enumerate}[wide, itemsep=0mm, labelwidth=!, labelindent=0pt, label=\color{gregoriocolor}\theenumi]
\item Ut quid, Dómine, recessísti \textbf{lon}ge,~* déspicis in opportunitátibus, in tribula\textit{ti}\textbf{ó}ne?
\item Dum supérbit ímpius, incénditur \textbf{pau}per:~* comprehendúntur in consíliis qui\textit{bus} \textbf{có}\textbf{gi}tant.
\item Quóniam laudátur peccátor in desidériis ánimæ \textbf{su}æ:~* et iníquus be\textit{ne}\textbf{dí}\textbf{ci}tur.
\item Exacerbávit Dóminum pec\textbf{cá}tor,~* secúndum multitúdinem iræ suæ \textit{non} \textbf{quæ}ret.
\item Non est Deus in conspéctu \textbf{e}jus:~* inquinátæ sunt viæ illíus in om\textit{ni} \textbf{tém}\textbf{po}re.
\item Auferúntur judícia tua a fácie \textbf{e}jus:~* ómnium inimicórum suórum do\textit{mi}\textbf{ná}\textbf{bi}tur.
\item Dixit enim in corde \textbf{su}o:~* Non movébor a generatióne in generatiónem si\textit{ne} \textbf{ma}lo.
\item Cujus maledictióne os plenum est, et amaritúdine, et \textbf{do}lo:~* sub lingua ejus labor \textit{et} \textbf{do}lor.
\item Sedet in insídiis cum divítibus in oc\textbf{cúl}tis:~* ut interfíciat in\textit{no}\textbf{cén}tem.
\item Oculi ejus in páuperem re\textbf{spí}ciunt:~* insidiátur in abscóndito, quasi leo in spelún\textit{ca} \textbf{su}a.
\item Insidiátur ut rápiat \textbf{páu}perem:~* rápere páuperem, dum áttra\textit{hit} \textbf{e}um.
\item In láqueo suo humiliábit \textbf{e}um:~* inclinábit se, et cadet, cum dominátus fúe\textit{rit} \textbf{páu}\textbf{pe}rum.
\item Dixit enim in corde suo: Oblítus est \textbf{De}us,~* avértit fáciem suam ne vídeat \textit{in} \textbf{fi}nem.
\item Glória Patri, et \textbf{Fí}lio,~* et Spirítu\textit{i} \textbf{Sanc}to.
\item Sicut erat in princípio, et nunc, et \textbf{sem}per,~* et in sǽcula sæculó\textit{rum}. \textbf{A}men.
\end{enumerate}

\switchcolumn

\paragraph{Psaume 9 - III}
\aa Pourquoi, Seigneur, te tenir à l'écart ?

\gscore{2}

\begin{enumerate}[wide, itemsep=0mm, labelwidth=!, labelindent=0pt, label=\color{gregoriocolor}\theenumi]
\item Pourquoi, Seigneur, es-tu si loin ? Pourquoi te cacher aux jours d'angoisse ?
\item L'impie, dans son orgueil, poursuit les malheureux : ils se font prendre aux ruses qu'il invente.
\item L'impie se glorifie du désir de son âme, l'arrogant blasphème, il brave le Seigneur ;
\item plein de suffisance, l'impie ne cherche plus : « Dieu n'est rien », voilà toute sa ruse.
\item A tout moment, ce qu'il fait réussit ; + tes sentences le dominent de très haut. * (Tous ses adversaires, il les méprise.)
\item Il s'est dit : « Rien ne peut m'ébranler, je suis pour longtemps à l'abri du malheur. »
\item Sa bouche qui maudit n'est que fraude et violence, sa langue, mensonge et blessure.
\item Il se tient à l'affût près des villages, il se cache pour tuer l'innocent. Des yeux, il épie le faible,
\item il se cache à l'affût, comme un lion dans son fourré ; il se tient à l'affût pour surprendre le pauvre, il attire le pauvre, il le prend dans son filet.
\item Il se baisse, il se tapit ; de tout son poids, il tombe sur le faible.
\item Il dit en lui-même : « Dieu oublie ! il couvre sa face, jamais il ne verra ! »
\end{enumerate}

\switchcolumn*

\paragraph{Psalmus 9 - IV}

\gscore{a8_5a}

\begin{enumerate}[wide, itemsep=0mm, labelwidth=!, labelindent=0pt, label=\color{gregoriocolor}\theenumi]
\item Exsúrge, Dómine Deus, exaltétur manus \textbf{tu}a:~* ne oblivis\textbf{cá}ris \textbf{páu}perum.
\item Propter quid irritávit ímpius \textbf{De}um?~* dixit enim in corde suo: \textbf{Non} re\textbf{quí}ret.
\item Vides quóniam tu labórem et dolórem con\textbf{sí}deras:~* ut tradas eos in \textbf{ma}nus \textbf{tu}as.
\item Tibi derelíctus est \textbf{pau}per:~* órphano tu \textbf{e}ris ad\textbf{jú}tor.
\item Cóntere bráchium peccatóris et ma\textbf{lí}gni:~* quærétur peccátum illíus, et non in\textbf{ve}ni\textbf{é}tur.
\item Dóminus regnábit in ætérnum, et in sǽculum \textbf{sǽ}culi:~* períbitis, Gentes, de \textbf{ter}ra il\textbf{lí}us.
\item Desidérium páuperum exaudívit \textbf{Dó}minus:~* præparatiónem cordis eórum audívit \textbf{au}ris \textbf{tu}a.
\item Judicáre pupíllo et \textbf{hú}mili,~* ut non appónat ultra magnificáre se homo \textbf{su}per \textbf{ter}ram.
\item Glória Patri, et \textbf{Fí}lio,~* et Spi\textbf{rí}tui \textbf{Sanc}to.
\item Sicut erat in princípio, et nunc, et \textbf{sem}per,~* et in sǽcula sæcu\textbf{ló}rum. \textbf{A}men.
\end{enumerate}

\switchcolumn

\paragraph{Psaume 9 - IV}
\aa Lève-toi, Seigneur Dieu, que soit exaltée ta main.

\gscore{5}

\begin{enumerate}[wide, itemsep=0mm, labelwidth=!, labelindent=0pt, label=\color{gregoriocolor}\theenumi]
\item Lève-toi, Seigneur ! Dieu, étends la main ! N'oublie pas le pauvre !
\item Pourquoi l'impie brave-t-il le Seigneur en lui disant : « Viendras-tu me cher\-cher~?~»
\item Mais tu as vu : tu regardes le mal et la souffrance, tu les prends dans ta main ; sur toi repose le faible, c'est toi qui viens en aide à l'orphelin.
\item Brise le bras de l'impie, du méchant ; alors tu chercheras son impiété sans la trouver.
\item A tout jamais, le Seigneur est roi : les païens ont péri sur sa terre.
\item Tu entends, Seigneur, le désir des pauvres, tu rassures leur coeur, tu les écoutes.
\item Que justice soit rendue à l'orphelin, qu'il n'y ait plus d'opprimé, * et que tremble le mortel, né de la terre !
\end{enumerate}

\switchcolumn*

\paragraph{Psalmus 10}

\gscore{a9_1g}

\begin{enumerate}[wide, itemsep=0mm, labelwidth=!, labelindent=0pt, label=\color{gregoriocolor}\theenumi]
\item In Dómino confído:~† quómodo dícitis \textbf{á}nimæ \textbf{me}æ:~* Tránsmigra in montem \textit{sic}\textit{ut} \textbf{pas}ser?
\item Quóniam ecce peccatóres intendérunt arcum,~† paravérunt sagíttas \textbf{su}as in \textbf{phá}retra,~* ut sagíttent in obscúro \textit{rec}\textit{tos} \textbf{cor}de.
\item Quóniam quæ perfecísti, \textbf{de}stru\textbf{xé}runt:~* justus au\textit{tem} \textit{quid} \textbf{fe}cit?
\item Dóminus in templo \textbf{sanc}to \textbf{su}o,~* Dóminus in cælo \textit{se}\textit{des} \textbf{e}jus.
\item Oculi ejus in páupe\textbf{rem} re\textbf{spí}ciunt:~* pálpebræ ejus intérrogant fí\textit{li}\textit{os} \textbf{hó}minum.
\item Dóminus intérrogat \textbf{jus}tum et \textbf{ím}pium:~* qui autem díligit iniquitátem, odit á\textit{ni}\textit{mam} \textbf{su}am.
\item Pluet super pecca\textbf{tó}res \textbf{lá}queos:~* ignis, et sulphur, et spíritus procellárum pars cáli\textit{cis} \textit{e}\textbf{ó}rum.
\item Quóniam justus Dóminus, et justíti\textbf{as} di\textbf{lé}xit:~* æquitátem vidit \textit{vul}\textit{tus} \textbf{e}jus.
\item Glória \textbf{Pa}tri, et \textbf{Fí}lio,~* et Spirí\textit{tu}\textit{i} \textbf{Sanc}to.
\item Sicut erat in princípio, et \textbf{nunc}, et \textbf{sem}per,~* et in sǽcula sæcu\textit{ló}\textit{rum}. \textbf{A}men.
\end{enumerate}

\switchcolumn

\paragraph{Psaume 10}
\aa Le Seigneur est juste, et il aime la justice.

\gscore{1_g}

\begin{enumerate}[wide, itemsep=0mm, labelwidth=!, labelindent=0pt, label=\color{gregoriocolor}\theenumi]
\item Auprès du Seigneur j'ai mon refuge. Comment pouvez-vous me dire : Oiseaux, fuyez à la montagne !
\item Voici que les méchants tendent l'arc : + ils ajustent leur flèche à la corde pour viser dans l'ombre l'homme au coeur droit.
\item Quand sont ruinées les fondations, que peut faire le juste ?
\item Mais le Seigneur, dans son temple saint, + le Seigneur, dans les cieux où il trône, garde les yeux ouverts sur le monde. Il voit, il scrute les hommes ; +
\item le Seigneur a scruté le juste et le méchant : l'ami de la violence, il le hait.
\item Il fera pleuvoir ses fléaux sur les méchants, + feu et soufre et vent de tempête ; c'est la coupe qu'ils auront en partage.
\item Vraiment, le Seigneur est juste ; + il aime toute justice : les hommes droits le verront face à face.
\end{enumerate}

\switchcolumn*

\versiculusabsolutio
	{Scuto circúmdabit te véritas ejus}
	{Non timébis a timóre noctúrno}
	{A vínculis peccatórum nostrórum absólvat nos omnípotens et miséricors Dóminus}
	{D’un bouclier, elle te couvrira, sa vérité}
	{Et tu ne craindras pas la terreur de la nuit}
	{Que le Dieu tout-puissant et miséricordieux daigne nous délivrer des liens de nos péchés}

\lectioresponsorium
	{Lectio \textsc{vii}}
	{Septième lecture}
	{Evangélica léctio sit nobis salus et protéctio}
	{Que la lecture du saint Evangile nous soit salut et protection}
	{
		Léctio sancti Evangélii secúndum Joánnem
		
		In illo témpore: Erat quidam régulus, cujus fílius infirmabátur Caphárnaum. Et réliqua.

		Homilía sancti Gregórii Papæ

		Léctio sancti Evangélii, quam modo, fratres, audístis, expositióne non índiget: sed ne hanc táciti præteriísse videámur, exhortándo pótius quam exponéndo in ea áliquid loquámur. Hoc autem nobis solúmmodo de expositióne vídeo esse requiréndum, cur is, qui ad salútem fílio peténdam vénerat, audívit: Nisi signa et prodígia vidéritis, non créditis. Qui enim salútem fílio quærébat, proculdúbio credébat; neque enim ab eo quǽreret salútem, quem non créderet Salvatórem. Quare ergo dícitur: Nisi signa et prodígia vidéritis, non créditis: qui ante crédidit, quam signa vidéret?
	}
	{
		Évangile de Jésus-Christ selon saint Jean
		
		\rubrique{Joannes 4 : 46-53}
		
		En ce temps-là, il y avait un fonctionnaire royal, dont le fils était malade à Capharnaüm. Et le reste.
		
		Homélie de saint Grégoire, Pape
		
		\rubrique{Homilia 28 in Evangelio}
		
		La lecture du saint Évangile que vous venez d’entendre, frères, n’a pas besoin d’explication ; mais pour ne pas sembler la passer sous silence, disons un mot d’exhortation plutôt que d’explication. Je ne vois rien que nous devions expliquer, sauf ceci : pourquoi celui qui était venu demander le salut pour son fils s’est-il entendu dire : « Si vous ne voyez des signes et des prodiges vous ne croirez pas » ? Il est évident que celui qui cherchait à sauver son fils croyait. Autrement, aurait-il cherché le salut auprès de quelqu’un qu’il ne croyait pas être Sauveur ? Pourquoi, donc, est-il dit : « Si vous ne voyez des signes et des prodiges, vous ne croirez pas », à celui qui a cru avant d’avoir vu des miracles ?
	}
	{r7}
	{\vspace{2cm}
	
	\rr Le soleil envoya son éclat sur les boucliers d'or, et les montagnes resplendirent,\\
	\GreSpecial{*} Et le courage des peuples fut anéanti.\\
	\vv L'armée était en effet fameusement grande et forte, et Judas s'approcha avec son armée pour le combat.\\
	\GreSpecial{*} Et le courage des peuples fut anéanti.}
	
\lectioresponsorium
	{Lectio \textsc{viii}}
	{Huitième lecture}
	{Divínum auxílium máneat semper nobíscum}
	{Que le secours divin demeure toujours avec nous}
	{
		Sed mementóte quid pétiit; et apérte cognoscétis, quia in fide dubitávit. Popóscit namque, ut descénderet et sanáret fílium ejus. Corporálem ergo præséntiam Dómini quærébat, qui per spíritum nusquam déerat. Minus ítaque in illum crédidit, quem non putávit posse salútem dare, nisi præsens esset et córpore. Si enim perfécte credidísset, proculdúbio sciret, quia non esset locus ubi non esset Deus.
	}
	{
		Rappelez-vous ce qu’il a demandé alors vous verrez plus clairement qu’il a douté dans sa foi. Car il lui demanda de «~descendre et de guérir son fils~». Donc il cherchait la présence corporelle du Seigneur qui, par son esprit était présent partout. C’est en cela qu’il n’a pas cru assez en celui qu’il n’a pas estimé capable de rendre le salut s’il n’était pas présent corporellement. S’il avait cru parfaitement, il aurait tenu pour certain qu’il n’y a pas de lieu où Dieu ne soit.
	}
	{r8}
	{\pagebreak \null\vfill \rr Deux Séraphins criaient l'un à l'autre :\\
	\GreSpecial{*} Saint, Saint, Saint, le Seigneur, Dieu Sabaoth.\\
	\GreSpecial{+} Toute la terre est remplie de la majesté de sa gloire.\\
	\vv Ils sont trois qui rendent témoignage dans le ciel : le Père, le Verbe et l’Esprit Saint ; et ces trois sont un.\\
	\GreSpecial{*} Saint, Saint, Saint, le Seigneur, Dieu Sabaoth.\\
	\GreSpecial{+} Toute la terre est remplie de la majesté de sa gloire.\\
	\versetGloireAuPere{}\\
	\GreSpecial{+} Toute la terre est remplie de la majesté de sa gloire.\vfill\newpage}

\newpage	
\lectioresponsorium
	{Lectio \textsc{ix}}
	{Neuvième lecture}
	{Ad societátem cívium supernórum perdúcat nos Rex Angelórum}
	{Que le Roi des Anges nous fasse parvenir à la société des citoyens célestes}
	{
		Ex magna ergo parte diffísus est, qui virtútem non dedit majestáti, sed præséntiæ corporáli. Salútem ítaque fílio pétiit, et tamen in fide dubitávit; quia eum ad quem vénerat, et poténtem ad curándum crédidit, et tamen moriénti fílio esse abséntem putávit. Sed Dóminus, qui rogátur ut vadat, quia non desit ubi invitátur, índicat: solo jussu salútem réddidit, qui voluntáte ómnia creávit.
	}
	{
		Il a donc grandement manqué de confiance parce qu’il n’a pas rendu honneur à la majesté, mais à la présence corporelle. Il demanda donc le salut de son fils, et cependant il douta dans sa foi. Il crut celui à qui il était venu puissant pour guérir, pourtant il l’estima éloigné de son fils mourant. Mais le Seigneur qui est prié de venir montre qu’il n’est pas absent du lieu où il est invité : par son seul commandement il rendit le salut, lui qui par sa volonté a tout créé.
	}
	{}
	{}


\paragraph{Te Deum}
	
\gscore{Te_Deum_simplex}
\newpage

\switchcolumn

\paragraph{Te Deum}
\vfill
À toi, Dieu, notre louange ! Nous t’acclamons, tu es Seigneur !\\
À toi, Père éternel, l’hymne de l’univers.\\
Devant toi se prosternent les archanges, \\
Les anges et les esprits des cieux ;\\
Ils te rendent grâce ; ils adorent et ils chantent :\\
Saint, Saint, Saint, le Seigneur, Dieu de l’univers ;\\
Le ciel et la terre sont remplis de la majesté de ta gloire.\\
C’est toi que les Apôtres glorifient,\\
Toi que proclament les prophètes,\\
Toi dont témoignent les martyrs ;\\
\vfill
\newpage
\null\vfill
C’est toi que par le monde entier, l’Église annonce et reconnaît.\\
Dieu, nous t’adorons : Père infiniment saint,\\
Fils éternel et bien-aimé, Esprit de puissance et de paix.\\
Christ, le Fils du Dieu vivant, le Seigneur de la gloire,\\
Tu n’as pas craint de prendre chair dans le corps d’une vierge\\
Pour libérer l’humanité captive.\\
Par ta victoire sur la mort, \\
Tu as ouvert à tout croyant les portes du Royaume ;\\
Tu règnes dans la gloire à la droite du Père ;\\
Nous croyons que tu viendras pour le jugement.\\
Nous t'en supplions donc, porte secours à tes fidèles,\\
Que tu as rachetés par ton sang :\\
Prends-les pour l'éternité avec tous les saints dans ta lumière.\\
\vfill
\newpage
\null\vfill
Sauve ton peuple, Seigneur et bénis ton héritage ;\\
Sois leur guide et conduis-les sur le chemin d’éternité.\\
Chaque jour nous te bénissons,\\
Nous te louons à jamais dans les siècles des siècles.\\
Daigne, Seigneur, en ce jour, nous garder de tout péché.\\
Aie pitié de nous, Seigneur : aie pitié de nous.\\
Que ta miséricorde soit sur nous puisque tu es notre espoir.\\
Tu es, Seigneur, mon espérance : jamais, je ne serai déçu.
\vfill

\switchcolumn*

\vv Dóminus vobíscum. \\
\rr Et cum spíritu tuo.

Orémus.\\
Largíre, quǽsumus, Dómine, fidélibus tuis indulgéntiam placátus et pacem: ut páriter ab ómnibus mundéntur offénsis, et secúra tibi mente desérviant.
Per Dóminum nostrum Iesum Christum, Fílium tuum: qui tecum vivit et regnat in unitáte Spíritus Sancti, Deus, per ómnia sǽcula sæculórum. \\
\rr Amen.

\switchcolumn

\vv Le Seigneur soit avec vous. \\
\rr Et avec votre esprit.

Prions. \\
Laisse-toi fléchir, Seigneur, et accorde à tes fidèles le pardon et la paix, afin qu’ils soient purifiés de toutes leurs fautes, et qu’ils te servent avec un cœur rempli de confiance.
Par Notre Seigneur Jésus Christ, ton Fils, qui vit et règne avec toi et le Saint-Esprit, Dieu, maintenant et pour les siècles des siècles.\\
\rr Amen.

\switchcolumn*

\vv Dóminus vobíscum. \\
\rr Et cum spíritu tuo.

~~

\smallscore{or--benedicamus_domino}

\switchcolumn

\vv Le Seigneur soit avec vous. \\
\rr Et avec votre esprit.

~~

\vv Bénissons le Seigneur. \\
\rr Nous rendons grâces à Dieu.

\switchcolumn*

~~

\vv Fidélium ánimæ per misericórdiam Dei requiéscant in pace. \\
\rr Amen.

\switchcolumn

~~

\vv Que par la miséricorde de Dieu, les âmes des fidèles trépassés reposent en paix. \\
\rr Amen.

\end{paracol}
\end{document}