\documentclass[twoside]{article}


\usepackage[paperwidth=150mm, paperheight=230mm]{geometry}
\usepackage{fontspec}
%\usepackage[latin1]{inputenc}
\usepackage[french]{babel}
\usepackage[strict]{changepage}
\usepackage{fancyhdr}
\usepackage{paracol}
\usepackage{tableof}
\usepackage{setspace}
\usepackage{alltt}
\usepackage{titlesec}
\usepackage{xcolor}
\usepackage{xstring}
\usepackage{parskip}
\usepackage{enumitem}
\usepackage{etoolbox}
\usepackage{needspace}

%%%%%%%%%%%%%%%%%%%%%%%%%%%%%%%%%%%%%%%%%%%%%%%%%%% Mise en page %%%%%%%%%%%%%%%%%%%%%%%%%%%%%%
% on numérote les nbp par page et non globalement
\usepackage[perpage]{footmisc}

% définition des en-têtes et pieds de page
\pagestyle{empty}
\fancyhead{}
\fancyfoot{}
\renewcommand{\headrulewidth}{0pt}
\setlength{\headheight}{0pt}

% la commande titres permet de changer les titres de gauche et de droite.
\newcommand{\titres}[2]{
	\renewcommand{\rightmark}{\textcolor{red}{\sc #2}}
	\renewcommand{\leftmark}{\textcolor{red}{\sc #1}}
}
\titres{}{}

% pas d'indentation
\setlength{\parindent}{0mm}

\geometry{
inner=20mm,
outer=20mm,
top=10mm,
bottom=25mm,
headsep=0mm,
}

\twosided[p]

%%%%%%%%%%%%%%%%%%%%%%%%%%%%%%%%%%%%%%%%%%%%%%%%% Options gregorio %%%%%%%%%%%%%%%%%%%%%%%%%

\usepackage[autocompile]{gregoriotex}
%\usepackage{gregoriotex}

\definecolor{gregoriocolor}{RGB}{215,65,29}

%% style général de gregorio :
% lignes rouges, commenter pour du noir
%\gresetlinecolor{gregoriocolor}

% texte <alt> (au-dessus de la portée) en rouge et en petit, avec réglage de sa position verticale
\grechangestyle{abovelinestext}{\color{gregoriocolor}\footnotesize}
\newcommand{\altraise}{-1mm}
\grechangedim{abovelinestextraise}{\altraise}{scalable}

% taille des initiales
\newcommand{\initialsize}[1]{
    \grechangestyle{initial}{\fontspec{ZallmanCaps}\fontsize{#1}{#1}\selectfont}
}
\newcommand{\defaultinitialsize}{32}
\initialsize{\defaultinitialsize}
% espace avant et après les initiales
\newcommand{\initialspace}[1]{
  \grechangedim{afterinitialshift}{#1}{scalable}
  \grechangedim{beforeinitialshift}{#1}{scalable}
}
\newcommand{\defaultinitialspace}{0cm}
\initialspace{\defaultinitialspace}


% on définit le système qui capture des headers pour générer des annotations
% cette commande sera appelée pour définir des abréviations ou autres substitutions
\newcommand{\resultat}{}
\newcommand{\abbrev}[3]{
  \IfSubStr{#1}{#2}{ \renewcommand{\resultat}{#3} }{}
}
\newcommand{\officepartannotation}[1]{
  \renewcommand{\resultat}{#1}
  \abbrev{#1}{ntro}{Intr.}
  \abbrev{#1}{re}{Resp.}
  \abbrev{#1}{espo}{Resp.}
  \abbrev{#1}{adu}{Gr.}
  \abbrev{#1}{ll}{All.}
  \abbrev{#1}{act}{Tract.}
  \abbrev{#1}{equen}{Seq.}
  \abbrev{#1}{ffert}{Off.}
  \abbrev{#1}{ommun}{Co.}
  \abbrev{#1}{an}{Ant.}
  \abbrev{#1}{ntiph}{Ant.}
  \abbrev{#1}{ntic}{Cant.}
  \abbrev{#1}{ymn}{Hy.}
  \abbrev{#1}{salm}{}
  \abbrev{#1}{Toni Communes}{}
  \abbrev{#1}{yrial}{}
  \greannotation{\resultat}
}
\newcommand{\modeannotation}[1]{
  \renewcommand{\resultat}{#1}
  \abbrev{#1}{1}{ {\sc i} }
  \abbrev{#1}{2}{ {\sc ii} }
  \abbrev{#1}{3}{ {\sc iii} }
  \abbrev{#1}{4}{ {\sc iv} }
  \abbrev{#1}{5}{ {\sc v} }
  \abbrev{#1}{6}{ {\sc vi} }
  \abbrev{#1}{7}{ {\sc vii} }
  \abbrev{#1}{8}{ {\sc viii} }
  \greannotation{\resultat}
}
\gresetheadercapture{office-part}{officepartannotation}{}
\gresetheadercapture{mode}{modeannotation}{string}

%%%%%%%%%%%%%%%%%%%%%%%%%%%%%%%%%%%%%%%%%%%%%% Graphisme %%%%%%%%%%%%%%%%%%%%%%%%%%%
% on définit l'échelle générale

\newcommand{\echelle}{1}

% on centre les titres et on ne les numérote pas
\titleformat{\section}[block]{\Large\filcenter\sc}{}{}{}
\titleformat{\subsection}[block]{\large\filcenter\sc}{}{}{}
\titleformat{\paragraph}[block]{\filcenter\sc}{}{}{}
\setcounter{secnumdepth}{0}
% on diminue l'espace avant les titres
\titlespacing*{\paragraph}{0pt}{1.8ex plus .4ex minus .4ex}{1.2ex plus .2ex minus .2ex}

% commandes versets, repons et croix
\newcommand{\vv}{\textcolor{gregoriocolor}{\fontspec[Scale=\echelle]{Charis SIL}℣.\hspace{3mm}}}
\newcommand{\rr}{\textcolor{gregoriocolor}{\fontspec[Scale=\echelle]{Charis SIL}℟.\hspace{3mm}}}
\newcommand{\cc}{\textcolor{gregoriocolor}{\fontspec[Scale=\echelle]{FreeSerif}\symbol{"2720}~}}
\renewcommand{\aa}{\textcolor{gregoriocolor}{\fontspec[Scale=\echelle]{Charis SIL}\Abar.\hspace{3mm}}}

% commandes diverses
\newcommand{\antiphona}{\textcolor{gregoriocolor}{\noindent Antiphona.\hspace{4mm}}}
\newcommand{\antienne}{\textcolor{gregoriocolor}{\noindent Antienne.\hspace{4mm}}}
% rubrique
\newcommand{\rubrique}[1]{\textcolor{gregoriocolor}{\emph{#1}}}
% pour afficher du texte noir roman au milieu d'une rubrique
\newcommand{\normaltext}[1]{{\normalfont\normalcolor #1}}

\newcommand{\saut}{\hspace{1cm}}
\newcommand{\capsaut}{\hspace{3mm}}
% pour affichier 1 en rouge et un peu d'espace
\newcommand{\un}{{\color{gregoriocolor} 1~~~}}


% abréviations
\newcommand{\tpalleluia}{\rubrique{(T.P.} \mbox{Allelúia.\rubrique{)}}}
\newcommand{\tpalleluiafr}{\rubrique{(T.P.} \mbox{Alléluia.\rubrique{)}}}

\newcommand{\tqomittitur}{{\small \rubrique{(In Tempore Quadragesimæ ommittitur} Allelúia.\rubrique{)}}}
\newcommand{\careme}{{\small \rubrique{(Pendant le Carême on omet l'}Alléluia.\rubrique{)}}}

% environnement hymne : alltt + normalfont + marges custom
\newenvironment{hymne}
  {
  \begin{adjustwidth}{1.6cm}{1mm}
  \begin{alltt}\normalfont
  }
  {
  \end{alltt}
  \end{adjustwidth}
  }
  
% la commande \u permet de souligner les inflexions
\let\u\textbf

% on définit la police par défaut
\setmainfont[Ligatures=TeX, Scale=\echelle]{Charis SIL}
%renderer=ICU a l'air de ne plus marcher...
%\setmainfont[Renderer=ICU, Ligatures=TeX, Scale=\echelle]{Charis SIL}
\setstretch{0.9}


%%%%%%%%%%%%%%%% Commandes de mise en forme %%%%%%%%%%%%%%%%

\newcommand{\lectioresponsorium}[8]{
	\needspace{3\baselineskip}
	\paragraph{#1}

	\vv Jube, domne, benedícere.\\
	\vv #3.\\
	\rr Amen.

	#5

	\vv Tu autem, Dómine, miserére nobis.\\
	\rr Deo grátias.

	\ifblank{#7}{}{\gregorioscore{gabc/#7}}

	\switchcolumn
	%\needspace{3\baselineskip}
	\paragraph{#2}

	\vv Veuillez, maître, bénir.\\
	\vv #4.\\
	\rr Amen.

	#6

	\vv Et toi Seigneur, prends pitié de nous.\\
	\rr Nous rendons grâces à Dieu.

	\ifblank{#8}{}{#8}

	\switchcolumn*

}

\newcommand{\versiculus}[2]{
	\vv #1.\\
	\rr #2.\\
}

\newcommand{\versiculusabsolutio}[6]{

	\paragraph{Versiculus et Absolutio}

	\versiculus{#1}{#2}
	\vv Pater noster... \rubrique{(secrètement)} Et ne nos indúcas in tentatiónem. \\
	\rr Sed líbera nos a malo. \\
	\vv #3. \rr Amen.

	\switchcolumn

	\paragraph{Versicule et Absolution}

	\versiculus{#4}{#5}
	\vv Notre Père... Et ne nous laisse pas entrer en tentation. \\
	\rr Mais délivre-nous du mal. \\
	\vv #6. \rr Amen.

	\switchcolumn*

}

\newcommand{\versetGloireAuPere}{
	\vv Gloire au Père, au Fils, et au Saint-Esprit.
}

\newcommand{\gscore}[1]{
	\gregorioscore{gabc/#1}
}

\newcommand{\smallscore}[1]{
	\gresetinitiallines{0}
	\gscore{#1}
	\gresetinitiallines{1}
}

\begin{document}
\selectlanguage{french}

\sloppy

\begin{center}\begin{doublespace}
{\fontspec[Scale=\echelle]{Futura Book}
\MakeUppercase{\Large Messe\\20\ieme~dimanche après la Pentecôte}\\
selon l'usage ancien du rite romain
}
\end{doublespace}\end{center}

\paragraph{Aspersion}

\gscore{01an--asperges_me--solesmes}

\emph{\rr Tu m'aspergeras, Seigneur, avec l'hysope, et je serai purifié; tu me laveras, et je serai plus blanc que la neige.\\
\vv Aie pitié de moi, mon Dieu, selon ta grande miséricorde. Gloire au Père...}

\begin{paracol}{2}
\vv Osténde nobis, Dómine, misericórdiam tuam.\\
\rr Et salutáre tuum da nobis.\\
\vv Dómine exáudi oratiónem meam.\\
\rr Et clamor meus ad te véniat.\\
\vv Dóminus vobíscum.\\
\rr Et cum spíritu tuo.\\
\vv Orémus.\\
Exáudi nos, Dómine sancte, Pater omnípotens, ætérne Deus :
et míttere dignéris sanctum Angelum tuum de cælis; qui custódiat,
fóveat, prótegat, vísitet, atque deféndat omnes habitántes in hoc habitáculo.
Per Christum Dóminum nostrum.\\
\rr Amen.

\switchcolumn

\vv Montre-nous, Seigneur, ta miséricorde.\\
\rr Et donne-nous ton salut.\\
\vv Seigneur, écoute ma prière.\\
\rr Et que mon cri parvienne jusqu’à toi.\\
\vv Le Seigneur soit avec vous.\\
\rr Et avec votre esprit.\\
\vv Prions. \\
Exauce-nous, Seigneur saint, Père tout-puissant Dieu éternel,
et daigne envoyer du ciel ton saint ange pour qu’il garde et soutienne,
qu’il protège, visite et défende tous ceux qui sont réunis en ce lieu.
Par le Christ, notre Seigneur.\\
\rr Ainsi soit-il.

\end{paracol}


\paragraph{Prières au bas de l'autel}

\begin{paracol}{2}

\vv In nómine Patris, \cc et Fílii, et Spíritus Sancti. \\
\rr Amen.

\switchcolumn

\rr Au nom du Père et du Fils, et du Saint-Esprit. \\
\rr Ainsi soit-il.

\switchcolumn*

\vv Introíbo ad altáre Dei.\\
\rr Ad Deum, qui lætíficat iuventútem meam.\\
\vv Iúdica me, Deus, et discérne causam meam de gente non sancta: ab hómine iníquo et dolóso érue me.\\
\rr Quia tu es, Deus, fortitúdo mea: quare me repulísti, et quare tristis incédo, dum afflígit me inimícus?\\
\vv Emítte lucem tuam et veritátem tuam: ipsa me deduxérunt, et adduxérunt in montem sanctum tuum et in tabernácula tua.\\
\rr Et introíbo ad altáre Dei: ad Deum, qui lætíficat iuventútem meam.\\
\vv Confitébor tibi in cíthara, Deus, Deus meus: quare tristis es, ánima mea, et quare contúrbas me?\\
\rr Spera in Deo, quóniam adhuc confitébor illi: salutáre vultus mei, et Deus meus.\\
\vv Glória Patri, et Fílio, et Spirítui Sancto.\\
\rr Sicut erat in princípio, et nunc, et semper: et in sǽcula sæculórum. Amen.\\
\vv Introíbo ad altáre Dei.\\
\rr Ad Deum, qui lætíficat iuventútem meam.

\switchcolumn

\vv J'entrerai à l'autel de Dieu.\\
\rr Au Dieu qui réjouit ma jeunesse.\\
\vv Juge-moi, ô Dieu, et séparez ma cause de celle d'une nation qui n'est pas sainte; délivre-moi de l'homme méchant et trompeur.\\
\rr Car tu es ma force, ô Dieu; pourquoi m'as-tu repoussé, et pourquoi dois-je marcher attristé, pendant que l'ennemi m'afflige?\\
\vv Envoie ta lumière et ta vérité: elles me conduiront et m'amèneront à ta montagne sainte et à tes tabernacles.\\
\rr Et j'entrerai à l'autel de Dieu, au Dieu qui réjouit ma jeunesse.\\
\vv Je te louerai sur la harpe, ô Dieu, mon Dieu. Pourquoi es-tu triste, mon âme? Et pourquoi me troubles-tu?\\
\rr Espère en Dieu, car je Le louerai encore, lui, le salut de mon visage et mon Dieu.\\
\vv Gloire au Père et au Fils, et au Saint-Esprit.\\
\rr Comme il était au commencement, maintenant et toujours, et dans les siècles des siècles. Ainsi soit-il.\\
\vv J'entrerai à l'autel de Dieu.\\
\rr Au Dieu qui réjouit ma jeunesse.

\switchcolumn*

\vv Adiutórium nostrum \cc in nómine Dómini.\\
\rr Qui fecit cælum et terram.

\switchcolumn

\vv Notre secours est dans le Nom du Seigneur.\\
\rr Qui a fait le ciel et la terre.

\switchcolumn*

\vv Confíteor Deo omnipoténti, beátæ Maríæ semper Vírgini, beáto Michaéli Archángelo, beáto Ioánni Baptístæ, sanctis Apóstolis Petro et Paulo, ómnibus Sanctis, et vobis, fratres: quia peccávi nimis cogitatióne, verbo et ópere: mea culpa, mea culpa, mea máxima culpa. Ideo precor beátam Maríam semper Vírginem, beátum Michaélem Archángelum, beátum Ioánnem Baptístam, sanctos Apóstolos Petrum et Paulum, omnes Sanctos, et vos, fratres, oráre pro me ad Dóminum, Deum nostrum.\\
\rr Misereátur tui omnípotens Deus, et, dimíssis peccátis tuis, perdúcat te ad vitam ætérnam.\\
\vv Amen.\\
\rr Confíteor Deo omnipoténti, beátæ Maríæ semper Vírgini, beáto Michaéli Archángelo, beáto Ioánni Baptístæ, sanctis Apóstolis Petro et Paulo, ómnibus Sanctis, et tibi, pater: quia peccávi nimis cogitatióne, verbo et ópere: mea culpa, mea culpa, mea máxima culpa. Ideo precor beátam Maríam semper Vírginem, beátum Michaélem Archángelum, beátum Ioánnem Baptístam, sanctos Apóstolos Petrum et Paulum, omnes Sanctos, et te, pater, oráre pro me ad Dóminum, Deum nostrum.\\
\vv Misereátur vestri omnípotens Deus, et, dimíssis peccátis vestris, perdúcat vos ad vitam ætérnam.\\
\rr Amen.\\
\vv Indulgéntiam, \cc absolutiónem et remissiónem peccatórum nostrórum tríbuat nobis omnípotens et miséricors Dóminus.\\
\rr Amen.

\switchcolumn

\vv Je confesse à Dieu tout-puissant, à la bienheureuse Marie toujours vierge, à saint Michel Archange, à saint Jean-Baptiste, aux saints apôtres Pierre et Paul, à tous les saints, et à vous mes frères, que j’ai beaucoup péché, en pensée, en parole et par action. C’est ma faute, c’est ma faute, c’est ma très grande faute. C’est pourquoi je supplie la bienheureuse Marie toujours vierge, saint Michel Archange, saint Jean-Baptiste, les saints apôtres Pierre et Paul, tous les saints et vous, mes frères, de prier pour moi le Seigneur notre Dieu. \\
\rr Que le Dieu tout-puissant vous fasse miséricorde, qu’il vous pardonne vos péchés et vous conduise à la vie éternelle. \\
\vv Ainsi soit-il. \\
\rr Je confesse à Dieu tout-puissant, à la bienheureuse Marie toujours vierge, à saint Michel Archange, à saint Jean-Baptiste, aux saints apôtres Pierre et Paul, à tous les saints, et à vous mes frères, que j’ai beaucoup péché, en pensée, en parole et par action.
C’est ma faute, c’est ma faute, c’est ma très grande faute. C’est pourquoi je supplie la bienheureuse Marie toujours vierge, saint Michel Archange, saint Jean-Baptiste, les saints apôtres Pierre et Paul, tous les saints et vous, mes frères, de prier pour moi le Seigneur notre Dieu. \\
\vv Que le Dieu tout-puissant vous fasse miséricorde, qu’il vous pardonne vos péchés et vous conduise à la vie éternelle. \\
\rr Ainsi soit-il.

\end{paracol}
\vv Que le Seigneur tout-puissant et miséricordieux nous accorde le pardon, l’absolution et la rémission de nos péchés.\\
\rr Ainsi soit-il.

\begin{paracol}{2}

\vv Deus, tu convérsus vivificábis nos.\\
\rr Et plebs tua lætábitur in te.\\
\vv Osténde nobis, Dómine, misericórdiam tuam.\\
\rr Et salutáre tuum da nobis.\\
\vv Dómine, exáudi oratiónem meam.\\
\rr Et clamor meus ad te véniat.\\
\vv Dóminus vobíscum.\\
\rr Et cum spíritu tuo.\\
\vv Orémus. \\
Aufer a nobis, quǽsumus, Dómine, iniquitátes nostras: ut ad Sancta sanctórum puris mereámur méntibus introíre. Per Christum, Dóminum nostrum. Amen.\\
Orámus te, Dómine, per mérita Sanctórum tuórum, quorum relíquiæ hic sunt, et ómnium Sanctórum: ut indulgére dignéris ómnia peccáta mea. Amen.

\switchcolumn

\vv Dieu, tourne-toi vers nous et donne-nous la vie.\\
\rr Et ton peuple se réjouira en toi.\\
\vv Montre-nous, Seigneur, ta miséricorde.\\
\rr Et accorde-nous votre salut.\\
\vv Seigneur, exauce ma prière.\\
\rr Et que mon cri parvienne jusqu’à toi.\\
\vv Le Seigneur soit avec vous.\\
\rr Et avec votre esprit.\\
\vv Prions. \\
Enlève nos fautes, Seigneur, nous t'en prions, afin que nous puissions pénétrer jusqu’au Saint des Saints avec une âme pure. Par le Christ notre Seigneur. Ainsi soit-il.\\
Nous te prions, Seigneur, par les mérites de tes saints dont nous conservons ici les reliques, et de tous les saints, de daigner me pardonner tous mes péchés. Ainsi soit-il.

\end{paracol}

\gscore{omnia_quae_fecisti}

\rubrique{Dan. 3: 31; 31: 29; 31: 35 ; Ps. 118.}\\
\emph{\rr Tout ce que tu nous as fait, Seigneur, c'est en toute justice que tu l'as fait, car nous avons péché contre toi, et nous n'avons pas obéi à tes commandements ; mais donne gloire à ton Nom, et agis envers nous selon la grandeur de ta miséricorde.\\
\vv \rubrique{\emph{1. }} Heureux les hommes intègres dans leurs voies qui marchent suivant la loi du Seigneur !\\
\vv \rubrique{\emph{2. }} Heureux ceux qui gardent ses exigences, ils le cherchent de tout coeur !\\
\vv \rubrique{\emph{3. }} Épargne-moi l'insulte et le mépris : je garde tes exigences.}

\gscore{02ky--kyrie_xi--solesmes.1}

\gscore{03ky--gloria_xi--solesmes}

\begin{alltt}\normalfont
\emph{Gloire à Dieu, au plus haut des cieux, 
et paix sur la terre aux hommes qu’il aime.
Nous te louons, nous te bénissons, nous t’adorons,
Nous te glorifions, nous te rendons grâce, pour ton immense gloire,
Seigneur Dieu, Roi du ciel, Dieu le Père tout-puissant.
Seigneur, Fils unique, Jésus Christ,
Seigneur Dieu, Agneau de Dieu, le Fils du Père.
Toi qui enlèves le péché du monde, prends pitié de nous
Toi qui enlèves le péché du monde, reçois notre prière ;
Toi qui es assis à la droite du Père, prends pitié de nous.
Car toi seul es saint, toi seul es Seigneur,
Toi seul es le Très-Haut, Jésus Christ, avec le Saint-Esprit
Dans la gloire de Dieu le Père. Amen.}
\end{alltt}

\paragraph{Collecte}

\begin{paracol}{2}

\vv Dóminus vobíscum.\\
\rr Et cum spíritu tuo.\\
\vv Orémus.\\
Largíre, quǽsumus, Dómine, fidélibus tuis indulgéntiam placátus et pacem: ut páriter ab ómnibus mundéntur offénsis, et secúra tibi mente desérviant.
Per Dóminum nostrum Iesum Christum, Fílium tuum: qui tecum vivit et regnat in unitáte Spíritus Sancti Deus, per ómnia sǽcula sæculórum.\\
\rr Amen.

\switchcolumn

\vv Le Seigneur soit avec vous. \\
\rr Et avec votre esprit.\\
\vv Prions.\\
Prodigue, Seigneur, à tes fidèles, en signe de faveur, le pardon et la paix, afin qu'ils soient purifiés de tout péché et te servent en toute tranquillité d'esprit.
Par notre Seigneur Jésus-Christ, ton Fils, qui, étant Dieu, vit et règne avec toi, dans l’unité du Saint-Esprit, pour les siècles des siècles.\\
\rr Ainsi soit-il.

\end{paracol}

\newpage

\paragraph{Épître}

\begin{paracol}{2}
Lectio Epistolæ beáti Pauli Apóstoli ad Ephésios

\rubrique{Eph 5 : 15-21}\\
Fratres: Vidéte, quómodo caute ambulétis: non quasi insipiéntes, sed ut sapiéntes, rediméntes tempus, quóniam dies mali sunt. Proptérea nolíte fíeri imprudéntes, sed intellegéntes, quæ sit volúntas Dei. Et nolíte inebriári vino, in quo est luxúria: sed implémini Spíritu Sancto, loquéntes vobismetípsis in psalmis et hymnis et cánticis spirituálibus, cantántes et psalléntes in córdibus vestris Dómino: grátias agéntes semper pro ómnibus, in nómine Dómini nostri Iesu Christi, Deo et Patri. Subiecti ínvicem in timóre Christi.

\rubrique{À la messe chantée, on ne répond rien. À la messe basse, on répond :}\\
\rr Deo grátias.

\switchcolumn
Épître de saint Paul Apôtre aux Éphésiens.

Frères, prenez bien garde à votre conduite : ne vivez pas comme des fous, mais comme des sages.
Tirez parti du temps présent, car nous traversons des jours mauvais.
Ne soyez donc pas insensés, mais comprenez bien quelle est la volonté du Seigneur.
Ne vous enivrez pas de vin, car il porte à l’inconduite ; soyez plutôt remplis de l’Esprit Saint.
Dites entre vous des psaumes, des hymnes et des chants inspirés, chantez le Seigneur et célébrez-le de tout votre cœur.
À tout moment et pour toutes choses, au nom de notre Seigneur Jésus Christ, rendez grâce à Dieu le Père.
Par respect pour le Christ, soyez soumis les uns aux autres.

\rr Nous rendons grâces à Dieu.

\end{paracol}

\gscore{oculi_omnium}

\rubrique{Ps. 144 : 15-16}\\
\emph{\rr Les yeux sur toi, tous, ils espèrent: tu leur donnes la nourriture au temps voulu;\\
\vv Tu ouvres ta main: tu rassasies avec bonté tout ce qui vit.}

\gscore{alleluia_paratum}

\rubrique{Ps. 107 : 2}\\
\emph{Alléluia, alléluia. Mon cœur est prêt, ô Dieu, mon cœur est prêt; je chanterai et psalmodierai pour toi qui es ma gloire. Alléluia.}

\paragraph{Évangile}

\begin{paracol}{2}

\rubrique{Le prêtre dit à voix basse :}\\
Munda cor meum ac lábia mea, omnípotens Deus, qui lábia Isaíæ Prophétæ cálculo mundásti igníto: ita me tua grata miseratióne dignáre mundáre, ut sanctum Evangélium tuum digne váleam nuntiáre. Per Christum, Dóminum nostrum. Amen.\\
Iube, Dómine, benedícere. Dóminus sit in corde meo et in lábiis meis: ut digne et competénter annúntiem Evangélium suum. Amen.

\switchcolumn

~\\
Purifiez mon cœur et mes lèvres, Dieu tout-puissant, qui avez purifié les lèvres du prophète Isaïe avec un charbon ardent ; daignez aussi me purifier par votre miséricordieuse bonté, afin que je puisse proclamer dignement votre saint Évangile. Par le Christ notre Seigneur. Ainsi soit-il. \\
Seigneur, veuillez me bénir. Que le Seigneur soit dans mon cœur et sur mes lèvres, afin que j’annonce dignement et convenablement son Évangile. Ainsi soit-il.

\switchcolumn*

\vv Dóminus vobíscum.\\
\rr Et cum spíritu tuo.\\
\vv Sequéntia \cc sancti Evangélii secúndum Ioánnem.\\
\rr Glória tibi, Dómine.

\switchcolumn

\vv Le Seigneur soit avec vous.\\
\rr Et avec votre esprit.\\
\vv Suite du saint Évangile selon saint Jean.\\
\rr Gloire à toi, Seigneur.

\switchcolumn*

\rubrique{Jn 4 : 46-53}\\
In illo témpore: Erat quidam régulus, cuius fílius infirmabátur Caphárnaum. Hic cum audísset, quia Iesus adveníret a Iudǽa in Galilǽam, ábiit ad eum, et rogábat eum, ut descénderet et sanáret fílium eius: incipiébat enim mori. Dixit ergo Iesus ad eum: Nisi signa et prodígia vidéritis, non créditis. Dicit ad eum régulus: Dómine, descénde, priúsquam moriátur fílius meus. Dicit ei Iesus: Vade, fílius tuus vivit. Crédidit homo sermóni, quem dixit ei Iesus, et ibat. Iam autem eo descendénte, servi occurrérunt ei et nuntiavérunt, dicéntes, quia fílius eius víveret. Interrogábat ergo horam ab eis, in qua mélius habúerit. Et dixérunt ei: Quia heri hora séptima relíquit eum febris. Cognóvit ergo pater, quia illa hora erat, in qua dixit ei Iesus: Fílius tuus vivit: et crédidit ipse et domus eius tota.

\rubrique{À la messe chantée, on ne répond rien. À la messe basse, on répond :}\\
\rr Laus tibi, Christe.

\rubrique{Le prêtre dit à voix basse :} Per Evangélica dicta, deleántur nostra delícta.

\switchcolumn

En ce temps-là, il y avait un fonctionnaire royal, dont le fils était malade à Capharnaüm.
Ayant appris que Jésus arrivait de Judée en Galilée, il alla le trouver ; il lui demandait de descendre à Capharnaüm pour guérir son fils qui était mourant.
Jésus lui dit : « Si vous ne voyez pas de signes et de prodiges, vous ne croirez donc pas ! »
Le fonctionnaire royal lui dit : « Seigneur, descends, avant que mon enfant ne meure ! »
Jésus lui répond : « Va, ton fils est vivant. » L’homme crut à la parole que Jésus lui avait dite et il partit.
Pendant qu’il descendait, ses serviteurs arrivèrent à sa rencontre et lui dirent que son enfant était vivant.
Il voulut savoir à quelle heure il s’était trouvé mieux. Ils lui dirent : « C’est hier, à la septième heure, (au début de l’après-midi), que la fièvre l’a quitté. »
Le père se rendit compte que c’était justement l’heure où Jésus lui avait dit : « Ton fils est vivant. » Alors il crut, lui, ainsi que tous les gens de sa maison.

\rr Louange à toi, ô Christ.

Que par les paroles de l’Évangile, nos péchés soient effacés.

\end{paracol}

\gscore{04ky--credo_v--solesmes}
\begin{alltt}\normalfont
\emph{Je crois en un seul Dieu, le Père tout puissant,
	créateur du ciel et de la terre, de l’univers visible et invisible,
Je crois en un seul Seigneur, Jésus Christ,
	le Fils unique de Dieu, né du Père avant tous les siècles :
Il est Dieu, né de Dieu, lumière, née de la lumière, vrai Dieu, né du vrai Dieu.
Engendré non pas créé, consubstantiel au Père ; et par lui tout a été fait.
Pour nous les hommes, et pour notre salut, il descendit du ciel;
Par l’Esprit Saint, il a pris chair de la Vierge Marie, et s’est fait homme.
Crucifié pour nous sous Ponce Pilate, 
	il souffrit sa passion et fut mis au tombeau.
Il ressuscita le troisième jour, conformément aux Ecritures,
	et il monta au ciel; il est assis à la droite du Père.
Il reviendra dans la gloire, pour juger les vivants et les morts
	et son règne n’aura pas de fin.
Je crois en l’Esprit Saint, qui est Seigneur et qui donne la vie;
	il procède du Père et du Fils.
Avec le Père et le Fils, il reçoit même adoration et même gloire;
	il a parlé par les prophètes.
Je crois en l’Église, une, sainte, catholique et apostolique.
Je reconnais un seul baptême pour le pardon des péchés.
J’attends la résurrection des morts, et la vie du monde à venir.}
\end{alltt}

\paragraph{Offertoire}

\begin{paracol}{2}
\vv Dóminus vobíscum.\\
\rr Et cum spíritu tuo.\\
\vv Orémus.

\switchcolumn

\vv Le Seigneur soit avec vous.\\
\rr Et avec votre esprit.\\
\vv Prions.
\end{paracol}

\gscore{super_flumina}

\rubrique{Ps 136: 1}\\
\emph{Sur les fleuves de Babylone nous étions assis et nous pleurions en nous souvenant de toi, Sion.}

\begin{paracol}{2}

Súscipe, sancte Pater, omnípotens ætérne Deus, hanc immaculátam hóstiam, quam ego indígnus fámulus tuus óffero tibi Deo meo vivo et vero, pro innumerabílibus peccátis, et offensiónibus, et neglegéntiis meis, et pro ómnibus circumstántibus, sed et pro ómnibus fidélibus christiánis vivis atque defúnctis: ut mihi, et illis profíciat ad salútem in vitam ætérnam. Amen.

\switchcolumn

Recevez, Père saint, Dieu éternel et tout-puissant, cette hostie sans tache, que moi, votre indigne serviteur, je vous offre à vous, mon Dieu vivant et vrai, pour mes innombrables péchés, offenses et négligences, pour tous ceux qui m’entourent, ainsi que pour tous les fidèles chrétiens vivants et morts, afin qu’elle serve à mon salut et au leur pour la vie éternelle. Ainsi soit-il.

\switchcolumn*

Deus, qui humánæ substántiæ dignitátem mirabíliter condidísti, et mirabílius reformásti: da nobis per huius aquæ et vini mystérium, eius divinitátis esse consórtes, qui humanitátis nostræ fíeri dignátus est párticeps, Iesus Christus, Fílius tuus, Dóminus noster: Qui tecum vivit et regnat in unitáte Spíritus Sancti Deus: per ómnia sǽcula sæculórum. Amen.

\switchcolumn

Dieu, qui avez admirablement fondé la dignité de la nature humaine et l’avez restaurée plus admirablement encore : donnez-nous, par le mystère de cette eau et de ce vin, d’avoir part à la divinité de celui qui a daigné partager notre humanité, Jésus-Christ, votre Fils, notre Seigneur, qui, étant Dieu, vit et règne avec vous dans l’unité du Saint-Esprit, dans tous les siècles des siècles. Ainsi soit-il.

\switchcolumn*

Offérimus tibi, Dómine, cálicem salutáris, tuam deprecántes cleméntiam: ut in conspéctu divínæ maiestátis tuæ, pro nostra et totíus mundi salúte, cum odóre suavitátis ascéndat. Amen.
In spíritu humilitátis et in ánimo contríto suscipiámur a te, Dómine: et sic fiat sacrifícium nostrum in conspéctu tuo hódie, ut pláceat tibi, Dómine Deus.
Veni, sanctificátor omnípotens ætérne Deus: et béne dic hoc sacrifícium, tuo sancto nómini præparátum.

\switchcolumn

Nous vous offrons, Seigneur, le calice du salut, implorant votre clémence : qu’il s’élève en odeur de suavité devant votre divine majesté, pour notre salut et celui du monde entier. Ainsi soit-il.
En esprit d’humilité et le cœur contrit, puissions-nous être accueillis par vous, Seigneur : et que notre sacrifice ait lieu aujourd’hui devant vous de telle manière qu’il vous soit agréable, Seigneur Dieu.
Venez, Sanctificateur, Dieu éternel et tout-puissant, et bénissez ce sacrifice préparé pour la gloire de votre saint Nom.

\switchcolumn*

Lavábo inter innocéntes manus meas: et circúmdabo altáre tuum, Dómine: Ut áudiam vocem laudis, et enárrem univérsa mirabília tua. Dómine, diléxi decórem domus tuæ et locum habitatiónis glóriæ tuæ. Ne perdas cum ímpiis, Deus, ánimam meam, et cum viris sánguinum vitam meam: In quorum mánibus iniquitátes sunt: déxtera eórum repléta est munéribus. Ego autem in innocéntia mea ingréssus sum: rédime me et miserére mei. Pes meus stetit in dirécto: in ecclésiis benedícam te, Dómine.
Glória Patri, et Fílio, et Spirítui Sancto. Sicut erat in princípio, et nunc, et semper, et in sǽcula sæculórum. Amen.

\switchcolumn

Je laverai mes mains parmi les innocents, et je me tiendrai autour de Votre autel, Seigneur. Pour entendre la voix de Vos louanges, et pour raconter toutes Vos merveilles. Seigneur, j'ai aimé la beauté de Votre maison, et le lieu où habite Votre gloire. Ne perdez pas, ô Dieu, mon âme avec les impies, ni ma vie avec les hommes de sang qui ont l'iniquité dans les mains, et dont la droite est remplie de présents. Pour moi j'ai marché dans mon innocence : délivrez-moi et ayez pitié de moi. Mon pied s’est tenu dans le droit chemin : je Vous bénirai, Seigneur, dans les assemblées. Gloire au Père et au Fils, et au Saint-Esprit. Comme il était au commencement, maintenant et toujours, et dans les siècles des siècles. Ainsi soit-il.

\switchcolumn*

Súscipe, sancta Trínitas, hanc oblatiónem, quam tibi offérimus ob memóriam passiónis, resurrectiónis, et ascensiónis Iesu Christi, Dómini nostri: et in honórem beátæ Maríæ semper Vírginis, et beáti Ioannis Baptistæ, et sanctórum Apostolórum Petri et Pauli, et istórum et ómnium Sanctórum: ut illis profíciat ad honórem, nobis autem ad salútem: et illi pro nobis intercédere dignéntur in cælis, quorum memóriam ágimus in terris. Per eúndem Christum, Dóminum nostrum. Amen.

\switchcolumn

Recevez, Trinité Sainte, cette offrande que nous vous présentons en mémoire de la Passion, de la Résurrection et de l’Ascension de Jésus-Christ notre Seigneur ; et en l’honneur de la bienheureuse Marie toujours vierge, de saint Jean-Baptiste, des saints apôtres Pierre et Paul, de ceux-ci et de tous vos saints : qu’elle serve à leur honneur et à notre salut ; et qu’ils daignent intercéder au ciel pour nous qui faisons mémoire d’eux sur la terre. Par le même Christ notre Seigneur. Ainsi soit-il.

\end{paracol}
\newpage
\paragraph{Secrète}

\begin{paracol}{2}

\vv Oráte, fratres: ut meum ac vestrum sacrifícium acceptábile fiat apud Deum Patrem omnipoténtem.\\
\rr Suscípiat Dóminus sacrifícium de mánibus tuis ad laudem et glóriam nominis sui, ad utilitátem quoque nostram, totiúsque Ecclésiæ suæ sanctæ.\\
\vv Amen.

Cœléstem nobis prǽbeant hæc mystéria, quǽsumus, Dómine, medicínam: et vítia nostri cordis expúrgent.
Per Dóminum nostrum Iesum Christum, Fílium tuum: qui tecum vivit et regnat in unitáte Spíritus Sancti Deus, per ómnia sǽcula sæculórum.\\
\rr Amen.

\switchcolumn

\vv Priez, mes frères, afin que mon sacrifice, qui est aussi le vôtre, soit agréé par Dieu le Père tout-puissant.\\
\rr Que le Seigneur reçoive de vos mains le sacrifice, à la louange et à la gloire de son nom, et aussi pour notre bien et celui de toute sa sainte Église.\\
\vv Ainsi soit-il.

Que ces mystères, Seigneur, nous soient un remède céleste et expulsent les vices de notre cœur.
Par notre Seigneur Jésus-Christ, ton Fils, qui, étant Dieu, vit et règne avec toi, dans l’unité du Saint-Esprit, pour les siècles des siècles.\\
\rr Ainsi soit-il.

\end{paracol}

\paragraph{Préface}

\gresetinitiallines{0}
\gscore{04bor--toni_praefationum_(1._tonus_solemnis)--solesmes}
\gresetinitiallines{1}

\emph{\vv Le Seigneur soit avec vous.\\
\rr Et avec votre esprit.\\
\vv Élevons nos cœurs.\\
\rr Ils sont tournés vers le Seigneur.\\
\vv Rendons grâces au Seigneur notre Dieu.\\
\rr Cela est digne et juste.}

\begin{paracol}{2}

Vere dignum et iustum est, æquum et salutáre, nos tibi semper et ubíque grátias ágere: Dómine sancte, Pater omnípotens, ætérne Deus: Qui cum unigénito Fílio tuo et Spíritu Sancto unus es Deus, unus es Dóminus: non in uníus singularitáte persónæ, sed in uníus Trinitáte substántiæ. Quod enim de tua glória, revelánte te, crédimus, hoc de Fílio tuo, hoc de Spíritu Sancto sine differéntia discretiónis sentímus. Ut in confessióne veræ sempiternǽque Deitátis, et in persónis propríetas, et in esséntia únitas, et in maiestáte adorétur æquálitas. Quam laudant Angeli atque Archángeli, Chérubim quoque ac Séraphim: qui non cessant clamáre cotídie, una voce dicéntes:
\switchcolumn
Il est vraiment juste et nécessaire, c’est notre devoir et notre salut, de vous rendre grâces toujours et partout, Seigneur, Père saint, Dieu éternel et tout‑puissant. Avec votre Fils unique et le Saint-Esprit, vous êtes un seul Dieu, un seul Seigneur; non dans l’individualité d’une seule personne, mais dans la Trinité d’une seule substance. Car ce que nous croyons, sur la foi de votre révélation, au sujet de votre gloire, nous le pensons indistinctement et de votre Fils et de l’Esprit Saint, sans aucune différence ; en sorte que, dans la confession de la véritable et éternelle divinité, sont adorées et la propriété dans les Personnes, et l’unité dans l’essence, et l’égalité dans la majesté. C’est elle que louent les Anges et les Archanges, les Chérubins et les Séraphins, qui ne cessent de chanter chaque jour, disant d’une seule voix:
\end{paracol}
\gscore{05ky--sanctus_xi--solesmes}
\emph{Saint, Saint, Saint, le Seigneur, Dieu des Forces célestes. Le ciel et la terre sont remplis de votre Gloire. Hosanna au plus haut des cieux.
Béni soit celui Qui vient au Nom du Seigneur. Hosanna au plus haut des cieux.}
\newpage
\paragraph{Canon romain}

\begin{paracol}{2}

Te ígitur, clementíssime Pater, per Iesum Christum, Fílium tuum, Dóminum nostrum, súpplices rogámus, ac pétimus, uti accépta hábeas et benedícas, hæc \cc dona, hæc \cc múnera, hæc \cc sancta sacrifícia illibáta, in primis, quæ tibi offérimus pro Ecclésia tua sancta cathólica: quam pacificáre, custodíre, adunáre et régere dignéris toto orbe terrárum: una cum fámulo tuo Papa nostro et Antístite nostro et ómnibus orthodóxis, atque cathólicæ et apostólicæ fídei cultóribus.

\switchcolumn

Père très clément, c’est donc vous que nous prions, suppliants, et à qui nous demandons, par Jésus-Christ votre Fils, notre Seigneur, d’accepter et de bénir ces dons, ces présents, ces offrandes saintes et immaculées.
Tout d’abord, nous vous les offrons pour votre sainte Église catholique : daignez lui donner la paix, la protéger, la réunir et la gouverner par toute la terre ; et en même temps pour votre serviteur notre Pape , notre évêque , tous ceux qui enseignent la vraie doctrine, et ceux qui gardent la foi catholique et apostolique.

\switchcolumn*

Meménto, Dómine, famulórum famularúmque tuarum \rubrique{N.} et \rubrique{N.} et ómnium circumstántium, quorum tibi fides cógnita est et nota devótio, pro quibus tibi offérimus: vel qui tibi ófferunt hoc sacrifícium laudis, pro se suísque ómnibus: pro redemptióne animárum suárum, pro spe salútis et incolumitátis suæ: tibíque reddunt vota sua ætérno Deo, vivo et vero.

\switchcolumn

Souvenez-vous, Seigneur, de vos serviteurs et de vos servantes \rubrique{N.} et \rubrique{N.}, et de tous ceux qui nous entourent : vous connaissez leur foi, vous avez éprouvé leur attachement. Nous vous offrons ou ils vous offrent eux-mêmes ce sacrifice de louange, pour eux et pour tous les leurs, pour la rédemption de leurs âmes, dans l’espérance de leur salut et de leur intégrité ; et ils vous adressent leurs prières, à vous, Dieu éternel, vivant et vrai.

\switchcolumn*

Communicántes, et memóriam venerántes, in primis gloriósæ semper Vírginis Maríæ, Genetrícis Dei et Dómini nostri Iesu Christi: sed et beáti Ioseph, eiúsdem Vírginis Sponsi,
et beatórum Apostolórum ac Mártyrum tuórum, Petri et Pauli, Andréæ, Iacóbi, Ioánnis, Thomæ, Iacóbi, Philíppi, Bartholomǽi, Matthǽi, Simónis et Thaddǽi: Lini, Cleti, Cleméntis, Xysti, Cornélii, Cypriáni, Lauréntii, Chrysógoni, Ioánnis et Pauli, Cosmæ et Damiáni: et ómnium Sanctórum tuórum; quorum méritis precibúsque concédas, ut in ómnibus protectiónis tuæ muniámur auxílio. Per eúndem Christum, Dóminum nostrum. Amen.

\switchcolumn

Unis dans une même communion, nous vénérons d’abord la mémoire de la glorieuse Marie toujours vierge, mère de notre Dieu et Seigneur Jésus-Christ, puis celle du bienheureux Joseph, l’époux de la Vierge,
de vos bienheureux apôtres et martyrs, Pierre et Paul, André, Jacques, Jean, Thomas, Jacques, Philippe, Barthélémy, Matthieu, Simon et Jude, Lin, Clet, Clément, Sixte, Corneille, Cyprien, Laurent, Chrysogone, Jean et Paul, Côme et Damien, et de tous vos saints.
À leurs prières et par leurs mérites, accordez-nous d’être fortifiés en toute occasion par le secours de votre protection. Par le même Christ notre Seigneur. Ainsi soit-il.

\switchcolumn*

Hanc ígitur oblatiónem servitútis nostræ, sed et cunctæ famíliæ tuæ,
quǽsumus, Dómine, ut placátus accípias: diésque nostros in tua pace dispónas, atque ab ætérna damnatióne nos éripi, et in electórum tuórum iúbeas grege numerári. Per Christum, Dóminum nostrum. Amen.

\switchcolumn

Cette oblation donc de notre ministère, mais aussi de votre famille entière,
nous vous prions, Seigneur, de l’accepter avec bienveillance, de disposer nos jours dans votre paix, et d’ordonner que nous soyons arrachés à la damnation éternelle et comptés dans la troupe de vos élus. Par le Christ notre Seigneur. Ainsi soit-il.

\switchcolumn*

Quam oblatiónem tu, Deus, in ómnibus, quǽsumus, bene~\cc~díctam, adscríp~\cc~tam, ra~\cc~tam, rationábilem, acceptabilémque fácere dignéris: ut nobis Cor~\cc~pus, et San~\cc~guis fiat dilectíssimi Fílii tui, Dómini nostri Iesu Christi.

\switchcolumn

Cette oblation, ô Dieu, nous vous en prions, daignez la rendre en tout point bénie, approuvée, ratifiée, digne et agréable : afin qu’elle devienne pour nous le Corps et le Sang de votre Fils bien-aimé, notre Seigneur Jésus-Christ.

\switchcolumn*

Qui prídie quam paterétur, accépit panem in sanctas ac venerábiles manus suas, elevátis óculis in cælum ad te Deum, Patrem suum omnipoténtem, tibi grátias agens, bene~\cc~díxit, fregit, dedítque discípulis suis, dicens: Accípite, et manducáte ex hoc omnes.

\switchcolumn

La veille du jour où il a souffert, il a pris du pain dans ses mains saintes et vénérables et, les yeux levés au ciel vers vous, Dieu son Père tout-puissant, vous rendant grâces, l’a béni, rompu et donné à ses disciples, en disant :
Prenez et mangez tous de ceci :

\switchcolumn*

HOC EST ENIM CORPUS MEUM.

\switchcolumn

CAR CECI EST MON CORPS.

\switchcolumn*

Símili modo postquam cenátum est, accípiens et hunc præclárum Cálicem in sanctas ac venerábiles manus suas: item tibi grátias agens, bene~\cc~díxit, dedítque discípulis suis, dicens: Accípite, et bíbite ex eo omnes.

\switchcolumn

De même, après le repas, prenant aussi ce très glorieux calice dans ses mains saintes et vénérables, vous rendant grâces encore, il l’a béni et donné à ses disciples, en disant :
« Prenez, et buvez-en tous :

\switchcolumn*

HIC EST ENIM CALIX SANGUINIS MEI, NOVI ET ÆTERNI TESTAMENTI: MYSTERIUM FIDEI: QUI PRO VOBIS ET PRO MULTIS EFFUNDETUR IN REMISSIONEM PECCATORUM.
Hæc quotiescúmque fecéritis, in mei memóriam faciétis.

\switchcolumn

CAR CECI EST LE CALICE DE MON SANG,
CELUI DE L’ALLIANCE NOUVELLE ET ÉTERNELLE
– MYSTÈRE DE LA FOI –
QUI SERA RÉPANDU POUR VOUS ET POUR BEAUCOUP
EN RÉMISSION DES PÉCHÉS.
Chaque fois que vous ferez cela, vous le ferez en mémoire de moi.

\switchcolumn*

Unde et mémores, Dómine, nos servi tui, sed et plebs tua sancta, eiúsdem Christi Fílii tui, Dómini nostri, tam beátæ passiónis, nec non et ab ínferis resurrectiónis, sed et in cælos gloriósæ ascensiónis: offérimus præcláræ maiestáti tuæ de tuis donis ac datis, hóstiam \cc puram, hóstiam \cc sanctam, hóstiam \cc immaculátam, Panem \cc sanctum vitæ ætérnæ, et Cálicem \cc salútis perpétuæ.

\switchcolumn

C’est pourquoi, Seigneur, nous vos serviteurs, et aussi votre peuple saint, en mémoire de la bienheureuse Passion de votre Fils Jésus-Christ notre Seigneur, de sa Résurrection des enfers et aussi de sa glorieuse Ascension dans les cieux, nous présentons à votre sublime majesté cette offrande venant des biens que vous nous avez donnés : la victime pure, la victime sainte, la victime immaculée, le Pain sacré de la vie éternelle et le Calice de l’éternel salut.

\switchcolumn*

Supra quæ propítio ac seréno vultu respícere dignéris: et accépta habére, sicúti accépta habére dignátus es múnera púeri tui iusti Abel, et sacrifícium Patriárchæ nostri Abrahæ: et quod tibi óbtulit summus sacérdos tuus Melchísedech, sanctum sacrifícium, immaculátam hóstiam.

\switchcolumn

Sur ces offrandes daignez jeter un regard favorable et serein, et les accepter comme vous avez bien voulu accepter les présents de votre serviteur Abel le Juste, le sacrifice de notre patriarche Abraham, et celui que vous offrit votre grand prêtre Melchisédech, sacrifice saint, victime immaculée.

\switchcolumn*

Súpplices te rogámus, omnípotens Deus: iube hæc perférri per manus sancti Angeli tui in sublíme altáre tuum, in conspéctu divínæ maiestátis tuæ: ut, quotquot ex hac altáris participatióne sacrosánctum Fílii tui Cor\cc pus, et Sán\cc guinem sumpsérimus, omni benedictióne cælésti et grátia repleámur. Per eúndem Christum, Dóminum nostrum. Amen.

\switchcolumn

Suppliants, nous vous en prions, Dieu tout-puissant : ordonnez que ces offrandes soient portées par les mains de votre saint Ange sur votre sublime autel, en présence de votre majesté divine ; afin que, nous tous qui recevrons par cette participation de l’autel le Corps et le Sang très saints de votre Fils, nous soyons comblés de toute grâce et bénédiction céleste. Par le même Christ notre Seigneur. Ainsi soit-il.

\switchcolumn*

Meménto étiam, Dómine, famulórum famularúmque tuárum N. et N., qui nos præcessérunt cum signo fídei, et dórmiunt in somno pacis. Ipsis, Dómine, et ómnibus in Christo quiescéntibus locum refrigérii, lucis, et pacis, ut indúlgeas, deprecámur. Per eúndem Christum, Dóminum nostrum. Amen.

\switchcolumn

Souvenez-vous aussi, Seigneur, de vos serviteurs et de vos servantes N. et N., qui nous ont précédés avec le signe de la foi, et qui dorment du sommeil de la paix. À eux, Seigneur, et à tous ceux qui reposent dans le Christ, nous vous supplions d’accorder le lieu du rafraîchissement, de la lumière et de la paix. Par le même Christ notre Seigneur. Ainsi soit-il.

\switchcolumn*

Nobis quoque peccatóribus fámulis tuis, de multitúdine miseratiónum tuárum sperántibus, partem áliquam et societátem donáre dignéris, cum tuis sanctis Apóstolis et Martýribus: cum Ioánne, Stéphano, Matthía, Bárnaba, Ignátio, Alexándro, Marcellíno, Petro, Felicitáte, Perpétua, Agatha, Lúcia, Agnéte, Cæcília, Anastásia, et ómnibus Sanctis tuis: intra quorum nos consórtium, non æstimátor mériti, sed véniæ, quǽsumus, largítor admítte. Per Christum, Dóminum nostrum.

\switchcolumn

À nous aussi, pécheurs, vos serviteurs, qui espérons en l’abondance de vos miséricordes, daignez accorder quelque participation à la société de vos saints apôtres et martyrs, avec Jean, le Baptiste, Étienne , Mathias , Barnabé , Ignace , Alexandre , Marcellin, Pierre , Félicité , Perpétue , Agathe, Lucie , Agnès , Cécile , Anastasie et avec tous vos saints ; vous qui donnez largement et ne regardez pas au mérite, mais au pardon, nous vous en prions, admettez-nous dans leur compagnie. Par le Christ notre Seigneur. Ainsi soit-il.

\switchcolumn*

Per quem hæc ómnia, Dómine, semper bona creas, sanctí~\cc~ficas, viví~\cc~ficas, bene~\cc~dícis et præstas nobis.
Per ip~\cc~sum, et cum ip~\cc~so, et in ip~\cc~so, est tibi Deo Patri \cc omnipoténti, in unitáte Spíritus \cc Sancti,
omnis honor, et glória.
Per ómnia sǽcula sæculórum. \\
\rr Amen.

\switchcolumn

Par lui, Seigneur, vous ne cessez de créer tous ces biens, de les sanctifier, de les vivifier, de les bénir et de nous les donner.
Par lui, et avec lui, et en lui, est à vous, Dieu le Père tout-puissant, en l’unité du Saint Esprit,
tout honneur et toute gloire.
Pour les siècles des siècles.\\
\rr Ainsi soit-il.

\switchcolumn*

Orémus. \\
Præcéptis salutáribus móniti, et divína institutióne formáti audémus dícere:

Pater noster, qui es in cælis. Sanctificétur nomen tuum. Advéniat regnum tuum. Fiat volúntas tua, sicut in cælo et in terra. Panem nostrum quotidiánum da nobis hódie. Et dimítte nobis débita nostra, sicut et nos dimíttimus debitóribus nostris. Et ne nos indúcas in tentatiónem:\\
\rr Sed líbera nos a malo.\\
\vv Amen.

\switchcolumn

Prions. Éclairés par de salutaires prescriptions et formés par l’enseignement divin, nous osons dire :

Notre Père, qui êtes aux cieux, que votre Nom soit sanctifié, que votre règne arrive, que votre volonté soit faite sur la terre comme au ciel. Donnez-nous aujourd’hui notre pain de chaque jour, pardonnez-nous nos offenses, comme nous pardonnons à ceux qui nous ont offensés, et ne nous laissez entrer en tentation. \\
\rr Mais délivrez-nous du mal.\\
\vv Amen.

\switchcolumn*

Líbera nos, quǽsumus, Dómine, ab ómnibus malis, prætéritis, præséntibus et futúris: et intercedénte beáta et gloriósa semper Vírgine Dei Genetríce María, cum beátis Apóstolis tuis Petro et Paulo, atque Andréa, et ómnibus Sanctis, da propítius pacem in diébus nostris: ut, ope misericórdiæ tuæ adiúti, et a peccáto simus semper líberi et ab omni perturbatióne secúri.
Per eúndem Dóminum nostrum Iesum Christum, Fílium tuum.
Qui tecum vivit et regnat in unitáte Spíritus Sancti Deus.\\
\vv Per ómnia sǽcula sæculórum.\\
\rr Amen.\\
\vv Pax Dómini sit semper vobíscum.\\
\rr Et cum spíritu tuo.

\switchcolumn

Délivrez-nous, Seigneur, nous vous en prions, de tous les maux passés, présents et à venir ; et par l’intercession de la bienheureuse et glorieuse Marie toujours vierge, Mère de Dieu, avec vos bienheureux apôtres Pierre et Paul, André, et tous les saints, soyez-nous favorable et donnez la paix à notre temps, afin qu’aidés par votre abondante miséricorde, nous soyons à jamais libérés du péché et préservés de toutes sortes de troubles.
Par le même Jésus-Christ, votre Fils, notre Seigneur,
qui, étant Dieu, vit et règne avec vous dans l’unité du Saint-Esprit.\\
\vv Dans tous les siècles des siècles.\\
\rr Ainsi soit-il.\\
\vv Que la paix du Seigneur soit toujours avec vous.\\
\rr Et avec votre esprit.

\switchcolumn*

Hæc commíxtio, et consecrátio Córporis et Sánguinis Dómini nostri Iesu Christi, fiat accipiéntibus nobis in vitam ætérnam. Amen.

\switchcolumn

Que ce mélange sacramentel du corps et du sang de notre Seigneur Jésus-Christ, que nous allons recevoir, nous serve pour la vie éternelle. Ainsi soit-il.

\end{paracol}

\gscore{06ky--agnus_dei_xi--solesmes}

\emph{Agneau de Dieu, qui enlevez les péchés du monde : ayez pitié de nous.\\
Agneau de Dieu, qui enlevez les péchés du monde : ayez pitié de nous.\\
Agneau de Dieu, qui enlevez les péchés du monde : donnez-nous la paix.}

\paragraph{Communion}
\begin{paracol}{2}

Dómine Iesu Christe, qui dixísti Apóstolis tuis: Pacem relínquo vobis, pacem meam do vobis: ne respícias peccáta mea, sed fidem Ecclésiæ tuæ; eámque secúndum voluntátem tuam pacificáre et coadunáre dignéris: Qui vivis et regnas Deus per ómnia sǽcula sæculórum. Amen.

\switchcolumn

Seigneur Jésus-Christ, qui avez dit à vos apôtres : Je vous laisse la paix, Je vous donne ma paix, ne regardez pas mes péchés, mais la foi de votre Église ; et daignez, conformément à votre volonté, lui donner la paix et l’unité. Vous qui, étant Dieu, vivez et régnez dans tous les siècles des siècles. Ainsi soit-il.

\switchcolumn*

Dómine Iesu Christe, Fili Dei vivi, qui ex voluntáte Patris, cooperánte Spíritu Sancto, per mortem tuam mundum vivificásti: líbera me per hoc sacrosánctum Corpus et Sánguinem tuum ab ómnibus iniquitátibus meis, et univérsis malis: et fac me tuis semper inhærére mandátis, et a te numquam separári permíttas: Qui cum eódem Deo Patre et Spíritu Sancto vivis et regnas Deus in sǽcula sæculórum. Amen.

\switchcolumn

Seigneur Jésus-Christ, Fils du Dieu vivant, qui, selon la volonté du Père et avec la coopération de l’Esprit Saint, avez donné la vie au monde par votre mort ; libérez-moi par votre corps et votre sang sacrés de tous mes péchés et de tous les maux : faites que je m’attache toujours à vos commandements, et ne permettez pas que je sois jamais séparé de vous. Vous qui, étant Dieu, vivez et régnez avec le même Dieu le Père et le Saint-Esprit, dans les siècles des siècles. Ainsi soit-il.

\switchcolumn*

Percéptio Córporis tui, Dómine Iesu Christe, quod ego indígnus súmere præsúmo, non mihi provéniat in iudícium et condemnatiónem: sed pro tua pietáte prosit mihi ad tutaméntum mentis et córporis, et ad medélam percipiéndam: Qui vivis et regnas cum Deo Patre in unitáte Spíritus Sancti Deus, per ómnia sǽcula sæculórum. Amen.

\switchcolumn

Que la réception de votre corps, que j’ose prendre, tout indigne que je suis, Seigneur Jésus-Christ, n’entraîne pour moi ni jugement ni condamnation ; mais que, par votre bonté, elle serve de soutien et de remède à mon âme et à mon corps. Vous qui, étant Dieu, vivez et régnez avec Dieu le Père dans l’unité du Saint-Esprit, dans tous les siècles des siècles. Ainsi soit-il.

\switchcolumn*

Panem cæléstem accípiam, et nomen Dómini invocábo.
Dómine, non sum dignus, ut intres sub tectum meum: sed tantum dic verbo, et sanábitur ánima mea.

\switchcolumn

Je prendrai le Pain du ciel, et j’invoquerai le Nom du Seigneur.
Seigneur, je ne suis pas digne que vous entriez sous mon toit, mais dites seulement une parole et mon âme sera guérie.

\switchcolumn*

\rubrique{Le prêtre communie au Corps :}\\
Corpus Dómini nostri Iesu Christi custódiat ánimam meam in vitam ætérnam. Amen.

\switchcolumn

~\\
Que le corps de notre Seigneur Jésus-Christ garde mon âme pour la vie éternelle. Ainsi soit-il.

\switchcolumn*

Quid retríbuam Dómino pro ómnibus, quæ retríbuit mihi? Cálicem salutáris accípiam, et nomen Dómini invocábo. Laudans invocábo Dóminum, et ab inimícis meis salvus ero.\\
\rubrique{Le prêtre communie au Sang :}\\
Sanguis Dómini nostri Iesu Christi custódiat ánimam meam in vitam ætérnam. Amen.

\switchcolumn

Que rendrai-je au Seigneur pour tous ses bienfaits à mon égard ? Je prendrai le calice du salut et j’invoquerai le Nom du Seigneur. J’invoquerai le Nom du Seigneur en le louant, et je serai sauvé de mes ennemis.
~\\
Que le sang de notre Seigneur Jésus-Christ garde mon âme pour la vie éternelle. Ainsi soit-il.

\switchcolumn*

Quod ore súmpsimus, Dómine, pura mente capiámus: et de múnere temporáli fiat nobis remédium sempitérnum.
Corpus tuum, Dómine, quod sumpsi, et Sanguis, quem potávi, adhǽreat viscéribus meis: et præsta; ut in me non remáneat scélerum mácula, quem pura et sancta refecérunt sacraménta: Qui vivis et regnas in sǽcula sæculórum. Amen.

\switchcolumn

Ce que nous avons reçu par la bouche, Seigneur, que nous l’embrassions d’une âme pure, et que de ce don temporel nous vienne un remède éternel.
Que votre corps que j’ai pris et votre sang que j’ai bu, Seigneur, adhèrent à mes entrailles ; et faites que le péché ne laisse aucune tache en moi, que de purs et saints mystères ont restauré. Vous qui vivez et régnez dans les siècles des siècles. Ainsi soit-il.

\switchcolumn*

\vv Ecce Agnus Dei, ecce qui tollit peccata mundi.\\
\rubrique{On répond trois fois, en se frappant la poitrine :}\\
\rr Dómine, non sum dignus, ut intres sub tectum meum: sed tantum dic verbo, et sanábitur ánima mea.

\switchcolumn

\vv Voici l'Agneau de Dieu, celui qui porte les péchés du monde.\\
~\\
\rr Seigneur, je ne suis pas digne que vous entriez sous mon toit, mais dites seulement une parole et mon âme sera guérie.
\end{paracol}

\gscore{memento_verbi}

\rubrique{Ps 118: 49-50}\\
\emph{\rr Rappelle-toi ta parole à ton serviteur, celle dont tu fis mon espoir. Elle est ma consolation dans mon épreuve : ta promesse me fait vivre.\\
\vv \rubrique{\emph{1. }} Heureux les hommes intègres dans leurs voies qui marchent suivant la loi du Seigneur !\capsaut
\vv \rubrique{\emph{2. }} Heureux ceux qui gardent ses exigences, ils le cherchent de tout coeur !\capsaut
\vv \rubrique{\emph{3. }} Mon âme est collée à la poussière ; fais-moi vivre selon ta parole.\capsaut
\vv \rubrique{\emph{4. }} La tristesse m'arrache des larmes : relève-moi selon ta parole.\capsaut
\vv \rubrique{\emph{5. }} Que vienne à moi, Seigneur, ton amour, et ton salut, selon ta promesse.\capsaut
\vv \rubrique{\emph{6. }} À me voir, ceux qui te craignent se réjouissent, car j'espère en ta parole.\capsaut
\vv \rubrique{\emph{7. }} Que j'aie pour consolation ton amour selon tes promesses à ton serviteur !\capsaut
\vv \rubrique{\emph{8. }} Usé par l'attente du salut, j'espère encore ta parole.\capsaut
\vv \rubrique{\emph{9. }} L'oeil usé d'attendre tes promesses, j'ai dit : « Quand vas-tu me consoler ? »\capsaut
\vv \rubrique{\emph{10. }} Toi, mon abri, mon bouclier ! J'espère en ta parole.
}
\newpage
\paragraph{Postcommunion}
\begin{paracol}{2}

\vv Dóminus vobíscum.\\
\rr Et cum spíritu tuo.\\
\vv Orémus.\\
Ut sacris, Dómine, reddámur digni munéribus: fac nos, quǽsumus, tuis semper obœdíre mandátis.
Per Dóminum nostrum Iesum Christum, Fílium tuum: qui tecum vivit et regnat in unitáte Spíritus Sancti Deus, per ómnia sǽcula sæculórum.\\
\rr Amen.

\switchcolumn

\vv Le Seigneur soit avec vous.\\
\rr Et avec votre esprit.\\
\vv Prions.\\
Pour nous rendre dignes, Seigneur, de vos dons sacrés, faite que nous obéissions toujours à vos commandements.
Par notre Seigneur Jésus-Christ, ton Fils, qui, étant Dieu, vit et règne avec toi, dans l’unité du Saint-Esprit, pour les siècles des siècles.\\
\rr Ainsi soit-il.

\end{paracol}

\paragraph{Envoi}

\begin{paracol}{2}

\vv Dóminus vobíscum.\\
\rr Et cum spíritu tuo.

\switchcolumn

\vv Le Seigneur soit avec vous.\\
\rr Et avec votre esprit.

\end{paracol}

\gscore{07ky--ite_xi--solesmes}

\emph{\vv Allez, c'est l'envoi.\\
\rr Nous rendons grâces à Dieu.}

\begin{paracol}{2}

Pláceat tibi, sancta Trínitas, obséquium servitútis meæ: et præsta; ut sacrifícium, quod óculis tuæ maiestátis indígnus óbtuli, tibi sit acceptábile, mihíque et ómnibus, pro quibus illud óbtuli, sit, te miseránte, propitiábile. Per Christum, Dóminum nostrum. Amen.

\switchcolumn

Agréez, Trinité Sainte, l’hommage de mon ministère : et faites que le sacrifice que, malgré mon indignité, j’ai présenté aux regards de votre Majesté, vous soit agréable, et que, par votre miséricorde, il puisse attirer votre faveur sur moi et sur tous ceux pour lesquels je vous l’ai offert. Par le Christ notre Seigneur. Ainsi soit-il.

\switchcolumn*

Benedícat vos omnípotens Deus,
Pater, et Fílius, \cc et Spíritus Sanctus. \\
\rr Amen.

\switchcolumn

Que le Dieu tout-puissant vous bénisse,
le Père, le Fils, et le Saint Esprit.\\
\rr Ainsi soit-il.

\end{paracol}
\newpage
\paragraph{Dernier Évangile}

\begin{paracol}{2}

\vv Dóminus vobíscum.\\
\rr Et cum spíritu tuo.\\
\vv Inítium \cc sancti Evangélii secúndum Ioánnem\\
\rr Glória tibi, Dómine.

\rubrique{Jn. 1, 1-14}\\
In princípio erat Verbum, et Verbum erat apud Deum, et Deus erat Verbum. Hoc erat in princípio apud Deum. Omnia per ipsum facta sunt: et sine ipso factum est nihil, quod factum est: in ipso vita erat, et vita erat lux hóminum: et lux in ténebris lucet, et ténebræ eam non comprehendérunt.
Fuit homo missus a Deo, cui nomen erat Ioánnes. Hic venit in testimónium, ut testimónium perhibéret de lúmine, ut omnes créderent per illum. Non erat ille lux, sed ut testimónium perhibéret de lúmine.
Erat lux vera, quæ illúminat omnem hóminem veniéntem in hunc mundum. In mundo erat, et mundus per ipsum factus est, et mundus eum non cognóvit. In própria venit, et sui eum non recepérunt. Quotquot autem recepérunt eum, dedit eis potestátem fílios Dei fíeri, his, qui credunt in nómine eius: qui non ex sanguínibus, neque ex voluntáte carnis, neque ex voluntáte viri, sed ex Deo nati sunt. Genuflectit dicens: Et Verbum caro factum est, Et surgens prosequitur: et habitávit in nobis: et vídimus glóriam eius, glóriam quasi Unigéniti a Patre, plenum grátiæ et
veritátis.\\
\rr Deo grátias.

\switchcolumn

\vv Le Seigneur soit avec vous.\\
\rr Et avec votre esprit.\\
\vv Commencement du saint Évangile selon saint Jean.\\
\rr Gloire à vous, Seigneur.\\

Au commencement était le Verbe, et le Verbe était auprès de Dieu et le Verbe était Dieu. Il était au commencement auprès de Dieu. Toutes choses ont été faites par lui, et rien de ce qui a été fait n’a été fait sans lui. En lui était la vie, et la vie était la lumière des hommes ; et la lumière luit dans les ténèbres, et les ténèbres ne l’ont point comprise.
Il y eut un homme, envoyé de Dieu, appelé Jean. Il vint en témoin pour rendre témoignage à la lumière, afin que tous crussent par lui. Il n’était pas lui-même la lumière, mais il vint pour rendre témoignage à la lumière.
Celui-là était la vraie lumière qui éclaire tout homme venant en ce monde. Il était dans le monde, et le monde a été fait par lui, et le monde ne l’a pas reconnu. Il est venu chez lui, et les siens ne l’ont pas reçu.
Mais à tous ceux qui l’ont reçu, il a donné le pouvoir de devenir enfants de Dieu, à ceux qui croient en son nom : qui ne sont point nés du sang, ni de la volonté de la chair, ni de la volonté de l’homme, mais de Dieu. On fléchit le genou avec le prêtre, qui dit : Et le Verbe s’est fait chair, Et se relevant, le prêtre poursuit : et il a habité parmi nous, et nous avons vu sa gloire, qui est la gloire du Fils unique du Père, plein de grâce et de vérité.\\
\rr Nous rendons grâces à Dieu.

\end{paracol}

\end{document}