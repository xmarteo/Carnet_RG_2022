\documentclass[twoside]{article}


\usepackage[paperwidth=150mm, paperheight=230mm]{geometry}
\usepackage{fontspec}
%\usepackage[latin1]{inputenc}
\usepackage[french]{babel}
\usepackage[strict]{changepage}
\usepackage{fancyhdr}
\usepackage{paracol}
\usepackage{tableof}
\usepackage{setspace}
\usepackage{alltt}
\usepackage{titlesec}
\usepackage{xcolor}
\usepackage{xstring}
\usepackage{parskip}
\usepackage{enumitem}
\usepackage{etoolbox}
\usepackage{needspace}

%%%%%%%%%%%%%%%%%%%%%%%%%%%%%%%%%%%%%%%%%%%%%%%%%%% Mise en page %%%%%%%%%%%%%%%%%%%%%%%%%%%%%%
% on numérote les nbp par page et non globalement
\usepackage[perpage]{footmisc}

% définition des en-têtes et pieds de page
\pagestyle{empty}
\fancyhead{}
\fancyfoot{}
\renewcommand{\headrulewidth}{0pt}
\setlength{\headheight}{0pt}

% la commande titres permet de changer les titres de gauche et de droite.
\newcommand{\titres}[2]{
	\renewcommand{\rightmark}{\textcolor{red}{\sc #2}}
	\renewcommand{\leftmark}{\textcolor{red}{\sc #1}}
}
\titres{}{}

% pas d'indentation
\setlength{\parindent}{0mm}

\geometry{
inner=20mm,
outer=20mm,
top=10mm,
bottom=25mm,
headsep=0mm,
}

\twosided[p]

%%%%%%%%%%%%%%%%%%%%%%%%%%%%%%%%%%%%%%%%%%%%%%%%% Options gregorio %%%%%%%%%%%%%%%%%%%%%%%%%

\usepackage[autocompile]{gregoriotex}
%\usepackage{gregoriotex}

\definecolor{gregoriocolor}{RGB}{215,65,29}

%% style général de gregorio :
% lignes rouges, commenter pour du noir
%\gresetlinecolor{gregoriocolor}

% texte <alt> (au-dessus de la portée) en rouge et en petit, avec réglage de sa position verticale
\grechangestyle{abovelinestext}{\color{gregoriocolor}\footnotesize}
\newcommand{\altraise}{-1mm}
\grechangedim{abovelinestextraise}{\altraise}{scalable}

% taille des initiales
\newcommand{\initialsize}[1]{
    \grechangestyle{initial}{\fontspec{ZallmanCaps}\fontsize{#1}{#1}\selectfont}
}
\newcommand{\defaultinitialsize}{32}
\initialsize{\defaultinitialsize}
% espace avant et après les initiales
\newcommand{\initialspace}[1]{
  \grechangedim{afterinitialshift}{#1}{scalable}
  \grechangedim{beforeinitialshift}{#1}{scalable}
}
\newcommand{\defaultinitialspace}{0cm}
\initialspace{\defaultinitialspace}


% on définit le système qui capture des headers pour générer des annotations
% cette commande sera appelée pour définir des abréviations ou autres substitutions
\newcommand{\resultat}{}
\newcommand{\abbrev}[3]{
  \IfSubStr{#1}{#2}{ \renewcommand{\resultat}{#3} }{}
}
\newcommand{\officepartannotation}[1]{
  \renewcommand{\resultat}{#1}
  \abbrev{#1}{ntro}{Intr.}
  \abbrev{#1}{re}{Resp.}
  \abbrev{#1}{espo}{Resp.}
  \abbrev{#1}{adu}{Gr.}
  \abbrev{#1}{ll}{All.}
  \abbrev{#1}{act}{Tract.}
  \abbrev{#1}{equen}{Seq.}
  \abbrev{#1}{ffert}{Off.}
  \abbrev{#1}{ommun}{Co.}
  \abbrev{#1}{an}{Ant.}
  \abbrev{#1}{ntiph}{Ant.}
  \abbrev{#1}{ntic}{Cant.}
  \abbrev{#1}{ymn}{Hy.}
  \abbrev{#1}{salm}{}
  \abbrev{#1}{Toni Communes}{}
  \abbrev{#1}{yrial}{}
  \greannotation{\resultat}
}
\newcommand{\modeannotation}[1]{
  \renewcommand{\resultat}{#1}
  \abbrev{#1}{1}{ {\sc i} }
  \abbrev{#1}{2}{ {\sc ii} }
  \abbrev{#1}{3}{ {\sc iii} }
  \abbrev{#1}{4}{ {\sc iv} }
  \abbrev{#1}{5}{ {\sc v} }
  \abbrev{#1}{6}{ {\sc vi} }
  \abbrev{#1}{7}{ {\sc vii} }
  \abbrev{#1}{8}{ {\sc viii} }
  \greannotation{\resultat}
}
\gresetheadercapture{office-part}{officepartannotation}{}
\gresetheadercapture{mode}{modeannotation}{string}

%%%%%%%%%%%%%%%%%%%%%%%%%%%%%%%%%%%%%%%%%%%%%% Graphisme %%%%%%%%%%%%%%%%%%%%%%%%%%%
% on définit l'échelle générale

\newcommand{\echelle}{1}

% on centre les titres et on ne les numérote pas
\titleformat{\section}[block]{\Large\filcenter\sc}{}{}{}
\titleformat{\subsection}[block]{\large\filcenter\sc}{}{}{}
\titleformat{\paragraph}[block]{\filcenter\sc}{}{}{}
\setcounter{secnumdepth}{0}
% on diminue l'espace avant les titres
\titlespacing*{\paragraph}{0pt}{1.8ex plus .4ex minus .4ex}{1.2ex plus .2ex minus .2ex}

% commandes versets, repons et croix
\newcommand{\vv}{\textcolor{gregoriocolor}{\fontspec[Scale=\echelle]{Charis SIL}℣.\hspace{3mm}}}
\newcommand{\rr}{\textcolor{gregoriocolor}{\fontspec[Scale=\echelle]{Charis SIL}℟.\hspace{3mm}}}
\newcommand{\cc}{\textcolor{gregoriocolor}{\fontspec[Scale=\echelle]{FreeSerif}\symbol{"2720}~}}
\renewcommand{\aa}{\textcolor{gregoriocolor}{\fontspec[Scale=\echelle]{Charis SIL}\Abar.\hspace{3mm}}}

% commandes diverses
\newcommand{\antiphona}{\textcolor{gregoriocolor}{\noindent Antiphona.\hspace{4mm}}}
\newcommand{\antienne}{\textcolor{gregoriocolor}{\noindent Antienne.\hspace{4mm}}}
% rubrique
\newcommand{\rubrique}[1]{\textcolor{gregoriocolor}{\emph{#1}}}
% pour afficher du texte noir roman au milieu d'une rubrique
\newcommand{\normaltext}[1]{{\normalfont\normalcolor #1}}

\newcommand{\saut}{\hspace{1cm}}
\newcommand{\capsaut}{\hspace{3mm}}
% pour affichier 1 en rouge et un peu d'espace
\newcommand{\un}{{\color{gregoriocolor} 1~~~}}


% abréviations
\newcommand{\tpalleluia}{\rubrique{(T.P.} \mbox{Allelúia.\rubrique{)}}}
\newcommand{\tpalleluiafr}{\rubrique{(T.P.} \mbox{Alléluia.\rubrique{)}}}

\newcommand{\tqomittitur}{{\small \rubrique{(In Tempore Quadragesimæ ommittitur} Allelúia.\rubrique{)}}}
\newcommand{\careme}{{\small \rubrique{(Pendant le Carême on omet l'}Alléluia.\rubrique{)}}}

% environnement hymne : alltt + normalfont + marges custom
\newenvironment{hymne}
  {
  \begin{adjustwidth}{1.6cm}{1mm}
  \begin{alltt}\normalfont
  }
  {
  \end{alltt}
  \end{adjustwidth}
  }
  
% la commande \u permet de souligner les inflexions
\let\u\textbf

% on définit la police par défaut
\setmainfont[Ligatures=TeX, Scale=\echelle]{Charis SIL}
%renderer=ICU a l'air de ne plus marcher...
%\setmainfont[Renderer=ICU, Ligatures=TeX, Scale=\echelle]{Charis SIL}
\setstretch{0.9}


%%%%%%%%%%%%%%%% Commandes de mise en forme %%%%%%%%%%%%%%%%

\newcommand{\lectioresponsorium}[8]{
	\needspace{3\baselineskip}
	\paragraph{#1}

	\vv Jube, domne, benedícere.\\
	\vv #3.\\
	\rr Amen.

	#5

	\vv Tu autem, Dómine, miserére nobis.\\
	\rr Deo grátias.

	\ifblank{#7}{}{\gregorioscore{gabc/#7}}

	\switchcolumn
	%\needspace{3\baselineskip}
	\paragraph{#2}

	\vv Veuillez, maître, bénir.\\
	\vv #4.\\
	\rr Amen.

	#6

	\vv Et toi Seigneur, prends pitié de nous.\\
	\rr Nous rendons grâces à Dieu.

	\ifblank{#8}{}{#8}

	\switchcolumn*

}

\newcommand{\versiculus}[2]{
	\vv #1.\\
	\rr #2.\\
}

\newcommand{\versiculusabsolutio}[6]{

	\paragraph{Versiculus et Absolutio}

	\versiculus{#1}{#2}
	\vv Pater noster... \rubrique{(secrètement)} Et ne nos indúcas in tentatiónem. \\
	\rr Sed líbera nos a malo. \\
	\vv #3. \rr Amen.

	\switchcolumn

	\paragraph{Versicule et Absolution}

	\versiculus{#4}{#5}
	\vv Notre Père... Et ne nous laisse pas entrer en tentation. \\
	\rr Mais délivre-nous du mal. \\
	\vv #6. \rr Amen.

	\switchcolumn*

}

\newcommand{\versetGloireAuPere}{
	\vv Gloire au Père, au Fils, et au Saint-Esprit.
}

\newcommand{\gscore}[1]{
	\gregorioscore{gabc/#1}
}

\newcommand{\smallscore}[1]{
	\gresetinitiallines{0}
	\gscore{#1}
	\gresetinitiallines{1}
}

\begin{document}

\null \newpage

\sloppy

\begin{paracol}[1]{2}

\begin{center}\begin{doublespace}

{\fontspec[Scale=\echelle]{Futura Book}
\MakeUppercase{\Large S. Raphaëlis Archangeli \\ ad Matutinum}\\
juxta usum antiquiorem ritus romani}
\end{doublespace}\end{center}
\selectlanguage{french}

\smallscore{domine_labia_mea}
~~

\switchcolumn

\begin{center}\begin{doublespace}
{\fontspec[Scale=\echelle]{Futura Book}
\MakeUppercase{\Large Matines de la fête de Saint Raphaël Archange}\\
selon l'usage ancien du rite romain}
\end{doublespace}\end{center}

~~

~~

\vv Seigneur, ouvre mes lèvres.

\rr Et ma bouche annoncera ta louange.

\vv Dieu, viens à mon aide.

\rr Seigneur, viens vite à mon secours.

\vv Gloire au Père, au Fils, et au Saint-Esprit.

\rr Comme il était au commencement, maintenant et toujours, et dans les siècles des siècles. Amen. Alléluia.

\switchcolumn*

\paragraph{Invitatorium}

\gscore{i}
\gscore{ip}

\switchcolumn

\paragraph{Invitatoire}

TODO

\switchcolumn*

\paragraph{Hymnus}

\gscore{h}

\switchcolumn

\paragraph{Hymne}

O Christ, gloire des saints Anges,\\
Du genre humain, Auteur et Rédempteur,\\
Aux heureux sièges des habitants du ciel,\\
Daignez nous faire monter.\\
~\\
Que l’Ange médecin de notre santé,\\
Raphaël, du ciel nous vienne en aide,\\
Guérisse tous les malades et, de la vie,\\
Dirige les actes hésitants.\\
~\\
Que la Vierge, Reine de paix et Mère de lumière,\\
Ainsi que le chœur sacré des Anges\\
Toujours nous assiste, avec la cour royale\\
Du ciel étincelant.\\
~\\
Qu’elle nous fasse ce don, l’heureuse Déité,\\
Du Père et du Fils et tout pareillement\\
Du Saint-Esprit, dont résonne en tous lieux,\\
La gloire en ce monde.\\
Ainsi soit-il.

\switchcolumn*

\subsection{In I Nocturno}

\switchcolumn

\subsection{Premier Nocturne}

\switchcolumn*

\paragraph{Psalmus 8}

\gscore{a1_1a2}

\begin{enumerate}[wide, itemsep=0mm, labelwidth=!, labelindent=0pt, label=\color{gregoriocolor}\theenumi]
\item Dómine, \textbf{Dó}minus \textbf{nos}ter,~* quam admirábile est nomen tuum in uni\textit{vér}\textit{sa} \textbf{ter}ra!
\item Quóniam eleváta est magnifi\textbf{cén}tia \textbf{tu}a,~* \textit{su}\textit{per} \textbf{cæ}los.
\item Ex ore infántium et lacténtium perfecísti laudem propter ini\textbf{mí}cos \textbf{tu}os,~* ut déstruas inimícum \textit{et} \textit{ul}\textbf{tó}rem.
\item Quóniam vidébo cælos tuos, ópera digi\textbf{tó}rum tu\textbf{ó}rum:~* lunam et stellas, quæ \textit{tu} \textit{fun}\textbf{dás}ti.
\item Quid est homo quod \textbf{me}mor es \textbf{e}jus?~* aut fílius hóminis, quóniam ví\textit{si}\textit{tas} \textbf{e}um?
\item Minuísti eum paulo minus ab Angelis,~† glória et honóre coro\textbf{nás}ti \textbf{e}um:~* et constituísti eum super ópera mánu\textit{um} \textit{tu}\textbf{á}rum.
\item Omnia subjecísti sub \textbf{pé}dibus \textbf{e}jus,~* oves et boves univérsas: ínsuper et pé\textit{co}\textit{ra} \textbf{cam}pi.
\item Vólucres cæli, et \textbf{pi}sces \textbf{ma}ris,~* qui perámbulant sé\textit{mi}\textit{tas} \textbf{ma}ris.
\item Dómine, \textbf{Dó}minus \textbf{nos}ter,~* quam admirábile est nomen tuum in uni\textit{vér}\textit{sa} \textbf{ter}ra!
\item Glória \textbf{Pa}tri, et \textbf{Fí}lio,~* et Spirí\textit{tu}\textit{i} \textbf{Sanc}to.
\item Sicut erat in princípio, et \textbf{nunc}, et \textbf{sem}per,~* et in sǽcula sæcu\textit{ló}\textit{rum}. \textbf{A}men.
\end{enumerate}

\switchcolumn

\paragraph{Psaume 8}
\aa Tobie étant sorti, trouva un jeune homme portant ceinture et comme tout préparé pour un voyage et, sans savoir que c’était un ange, il le salua.

\gscore{1_a2}

\begin{enumerate}[wide, itemsep=0mm, labelwidth=!, labelindent=0pt, label=\color{gregoriocolor}\theenumi]

\end{enumerate}

\switchcolumn*
\paragraph{Psalmus 10}

\gscore{a2_2d}

\begin{enumerate}[wide, itemsep=0mm, labelwidth=!, labelindent=0pt, label=\color{gregoriocolor}\theenumi]
\item In Dómino confído:~† quómodo dícitis ánimæ \textbf{me}æ:~* Tránsmigra in montem sic\textit{ut} \textbf{pas}ser?
\item Quóniam ecce peccatóres intendérunt arcum,~† paravérunt sagíttas suas in \textbf{phá}retra,~* ut sagíttent in obscúro rec\textit{tos} \textbf{cor}de.
\item Quóniam quæ perfecísti, destru\textbf{xé}runt:~* justus autem \textit{quid} \textbf{fe}cit?
\item Dóminus in templo sancto \textbf{su}o,~* Dóminus in cælo se\textit{des} \textbf{e}jus.
\item Oculi ejus in páuperem re\textbf{spí}ciunt:~* pálpebræ ejus intérrogant fíli\textit{os} \textbf{hó}minum.
\item Dóminus intérrogat justum et \textbf{ím}pium:~* qui autem díligit iniquitátem, odit áni\textit{mam} \textbf{su}am.
\item Pluet super peccatóres \textbf{lá}queos:~* ignis, et sulphur, et spíritus procellárum pars cálicis \textit{e}\textbf{ó}rum.
\item Quóniam justus Dóminus, et justítias di\textbf{lé}xit:~* æquitátem vidit vul\textit{tus} \textbf{e}jus.
\item Glória Patri, et \textbf{Fí}lio,~* et Spirítu\textit{i} \textbf{Sanc}to.
\item Sicut erat in princípio, et nunc, et \textbf{sem}per,~* et in sǽcula sæculó\textit{rum}. \textbf{A}men.
\end{enumerate}

\switchcolumn

\paragraph{Psaume 10}
\aa L’ange Raphaël, cachant sa personnalité, dit : Je suis Azarias, fils du grand Ananias.

\gscore{2}


\begin{enumerate}[wide, itemsep=0mm, labelwidth=!, labelindent=0pt, label=\color{gregoriocolor}\theenumi]

\end{enumerate}

\switchcolumn*

\paragraph{Psalmus 14}

\gscore{a3_3a}

\begin{enumerate}[wide, itemsep=0mm, labelwidth=!, labelindent=0pt, label=\color{gregoriocolor}\theenumi]
\item Dómine, quis habitábit in taber\textbf{ná}culo \textbf{tu}o?~* aut quis requiéscet in monte \textbf{sanc}to \textbf{tu}o?
\item Qui ingréditur \textbf{si}ne \textbf{má}\textbf{cu}la,~* et ope\textbf{rá}tur jus\textbf{tí}tiam:
\item Qui lóquitur veritátem in \textbf{cor}de \textbf{su}o,~* qui non egit dolum in \textbf{lin}gua \textbf{su}a:
\item Nec fecit próximo \textbf{su}o \textbf{ma}lum,~* et oppróbrium non accépit advérsus \textbf{pró}ximos \textbf{su}os.
\item Ad níhilum dedúctus est in conspéctu \textbf{e}jus ma\textbf{lí}gnus:~* timéntes autem Dómi\textbf{num} glo\textbf{rí}ficat:
\item Qui jurat próximo suo, \textbf{et} non \textbf{dé}\textbf{ci}pit,~* qui pecúniam suam non dedit ad usúram, et múnera super innocéntem \textbf{non} ac\textbf{cé}pit.
\item Qui \textbf{fa}\textbf{cit} hæc:~* non movébitur \textbf{in} æ\textbf{tér}num.
\item Glória \textbf{Pa}tri, et \textbf{Fí}\textbf{li}o,~* et Spi\textbf{rí}tui \textbf{Sanc}to.
\item Sicut erat in princípio, et \textbf{nunc}, et \textbf{sem}per,~* et in sǽcula sæcu\textbf{ló}rum. \textbf{A}men.
\end{enumerate}

\switchcolumn

\paragraph{Psaume 14}
\aa Je conduirai ton fils sain et sauf au pays des Mèdes et te le ramènerai de même, alléluia.

\gscore{3_si_a}

\begin{enumerate}[wide, itemsep=0mm, labelwidth=!, labelindent=0pt, label=\color{gregoriocolor}\theenumi]

\end{enumerate}

\switchcolumn*

\versiculusabsolutio
	{Dáta sunt Angelo incénsa multa}
	{Ut adoleret ea ante altáre áureum, quod est ante óculos Domini}
	{Exáudi, Dómine Iesu Christe, preces servórum tuórum, et miserére nobis: Qui cum Patre et Spíritu Sancto vivis et regnas in sǽcula sæculórum}
	{On donna à l’ange beaucoup de parfums}
	{Pour qu’il les brûlât devant l’autel d’or qui est sous les yeux du Seigneur}
	{Exaucez, Seigneur Jésus-Christ, les prières de vos serviteurs, et ayez pitié de nous, vous qui vivez et régnez avec le Père et le Saint-Esprit, dans les siècles des siècles}

\lectioresponsorium
	{Lectio \textsc{i}}
	{Première lecture}
	{Benedictióne perpétua benedícat nos Pater ætérnus}
	{Que le Père éternel nous bénisse d'une bénédiction perpétuelle}
	{
		De libro Tobíæ

		Vocávit ad se Tobías fílium suum, dixítque ei: Quid póssumus dare viro isti sancto, qui venit tecum?
		Respóndens Tobías, dixit patri suo: Pater, quam mercédem dábimus ei? aut quid dignum póterit esse benefíciis ejus?
		Me duxit et redúxit sanum, pecúniam a Gabélo ipse recépit, uxórem ipse me habére fecit, et dæmónium ab ea ipse compéscuit: gáudium paréntibus ejus fecit, meípsum a devoratióne piscis erípuit, te quoque vidére fecit lumen cæli, et bonis ómnibus per eum repléti sumus. Quid illi ad hæc potérimus dignum dare?
		Sed peto te, pater mi, ut roges eum, si forte dignábitur medietátem de ómnibus quæ alláta sunt, sibi assúmere.
	}
	{	\rubrique{Tob 12 : 1-4}
		
		TODO
	
	}
	{r1}
	{\rr En ce temps-là, les prières de tous les deux furent exaucées en la glorieuse présence du Dieu souverain ; \\
	\GreSpecial{*} Et le saint Ange du Seigneur, Raphaël, fut envoyé pour les guérir tous deux, eux dont les prières avaient été présentées au Seigneur en même temps. \\
	\vv Tobie et Sara, en proie à l’affliction, commencèrent à prier avec larmes.\\
	\GreSpecial{*} Et le saint Ange du Seigneur, Raphaël, fut envoyé pour les guérir tous deux, eux dont les prières avaient été présentées au Seigneur en même temps.}

\lectioresponsorium
	{Lectio \textsc{ii}}
	{Deuxième lecture}
	{Unigénitus Dei Fílius nos benedícere et adjuváre dignétur}
	{Que le Père éternel nous bénisse d'une bénédiction perpétuelle}
	{
		Et vocántes eum, pater scílicet et fílius, tulérunt eum in partem; et rogáre cœpérunt ut dignarétur dimídiam partem ómnium quæ attúlerant, accéptam habére.
		Tunc dixit eis occúlte: Benedícite Deum cæli, et coram ómnibus vivéntibus confitémini ei, quia fecit vobíscum misericórdiam suam.
		Etenim sacraméntum regis abscóndere bonum est, ópera autem Dei reveláre et confitéri honoríficum est.
		Bona est orátio cum jejúnio, et eleemósyna magis quam thesáuros auri recóndere;
		Quóniam eleemósyna a morte líberat, et ipsa est quæ purgat peccáta, et facit inveníre misericórdiam et vitam ætérnam.
		Qui autem fáciunt peccátum et iniquitátem, hostes sunt ánimæ suæ.
		Manifésto ergo vobis veritátem et non abscóndam a vobis occúltum sermónem.
		Quando orábas cum lácrimis, et sepeliébas mórtuos, et derelinquébas prándium tuum, et mórtuos abscondébas per diem in domo tua, et nocte sepeliébas eos, ego óbtuli oratiónem tuam Dómino.
		Et quia accéptus eras Deo, necésse fuit ut tentátio probáret te.
	}
	{	\rubrique{Tob 12 : 5-13} TODO
	
	}
	{r2}
	{\rr Tobie, étant sorti, trouva un jeune homme magnifique, debout, ceint, et comme prêt à marcher ; il le salua et dit :\\
	\GreSpecial{*} D’où es-tu, bon jeune homme ?\\
	\vv Et, ignorant que ce fût un Ange du Seigneur, il le salua et dit :\\
	\GreSpecial{*} D’où es-tu, bon jeune homme ?}

\lectioresponsorium
	{Lectio \textsc{iii}}
	{Troisième lecture}
	{Spíritus Sancti grátia illúminet sensus et corda nostra}
	{Que la grâce du Saint-Esprit illumine nos esprits et nos cœurs}
	{
		Et nunc misit me Dóminus ut curárem te, et Saram uxórem fílii tui a dæmónio liberárem;
		Ego enim sum Ráphaël Angelus, unus ex septem qui astámus ante Dóminum.
		Cumque hæc audíssent, turbáti sunt, et treméntes cecidérunt super terram in fáciem suam.
		Dixítque eis Angelus: Pax vobis, nolíte timére.
		Etenim cum essem vobíscum, per voluntátem Dei eram: ipsum benedícite, et cantáte illi.
		Vidébar quidem vobíscum manducáre, et bíbere: sed ego cibo invisíbili, et potu, qui ab homínibus vidéri non potest, utor.
		Tempus est ergo ut revértar ad eum, qui me misit: vos autem benedícite Deum, et narráte ómnia mirabília ejus.
		Et cum hæc dixísset, ab aspéctu eórum ablátus est, et ultra eum vidére non potuérunt.
		Tunc prostráti per horas tres in fáciem, benedixérunt Deum: et exsurgéntes narravérunt ómnia mirabília ejus.
	}
	{	\rubrique{Tob 12 : 14-22} TODO 
	
	}
	{r3}
	{\rr L’Ange étant entré auprès de Tobie, le salua et dit : Que la joie soit avec vous toujours ;\\
	\GreSpecial{*} Ayez bon courage, dans peu de temps vous serez guéri par Dieu.\\
	\vv Et Tobie répondant, dit : Quelle joie aurai-je, moi qui suis toujours dans les ténèbres, et qui ne vois point la lumière du ciel ?\\
	\GreSpecial{*} Ayez bon courage, dans peu de temps vous serez guéri par Dieu.\\
	\versetGloireAuPere{}\\
	\GreSpecial{*} Ayez bon courage, dans peu de temps vous serez guéri par Dieu.}

\subsection{In II Nocturno}

\switchcolumn

\subsection{Deuxième Nocturne}

\switchcolumn*

\paragraph{Psalmus 18}

\gscore{a4_4e}

\begin{enumerate}[wide, itemsep=0mm, labelwidth=!, labelindent=0pt, label=\color{gregoriocolor}\theenumi]
\item Cæli enárrant gló\textit{ri}\textit{am} \textbf{De}i:~* et ópera mánuum ejus annúntiat firma\textbf{mén}tum.
\item Dies diéi e\textit{rúc}\textit{tat} \textbf{ver}bum,~* et nox nocti índicat sci\textbf{én}tiam.
\item Non sunt loquélæ, ne\textit{que} \textit{ser}\textbf{mó}nes,~* quorum non audiántur voces e\textbf{ó}rum.
\item In omnem terram exívit so\textit{nus} \textit{e}\textbf{ó}rum:~* et in fines orbis terræ verba e\textbf{ó}rum.
\item In sole pósuit taberná\textit{cu}\textit{lum} \textbf{su}um:~* et ipse tamquam sponsus procédens de thálamo \textbf{su}o.
\item Exsultávit ut gigas ad cur\textit{rén}\textit{dam} \textbf{vi}am,~* a summo cælo egréssio \textbf{e}jus.
\item Et occúrsus ejus usque ad \textit{sum}\textit{mum} \textbf{e}jus:~* nec est qui se abscóndat a calóre \textbf{e}jus.
\item Lex Dómini immaculáta, con\textit{vér}\textit{tens} \textbf{á}nimas:~* testimónium Dómini fidéle, sapiéntiam præstans \textbf{pár}vulis.
\item Justítiæ Dómini rectæ, lætifi\textit{cán}\textit{tes} \textbf{cor}da:~* præcéptum Dómini lúcidum illúminans \textbf{ó}culos.
\item Timor Dómini sanctus, pérmanens in sǽ\textit{cu}\textit{lum} \textbf{sǽ}culi:~* judícia Dómini vera, justificáta in semet\textbf{íp}sa.
\item Desiderabília super aurum et lápidem preti\textit{ó}\textit{sum} \textbf{mul}tum:~* et dulcióra super mel et \textbf{fa}vum.
\item Etenim servus tuus cus\textit{tó}\textit{dit} \textbf{e}a,~* in custodiéndis illis retribútio \textbf{mul}ta.
\item Delícta quis intélligit?~† ab occúltis \textit{me}\textit{is} \textbf{mun}da me:~* et ab aliénis parce servo \textbf{tu}o.
\item Si mei non fúerint domináti, tunc immacu\textit{lá}\textit{tus} \textbf{e}ro:~* et emundábor a delícto \textbf{má}ximo.
\item Et erunt ut compláceant elóquia \textit{o}\textit{ris} \textbf{me}i:~* et meditátio cordis mei in conspéctu tuo \textbf{sem}per.
\item Dómine, ad\textit{jú}\textit{tor} \textbf{me}us,~* et redémptor \textbf{me}us.
\item Glória Pa\textit{tri}, \textit{et} \textbf{Fí}lio,~* et Spirítui \textbf{Sanc}to.
\item Sicut erat in princípio, et \textit{nunc}, \textit{et} \textbf{sem}per,~* et in sǽcula sæculórum. \textbf{A}men.
\end{enumerate}

\switchcolumn

\paragraph{Psaume 18}
\aa Mais l’Ange dit : prends le poisson par les ouïes et tire-le hors de l’eau.

\gscore{4_la_E}

\begin{enumerate}[wide, itemsep=0mm, labelwidth=!, labelindent=0pt, label=\color{gregoriocolor}\theenumi]


\end{enumerate}

\switchcolumn*

\paragraph{Psalmus 23}

\gscore{a5_5f}

\begin{enumerate}[wide, itemsep=0mm, labelwidth=!, labelindent=0pt, label=\color{gregoriocolor}\theenumi]
\item Dómini est terra, et plenitúdo \textbf{e}jus:~* orbis terrárum, et univérsi qui hábi\textbf{tant} in \textbf{e}o.
\item Quia ipse super mária fundávit \textbf{e}um:~* et super flúmina præpa\textbf{rá}vit \textbf{e}um.
\item Quis ascéndet in montem \textbf{Dó}mini?~* aut quis stabit in loco \textbf{sanc}to \textbf{e}jus?
\item Innocens mánibus et mundo corde,~† qui non accépit in vano ánimam \textbf{su}am,~* nec jurávit in dolo \textbf{pró}ximo \textbf{su}o.
\item Hic accípiet benedictiónem a \textbf{Dó}mino:~* et misericórdiam a Deo, salu\textbf{tá}ri \textbf{su}o.
\item Hæc est generátio quæréntium \textbf{e}um,~* quæréntium fáciem \textbf{De}i \textbf{Ja}cob.
\item Attóllite portas, príncipes, vestras,~† et elevámini, portæ æter\textbf{ná}les:~* et intro\textbf{í}bit Rex \textbf{gló}riæ.
\item Quis est iste Rex glóriæ?~† Dóminus fortis et \textbf{pot}ens:~* Dóminus \textbf{pot}ens in \textbf{prǽ}lio.
\item Attóllite portas, príncipes, vestras,~† et elevámini, portæ æter\textbf{ná}les:~* et intro\textbf{í}bit Rex \textbf{gló}riæ.
\item Quis est iste Rex \textbf{gló}riæ?~* Dóminus virtútum ipse \textbf{est} Rex \textbf{gló}riæ.
\item Glória Patri, et \textbf{Fí}lio,~* et Spi\textbf{rí}tui \textbf{Sanc}to.
\item Sicut erat in princípio, et nunc, et \textbf{sem}per,~* et in sǽcula sæcu\textbf{ló}rum. \textbf{A}men.
\end{enumerate}

\switchcolumn

\paragraph{Psaume 23}
\aa Je te prie, Azarias mon frère, de me dire à quel remède serviront les parties du poisson que tu m’as fait conserver.

\gscore{5}

\begin{enumerate}[wide, itemsep=0mm, labelwidth=!, labelindent=0pt, label=\color{gregoriocolor}\theenumi]


\end{enumerate}

\switchcolumn*

\paragraph{Psalmus 33}

\gscore{a6_6f}

\begin{enumerate}[wide, itemsep=0mm, labelwidth=!, labelindent=0pt, label=\color{gregoriocolor}\theenumi]
\item Benedícam Dóminum in om\textit{ni} \textbf{tém}pore:~* semper laus ejus in \textit{o}\textit{re} \textbf{me}o.
\item In Dómino laudábitur áni\textit{ma} \textbf{me}a:~* áudiant mansuéti, \textit{et} \textit{læ}\textbf{tén}tur.
\item Magnificáte Dómi\textit{num} \textbf{me}cum:~* et exaltémus nomen ejus \textit{in} \textit{id}\textbf{íp}sum.
\item Exquisívi Dóminum, et ex\textit{au}\textbf{dí}vit me:~* et ex ómnibus tribulatiónibus meis e\textit{rí}\textit{pu}\textbf{it} me.
\item Accédite ad eum, et illu\textit{mi}\textbf{ná}mini:~* et fácies vestræ non \textit{con}\textit{fun}\textbf{dén}tur.
\item Iste pauper clamávit, et Dóminus exaudí\textit{vit} \textbf{e}um:~* et de ómnibus tribulatiónibus ejus sal\textit{vá}\textit{vit} \textbf{e}um.
\item Immíttet Angelus Dómini in circúitu timénti\textit{um} \textbf{e}um:~* et erí\textit{pi}\textit{et} \textbf{e}os.
\item Gustáte, et vidéte quóniam suávis \textit{est} \textbf{Dó}minus:~* beátus vir, qui spe\textit{rat} \textit{in} \textbf{e}o.
\item Timéte Dóminum, omnes sanc\textit{ti} \textbf{e}jus:~* quóniam non est inópia timén\textit{ti}\textit{bus} \textbf{e}um.
\item Dívites eguérunt et esu\textit{ri}\textbf{é}runt:~* inquiréntes autem Dóminum non minuéntur \textit{om}\textit{ni} \textbf{bo}no.
\item Veníte, fílii, \textit{au}\textbf{dí}te me:~* timórem Dómi\textit{ni} \textit{do}\textbf{cé}bo vos.
\item Quis est homo qui \textit{vult} \textbf{vi}tam:~* díligit dies vi\textit{dé}\textit{re} \textbf{bo}nos?
\item Próhibe linguam tuam \textit{a} \textbf{ma}lo:~* et lábia tua ne lo\textit{quán}\textit{tur} \textbf{do}lum.
\item Divérte a malo, et \textit{fac} \textbf{bo}num:~* inquíre pacem, et persé\textit{que}\textit{re} \textbf{e}am.
\item Oculi Dómini su\textit{per} \textbf{jus}tos:~* et aures ejus in pre\textit{ces} \textit{e}\textbf{ó}rum.
\item Vultus autem Dómini super facién\textit{tes} \textbf{ma}la:~* ut perdat de terra memóri\textit{am} \textit{e}\textbf{ó}rum.
\item Clamavérunt justi, et Dóminus exaudí\textit{vit} \textbf{e}os:~* et ex ómnibus tribulatiónibus eórum libe\textit{rá}\textit{vit} \textbf{e}os.
\item Juxta est Dóminus iis, qui tribuláto \textit{sunt} \textbf{cor}de:~* et húmiles spíri\textit{tu} \textit{sal}\textbf{vá}bit.
\item Multæ tribulatiónes \textit{jus}\textbf{tó}rum:~* et de ómnibus his liberábit \textit{e}\textit{os} \textbf{Dó}minus.
\item Custódit Dóminus ómnia ossa \textit{e}\textbf{ó}rum:~* unum ex his non \textit{con}\textit{te}\textbf{ré}tur.
\item Mors peccató\textit{rum} \textbf{pés}sima:~* et qui odérunt jus\textit{tum}, \textit{de}\textbf{lín}quent.
\item Rédimet Dóminus ánimas servórum \textit{su}\textbf{ó}rum:~* et non delínquent omnes qui spe\textit{rant} \textit{in} \textbf{e}o.
\item Glória Patri, \textit{et} \textbf{Fí}lio,~* et Spirí\textit{tu}\textit{i} \textbf{Sanc}to.
\item Sicut erat in princípio, et nunc, \textit{et} \textbf{sem}per,~* et in sǽcula sæcu\textit{ló}\textit{rum}. \textbf{A}men.
\end{enumerate}

\switchcolumn

\paragraph{Psaume 33}
\aa Ce sont les yeux que le fiel guérit ; quant au cœur et au foie, ils ont la vertu de chasser l’emprise du diable.

\gscore{6_lib}

\begin{enumerate}[wide, itemsep=0mm, labelwidth=!, labelindent=0pt, label=\color{gregoriocolor}\theenumi]


\end{enumerate}

\switchcolumn*

\versiculusabsolutio
	{Ascéndit fumus arómatum in conspéctu Dómini}
	{De manu Angeli}
	{Ipsíus píetas et misericórdia nos ádjuvet, qui cum Patre et Spíritu Sancto vivit et regnat in sǽcula sæculórum}
	{La fumée des parfums monta en présence du Seigneur}
	{De la main de l’ange}
	{Qu'il nous secoure par sa bonté et sa miséricorde, celui qui, avec le Père et le Saint-Esprit, vit et règne dans les siècles des siècles}

\lectioresponsorium
	{Lectio \textsc{iv}}
	{Quatrième lecture}
	{Deus Pater omnípotens sit nobis propítius et clemens}
	{Que Dieu le Père tout-puissant soit pour nous propice et plein de clémence}
	{
		Sermo Sancti Bonaventuræ Epíscopi
		
		Ráphaël interpretátur medicina Dei. Et debémus notare quod eductio a malo est per tria beneficia, a Raphaéle nobis colláta medicante nos. Educit ergo nos Ráphaël médicus ab infirmitáte animi, inducéndo nos ad amaritúdinem contritiónis; unde in Tobía dicit Ráphaël: Ubi introíeris domum tuam, lini super óculos ejus ex felle. Sic fecit, et vidit. Quare Ráphaël non pótuit ipse fácere? Quia Angelus non dat compunctiónem, sed osténdit viam. Per fel intellígitur amaritúdo contritiónis, quæ sanat óculos interioris mentis; Psalmus: Qui sanat contritos corde. Hoc est optimum collyrium. In Júdicum secundo dícitur quod Angelus ascéndit ad locum fléntium, et dixit pópulo: Eduxi vos de terra Ægypti, feci vobis tot et tanta bona; et flevit omnis pópulus, ita ut locus ille appellarétur locus fléntium. Caríssimi, Angeli tota die narrant nobis benefícia Dei, et reducunt ea nobis ad memóriam: Quis est qui te creávit, qui te redémit? Quid fecísti, quem offendísti? Hoc si consideráveris, nullum habes remédium nisi flere.
		
	}
	{
		\rubrique{Cinquième sermon sur les Saint Anages}
		
		Sermon de saint Bonaventure, Évêque.
		
		Le nom de Raphaël veut dire médecine de Dieu. Et nous devons remarquer qu’on peut être retiré du mal par trois bienfaits que saint Raphaël nous accorde quand il nous guérit. D’abord Raphaël, le médecin céleste, nous arrache à l’infirmité spirituelle en nous amenant à l’amertume salutaire de la contrition, à laquelle se rapporte ce que Raphaël dit à Tobie : Dès que tu seras entré dans ta maison, oins ses yeux avec du fiel. Il le fit et son père recouvra la vue. Pourquoi ne dut-ce point être Raphaël lui-même qui fit cette onction ? Parce qu’un Ange ne donne point la componction ; son rôle est d’en montrer la voie. En ce fiel nous voyons donc l’image de l’amertume de la contrition, laquelle rend sains les yeux intérieurs de l’âme ; un psaume nous dit : « Il guérit ceux qui sont contrits de cœur. » Cette contrition est un collyre excellent. Au deuxième chapitre du livre des Juges, il est raconté que l’Ange monta auprès de ceux qui versaient des larmes et dit au peuple : « Je vous ai retirés de la terre d’Egypte ; j’ai accompli pour vous tant et tant de choses bonnes, et tout 1e peuple pleura de telle sorte que ce lieu fut appelé le lieu de ceux qui pleurent. »Mes très chers, les Anges nous parlent tout le long du jour des bienfaits de Dieu et nous les remettent en mémoire : Ils semblent nous dire : Qui t’a créé ? Qui t’a racheté ? Qu’as tu fait ? Qu’as-tu offensé ? Or, si nous nous arrêtons à considérer ce qui en est, nous ne trouverons d’autre remède que de pleurer.
	}
	{r4}
	{\rr Tobie demande à l’Ange : De quelle maison, ou de quelle tribu es-tu ? L’Ange répondant, dit :\\
	\GreSpecial{*} Je suis Azarias, fils du grand Ananias.\\
	\vv Est-ce la race du mercenaire que vous cherchez, ou bien le mercenaire lui-même qui doit conduire votre fils. Mais, de peur que je ne vous rende inquiet :
	\rr Je suis Azarias, fils du grand Ananias.}

\lectioresponsorium
	{Lectio \textsc{v}}
	{Cinquième lecture}
	{Christus perpétuæ det nobis gáudia vitæ}
	{Que le Christ nous donne les joies de l'éternelle vie}
	{
		Secundo Ráphaël edúcit de servitute diaboli, persuadéndo nobis memóriam passiónis Christi; in cujus figuram dictum est Tobíæ sexto: Cordis ejus particulam si super carbónes ponas, fumus ejus éxtricat omne genus dæmoniórum. Dícitur Tobíæ octavo quod pósuit Tobías particulam cordis super carbónes, et Ráphaël religávit dæmónium in desérto superioris Ægypti. Quid est hoc? Non poterat Ráphaël religare dæmónium nisi ponerétur cor super carbónes? Numquid cor piscis dabat Angelo tantam virtútem? Nequaquam! Nihil posset, nisi ibi mystérium esset. In hoc enim nobis datur intelligi quod nihil est quod ita nos líberet hódie a servitute diaboli sicut passio Christi, quæ processit ex radíce cordis sive caritátis. Cor enim fons est caloris cunctæ vitæ. Si ergo Cor Christi, hoc est passiónem quam sustinuit, procedéntem ex radíce caritátis et fonte caloris, ponas super carbónes, hoc est super inflammatam memóriam; statim dæmon religábitur, ut tibi nocére non possit.
	}
	{
		Secondement, saint Raphaël nous arrache à la servitude du diable, quand il fait pénétrer en nous le souvenir de la passion du Christ en figure de laquelle il est dit au sixième chapitre de Tobie : Si tu mets une parcelle de son cœur sur des charbons ardents, la fumée qui s’en dégagera mettra en fuite la race des démons. En effet, Raphaël relégua le démon dans un désert de la haute Egypte. Qu’est ceci ? Raphaël n’aurait pu éloigner le démon s’il n’avait mis le cœur sur des charbons ardents ? Est-ce le cœur d’un poisson qui donnait à l’Ange tant de pouvoir ? Nullement. Il serait demeuré sans aucune vertu s’il n’y avait eu ici un mystère. Par ce fait il nous est donné à entendre que rien aujourd’hui ne nous délivre de la servitude du diable comme la passion du Christ, et que cette passion procède de son cœur comme d’une racine, c’est-à-dire qu’elle est le fruit de son amour. Le cœur est en effet la source de toute notre chaleur vitale. Si donc tu mets le Cœur du Christ, c’est-à-dire la passion qu’il a soufferte et dont la racine est la charité, la source son ardeur, si tu mets ce Cœur divin sur des charbons en le rappelant à ta mémoire et que ton âme s’enflamme, aussitôt le démon sera éloigné, de sorte qu’il ne pourra te nuire.
	}
	{r5}
	{\rr Tobie sortit pour laver ses pieds, et voilà qu’un poisson énorme sortit pour le dévorer ; épouvanté, il cria d’une voix forte, disant : Seigneur, il s’élance sur moi ! Et l’ange lui dit : Saisis-le par les ouïes, et tire-le à toi :\\
	\GreSpecial{*} Éventre ce poisson, et réserve t’en le cœur, le fiel et le foie, parce que ces choses sont nécessaires pour des remèdes utiles.\\
	\vv Tobie tira le poisson à terre, et le poisson commença à palpiter à ses pieds. Alors l’Ange lui dit :\\
	\rr Éventre ce poisson, et réserve t’en le cœur, le fiel et le foie, parce que ces choses sont nécessaires pour des remèdes utiles.}
	
\lectioresponsorium
	{Lectio \textsc{vi}}
	{Sixième lecture}
	{Ignem sui amóris accéndat Deus in córdibus nostris}
	{Que Dieu daigne allumer dans nos cœurs le feu de son amour}
	{
		Tertio liberat nos a contrarietáte Dei, quam incurrimus per offensam Dei, et hoc inducéndo nos ad instantiam oratiónis; et hoc est quod dixit angelus Ráphaël Tobíæ duodecimo: Quando orabas cum lácrimis, ego óbtuli oratiónem tuam Dómino. Ipsi enim Angeli reconciliant nos Deo, quantum possunt. Accusatores nostri coram Deo sunt dæmones. Angeli autem excusant nos, quando ófferunt oratiónes nostras, ad quas devote faciendas nos inducunt; Apocalypsis octavo: Ascéndit fumus arómatum in conspéctu Dómini de manu Angeli. Arómata ista suáviter redoléntia sunt oratiónes Sanctórum. Vis placare Deum, quem offendísti? Ora devote. Offérunt Deo oratiónem tuam, ut te Deo reconcilient. Dícitur in Luca quod Christus, factus in agonía, prolixius orabat, et appáruit Angelus Dómini confortans eum. Et hoc totum factum est propter nos, quia non indiguit confortatióne sua, sed ut ostenderétur quod libenter assistunt devote orántibus, et libenter juvant eos et ipsos confortant, et oratiónes eórum Deo ófferunt. - Festum sancti Ráphaëlis Archangeli Benedíctus Papa décimus quintus ad universam Ecclésiam extendit.
	}
	{
		Troisièmement l’Archange Raphaël nous délivre de la peine de nous trouver en opposition avec Dieu, peine que nous encourons en offensant ce Dieu ; il nous en délivre quand il nous amène à prier avec instance ; et à ceci je rapporte ce que l’Ange Raphaël dit à Tobie au douzième chapitre : Quand tu priais avec larmes, moi j’ai offert ton oraison au Seigneur. Les Anges nous réconcilient avec Dieu, dans la mesure où ils le peuvent. Nos accusateurs devant Dieu, ce sont les démons. Quant aux Anges, ils nous excusent, lorsqu’ils offrent nos prières, ces prières qu’ils nous ont porté à faire dévotement. On lit au huitième chapitre de l’Apocalypse : « La fumée des parfums s’éleva de la main de l’Ange en présence du Seigneur ». Ces parfums se consumant suavement sont les prières des Saints. Veux-tu plaire au Dieu que tu as offensé ? Prie dévotement. Ils offrent à Dieu ta prière pour te réconcilier avec lui. Il est dit en saint Luc que le Christ étant tombé en agonie priait plus instamment et qu’un Ange de Dieu lui apparut le fortifiant. Tout cela s’est accompli en notre faveur, car le Sauveur n’avait point besoin d’être fortifié par un messager céleste ; mais il en a été ainsi pour montrer que les Anges assistent volontiers ceux qui prient avec piété et volontiers les aident ; ils les fortifient et offrent leurs oraisons à Dieu. Le Pape Benoît XV a étendu à l’Église universelle la fête de saint Raphaël, archange.
	}
	{r6}
	{\rr Dès que tu seras entré dans ta maison, dit l’Ange Raphaël à Tobie, aussitôt adore le Seigneur ton Dieu, et, lui rendant grâces, approche de ton père, et embrasse-le :\\
	\GreSpecial{*} Et aussitôt, frotte ses yeux avec ce fiel de poisson que tu portes avec toi ; car sache qu’à l’instant les yeux de ton père s’ouvriront, et que ton père verra la lumière du ciel, et qu’à ton aspect il se réjouira.\\
	\vv Prends avec toi de ce fiel du poisson, car il te sera nécessaire.\\
	\GreSpecial{*} Et aussitôt, frotte ses yeux avec ce fiel de poisson que tu portes avec toi ; car sache qu’à l’instant les yeux de ton père s’ouvriront, et que ton père verra la lumière du ciel, et qu’à ton aspect il se réjouira.\\
	\versetGloireAuPere{}\\
	\GreSpecial{*} Et aussitôt, frotte ses yeux avec ce fiel de poisson que tu portes avec toi ; car sache qu’à l’instant les yeux de ton père s’ouvriront, et que ton père verra la lumière du ciel, et qu’à ton aspect il se réjouira.}

\subsection{In III Nocturno}

\switchcolumn

\subsection{Troisième Nocturne}

\switchcolumn*

\paragraph{Psalmus 95}

\gscore{a7_7a}

\begin{enumerate}[wide, itemsep=0mm, labelwidth=!, labelindent=0pt, label=\color{gregoriocolor}\theenumi]
\item Cantáte Dómino \textbf{cán}ticum \textbf{no}vum:~* cantáte Dómino, \textbf{om}nis \textbf{ter}ra.
\item Cantáte Dómino, et benedícite \textbf{nó}mini \textbf{e}jus:~* annuntiáte de die in diem salu\textbf{tá}re \textbf{e}jus.
\item Annuntiáte inter gentes \textbf{gló}riam \textbf{e}jus,~* in ómnibus pópulis mira\textbf{bí}lia \textbf{e}jus.
\item Quóniam magnus Dóminus, et lau\textbf{dá}bilis \textbf{ni}mis:~* terríbilis est super \textbf{om}nes \textbf{de}os.
\item Quóniam omnes dii Génti\textbf{um} dæ\textbf{mó}nia:~* Dóminus autem \textbf{cæ}los \textbf{fe}cit.
\item Conféssio, et pulchritúdo in con\textbf{spéc}tu \textbf{e}jus:~* sanctimónia et magnificéntia in sanctificati\textbf{ó}ne \textbf{e}jus.
\item Afférte Dómino, pátriæ Géntium,~† afférte Dómino glóriam \textbf{et} ho\textbf{nó}rem:~* afférte Dómino glóriam \textbf{nó}mini \textbf{e}jus.
\item Tóllite hóstias, et introíte in \textbf{á}tria \textbf{e}jus:~* adoráte Dóminum in átrio \textbf{sanc}to \textbf{e}jus.
\item Commoveátur a fácie ejus uni\textbf{vér}sa \textbf{ter}ra:~* dícite in Géntibus quia Dómi\textbf{nus} re\textbf{gná}vit.
\item Etenim corréxit orbem terræ qui non \textbf{com}mo\textbf{vé}bitur:~* judicábit pópulos in \textbf{æ}qui\textbf{tá}te.
\item Læténtur cæli, et exsúltet terra:~† commoveátur mare et pleni\textbf{tú}do \textbf{e}jus:~* gaudébunt campi, et ómnia \textbf{quæ} in \textbf{e}is sunt.
\item Tunc exsultábunt ómnia ligna silvárum a fácie Dómini, \textbf{qui}a \textbf{ve}nit:~* quóniam venit judi\textbf{cá}re \textbf{ter}ram.
\item Judicábit orbem terræ in \textbf{æ}qui\textbf{tá}te,~* et pópulos in veri\textbf{tá}te \textbf{su}a.
\item Glória \textbf{Pa}tri, et \textbf{Fí}lio,~* et Spi\textbf{rí}tui \textbf{Sanc}to.
\item Sicut erat in princípio, et \textbf{nunc}, et \textbf{sem}per,~* et in sǽcula sæcu\textbf{ló}rum. \textbf{A}men.
\end{enumerate}

\switchcolumn

\paragraph{Psaume 95}
\aa Voici Sara la fille de Raguël ; celle qui te sera donnée en mariage, et tout son bien avec elle.

\gscore{7_a}

\begin{enumerate}[wide, itemsep=0mm, labelwidth=!, labelindent=0pt, label=\color{gregoriocolor}\theenumi]


\end{enumerate}

\switchcolumn*

\paragraph{Psalmus 96}

\gscore{a8_8g}

\begin{enumerate}[wide, itemsep=0mm, labelwidth=!, labelindent=0pt, label=\color{gregoriocolor}\theenumi]
\item Dóminus regnávit exsúltet \textbf{ter}ra:~* læténtur ín\textit{su}\textit{læ} \textbf{mul}tæ.
\item Nubes, et calígo in circúitu \textbf{e}jus:~* justítia, et judícium corréctio \textit{se}\textit{dis} \textbf{e}jus.
\item Ignis ante ipsum præ\textbf{cé}det:~* et inflammábit in circúitu ini\textit{mí}\textit{cos} \textbf{e}jus.
\item Illuxérunt fúlgura ejus orbi \textbf{ter}ræ:~* vidit et commó\textit{ta} \textit{est} \textbf{ter}ra.
\item Montes, sicut cera fluxérunt a fácie \textbf{Dó}mini:~* a fácie Dómini \textit{om}\textit{nis} \textbf{ter}ra.
\item Annuntiavérunt cæli justítiam \textbf{e}jus:~* et vidérunt omnes pópuli gló\textit{ri}\textit{am} \textbf{e}jus.
\item Confundántur omnes, qui adórant sculp\textbf{tí}lia:~* et qui gloriántur in simu\textit{lá}\textit{cris} \textbf{su}is.
\item Adoráte eum, omnes Angeli \textbf{e}jus:~* audívit, et lætá\textit{ta} \textit{est} \textbf{Si}on.
\item Et exsultavérunt fíliæ \textbf{Ju}dæ:~* propter judícia \textit{tu}\textit{a}, \textbf{Dó}mine:
\item Quóniam tu Dóminus Altíssimus super omnem \textbf{ter}ram:~* nimis exaltátus es super \textit{om}\textit{nes} \textbf{de}os.
\item Qui dilígitis Dóminum, odíte \textbf{ma}lum:~* custódit Dóminus ánimas sanctórum suórum, de manu peccatóris libe\textit{rá}\textit{bit} \textbf{e}os.
\item Lux orta est \textbf{jus}to,~* et rectis cor\textit{de} \textit{læ}\textbf{tí}tia.
\item Lætámini, justi in \textbf{Dó}mino:~* et confitémini memóriæ sanctificati\textit{ó}\textit{nis} \textbf{e}jus.
\item Glória Patri, et \textbf{Fí}lio,~* et Spirí\textit{tu}\textit{i} \textbf{Sanc}to.
\item Sicut erat in princípio, et nunc, et \textbf{sem}per,~* et in sǽcula sæcu\textit{ló}\textit{rum}. \textbf{A}men.
\end{enumerate}

\switchcolumn

\paragraph{Psaume 96}
\aa Elle a eu sept maris que le démon a étouffés ; je crains que même chose ne m’arrive.

\gscore{8_G}

\begin{enumerate}[wide, itemsep=0mm, labelwidth=!, labelindent=0pt, label=\color{gregoriocolor}\theenumi]

\end{enumerate}

\switchcolumn*

\paragraph{Psalmus 102}

\gscore{a9_1d-}

\begin{enumerate}[wide, itemsep=0mm, labelwidth=!, labelindent=0pt, label=\color{gregoriocolor}\theenumi]
\item Bénedic, ánima \textbf{me}a, \textbf{Dó}mino:~* et ómnia, quæ intra me sunt, nómini \textit{sanc}\textit{to} \textbf{e}jus.
\item Bénedic, ánima \textbf{me}a, \textbf{Dó}mino:~* et noli oblivísci omnes retributi\textit{ó}\textit{nes} \textbf{e}jus.
\item Qui propitiátur ómnibus iniqui\textbf{tá}tibus \textbf{tu}is:~* qui sanat omnes infirmi\textit{tá}\textit{tes} \textbf{tu}as.
\item Qui rédimit de intéritu \textbf{vi}tam \textbf{tu}am:~* qui corónat te in misericórdia et mise\textit{ra}\textit{ti}\textbf{ó}nibus.
\item Qui replet in bonis desi\textbf{dé}rium \textbf{tu}um:~* renovábitur ut áquilæ ju\textit{vén}\textit{tus} \textbf{tu}a.
\item Fáciens miseri\textbf{cór}dias \textbf{Dó}minus:~* et judícium ómnibus injúriam \textit{pa}\textit{ti}\textbf{én}tibus.
\item Notas fecit vias \textbf{su}as \textbf{Mó}ysi,~* fíliis Israël volun\textit{tá}\textit{tes} \textbf{su}as.
\item Miserátor, et mi\textbf{sé}ricors \textbf{Dó}minus:~* longánimis et mul\textit{tum} \textit{mi}\textbf{sé}ricors.
\item Non in perpétuum \textbf{i}ra\textbf{scé}tur:~* neque in ætérnum \textit{com}\textit{mi}\textbf{ná}bitur.
\item Non secúndum peccáta nostra \textbf{fe}cit \textbf{no}bis:~* neque secúndum iniquitátes nostras retrí\textit{bu}\textit{it} \textbf{no}bis.
\item Quóniam secúndum altitúdinem \textbf{cæ}li a \textbf{ter}ra:~* corroborávit misericórdiam suam su\textit{per} \textit{ti}\textbf{mén}tes se.
\item Quantum distat ortus ab \textbf{oc}ci\textbf{dén}te:~* longe fecit a nobis iniqui\textit{tá}\textit{tes} \textbf{nos}tras.
\item Quómodo miserétur pater filiórum,~† misértus est Dóminus ti\textbf{mén}ti\textbf{bus} se:~* quóniam ipse cognóvit fig\textit{mén}\textit{tum} \textbf{nos}trum.
\item Recordátus est quóniam pulvis sumus:~† homo, sicut fœnum \textbf{di}es \textbf{e}jus,~* tamquam flos agri sic \textit{ef}\textit{flo}\textbf{ré}bit.
\item Quóniam spíritus pertransíbit in illo, et \textbf{non} sub\textbf{sís}tet:~* et non cognóscet ámplius \textit{lo}\textit{cum} \textbf{su}um.
\item Misericórdia autem Dómini \textbf{ab} æ\textbf{tér}no,~* et usque in ætérnum super ti\textit{mén}\textit{tes} \textbf{e}um.
\item Et justítia illíus in fílios \textbf{fi}li\textbf{ó}rum,~* his qui servant testa\textit{mén}\textit{tum} \textbf{e}jus.
\item Et mémores sunt manda\textbf{tó}rum ip\textbf{sí}us,~* ad faci\textit{én}\textit{dum} \textbf{e}a.
\item Dóminus in cælo parávit \textbf{se}dem \textbf{su}am:~* et regnum ipsíus ómnibus \textit{do}\textit{mi}\textbf{ná}bitur.
\item Benedícite Dómino, omnes Angeli ejus:~† poténtes virtúte, faciéntes \textbf{ver}bum il\textbf{lí}us,~* ad audiéndam vocem ser\textit{mó}\textit{num} \textbf{e}jus.
\item Benedícite Dómino, omnes vir\textbf{tú}tes \textbf{e}jus:~* minístri ejus, qui fácitis volun\textit{tá}\textit{tem} \textbf{e}jus.
\item Benedícite Dómino, ómnia ópera ejus:~† in omni loco dominati\textbf{ó}nis \textbf{e}jus,~* bénedic, ánima \textit{me}\textit{a}, \textbf{Dó}mino.
\item Glória \textbf{Pa}tri, et \textbf{Fí}lio,~* et Spirí\textit{tu}\textit{i} \textbf{Sanc}to.
\item Sicut erat in princípio, et \textbf{nunc}, et \textbf{sem}per,~* et in sǽcula sæcu\textit{ló}\textit{rum}. \textbf{A}men.
\end{enumerate}

\switchcolumn

\paragraph{Psaume 102}
\aa Pendant trois jours, tu vaqueras à la prière avec ton épouse, pour qu’en la postérité d’Abraham, tu obtiennes des fils en bénédiction.

\gscore{1_D-}

\begin{enumerate}[wide, itemsep=0mm, labelwidth=!, labelindent=0pt, label=\color{gregoriocolor}\theenumi]


\end{enumerate}

\switchcolumn*

\versiculusabsolutio
	{Apprehéndit Angelus Ráphaël dæmónium}
	{Et religávit illud in desérto superióris Ægýpti}
	{A vínculis peccatórum nostrórum absólvat nos omnípotens et miséricors Dóminus}
	{L’Ange Raphaël saisit le démon}
	{Et il le lia dans le désert de la haute Ëgypte}
	{Que le Dieu tout-puissant et miséricordieux daigne nous délivrer des liens de nos péchés}

\lectioresponsorium
	{Lectio \textsc{vii}}
	{Septième lecture}
	{Evangélica léctio sit nobis salus et protéctio}
	{Que la lecture du saint Evangile nous soit salut et protection}
	{
		Léctio sancti Evangélii secúndum Joánnem
		
		In illo témpore: Erat dies festus Judæórum, et ascéndit Jesus Jerosolymam. Et réliqua.
		
		Homilía sancti Joánnis Chrysóstomi

		Quis hic curatiónis modus? quale mystérium subindicátur? Neque enim sine causa hæc scripta sunt; sed futura nobis quasi in figura et imagine describit, ne, si res stupenda accideret inexspéctata, auditórum multórum fidem aliquátenus labefactáret. Quænam ígitur hæc descriptio? Futurum baptisma dandum erat, plenum virtúte et grátia maxima, baptisma, quod peccáta ómnia ablúeret, quod ex mórtuis vivos redderet. Hæc ergo ut in imagine depingúntur in piscina et in aliis multis. Et primo quidem aquam dedit, quæ córporum máculas ablúeret, sordesque non veras, sed tales existimátas, ex funere nempe, ex lepra, et similes; multaque vidére est eádem de causa in veteri lege per aquam mundata.
	}
	{
		Joannes 5:1-4
		
		Homélie de saint Jean Chrysostome, Évêque.
		
		\rubrique{Homilia 36, alias 35, in Joánnem, num. 1}
		
		En ce passage de l’Évangile (rapportant) le miracle opéré en faveur du paralytique qui attendait le passage de l’ange près de la piscine appelée en hébreu Bethsaïde, à quel mode de guérison est-il fait allusion ? Quel mystère nous semble indiqué sous l’écorce du fait historique ? Les détails de ce fait n’ont pas été consignés sans motif, mais saint Jean nous annonce comme par une figure et une image ce qui devait s’accomplir dans la suite, de peur que si (la prédication) d’une chose aussi surprenante (qu’un baptême régénérateur) arrivait sans être aucunement attendue, la foi de beaucoup d’auditeurs ne fût jusqu’à un certain point ébranlée. Que signifie donc cette narration ? Elle prédit le baptême qui devait être conféré plus tard, plein de vertu et d’une grâce immense ; le baptême qui allait laver tous les péchés et rendre des morts à la vie. C’est donc le baptême qui est figuré par la piscine et plusieurs autres symboles. Parmi ceux-ci, le Seigneur a d’abord donné l’eau qui lave les taches corporelles et purifie les souillures, non réelles mais réputées telles provenant de funérailles, de lèpre et d’autres causes. Sous l’ancienne loi, il fallait en bien des circonstances, se purifier par l’eau.
	}
	{r7}
	{\rr Bénissez le Dieu du ciel, dit l’Ange Raphaël, et rendez-lui gloire devant tous les vivants :\\
	\GreSpecial{*} Parce qu’il a exercé envers vous sa miséricorde.\\
	\vv Bénissez-le et chantez-le, et racontez toutes ses merveilles.
	\GreSpecial{*} Parce qu’il a exercé envers vous sa miséricorde.}
	
\lectioresponsorium
	{Lectio \textsc{viii}}
	{Huitième lecture}
	{Cujus festum cólimus, ipse intercédat pro nobis ad Dóminum}
	{Que celui dont nous célébrons la fête intercède pour nous auprès du Seigneur}
	{
		Sed ad propositum jam redeamus. Primo ítaque, ut diximus, córporum maculas, deínde varias infirmitátes per aquam solvi curat. Ut enim nos Deus ad baptismi grátiam propius reduceret, non jam maculas solum, sed et morbos sanat. Imagines enim quæ propius ad veritátem accedunt, et in baptismate, et in passióne, et in aliis magis conspicuæ sunt quam vetustiores. Quemádmodum enim qui prope regem sunt satéllites, remotióribus sunt honoratiores; ita et in figuris factum est. Et Angelus descéndens turbábat aquam, et snándi vim indebat ipsi, ut discerent Judæi, Angelórum Dóminum multo magis posse ánimæ morbos omnes curare. Sed, quemádmodum hic aquárum natúra non simpliciter curábat (alioquin enim semper id fáceret), sed Angeli operatióne id fiebat; sic in nobis non aqua simpliciter operátur, sed, postquam Spíritus grátiam accéperit, tunc ómnia solvit peccáta.
	}
	{
		Mais poursuivons notre sujet. La Providence a donc voulu que l’eau servit en premier lieu, à purifier les souillures matérielles, puis à guérir diverses infirmités. Pour nous rapprocher davantage de la grâce du baptême, Dieu ne se contente plus de porter remède aux souillures, mais il guérit aussi les maladies. Soit à propos du baptême, soit à propos de la passion ou de tout autre sujet, les images qui touchent de plus près à la vérité sont plus claires que celles données plus anciennement. Il en est des figures comme des gardes de l’empereur ; les plus rapprochés de sa personne sont toujours plus élevés en dignité que les autres. L’Ange descendait dans la piscine Probatique, en agitait l’eau pour lui communiquer une vertu curative : ce qui préparait les Juifs à reconnaître à plus fortes raisons, au Seigneur des Anges, le pouvoir de guérir tous les maux de l’âme. Toutefois, de même que les eaux de la piscine ne guérissaient point par elles-mêmes, (autrement elles l’eussent toujours fait, ) mais en vertu de l’action de l’Ange, de même l’eau dans le baptême n’agit pas non plus par elle-même ; elle n’efface tous nos péchés que lorsqu’elle a reçu la grâce de l’Esprit.
	}
	{r8}
	{\rr Il est temps que je retourne vers celui qui m’a envoyé, dit l’Ange Raphaël :\\
	\GreSpecial{*} Mais vous, bénissez le Seigneur, et racontez toutes ses merveilles.\\
	\vv Rendez-lui gloire devant tous les vivants, parce qu’il a exercé envers vous sa miséricorde.\\
	\GreSpecial{*} Mais vous, bénissez le Seigneur, et racontez toutes ses merveilles.\\
	\versetGloireAuPere{}\\
	\GreSpecial{*} Mais vous, bénissez le Seigneur, et racontez toutes ses merveilles.}
	
\lectioresponsorium
	{Lectio \textsc{ix}}
	{Neuvième lecture}
	{Ad societátem cívium supernórum perdúcat nos Rex Angelórum}
	{Que le Roi des Anges nous fasse parvenir à la société des citoyens célestes}
	{
		Circa hanc piscinam jacebat multitúdo magna infirmórum, cæcórum, claudórum, aridórum, aquæ motum exspectántium. Sed tunc infirmitas impediménto erat quóminus is qui vellet, sanarétur; nunc autem unusquísque potestátem accedéndi habet. Non enim Angelus est qui aquam movet, sed Angelórum Dóminus ómnia éfficit. Nec dicere póssumus: Dum ego accedo, alius ante me descéndit. Sed, si totus orbis venerit, grátia non consúmitur, neque vis vel operátio déficit, sed semper éadem manet. Ac, quemádmodum solares radii quotídie illúminant, nec absumúntur, neque, quod multis subministréntur, lucis quídpiam amittunt; sic, immo multo minus, Spíritus operátio minúitur a multitúdine accipiéntium. Hoc autem factum est ut qui díscerent in aqua curándos esse corporis morbos, et hac in re diu exercitáti essent, facílius crederent étiam morbos animi posse curari.
	}
	{
		Autour de cette piscine «~gisait une grande multitude de malades, d’aveugles, de boiteux, de paralytiques attendant que l’eau fût mise en mouvement.~» Alors l’infirmité même de chacun d’eux mettait souvent obstacle à sa guérison bien qu’il la voulut. Aujourd’hui il dépend de chacun d’avoir accès à la piscine spirituelle. Ce n’est plus l’ange du Seigneur qui agite les eaux, c’est le Seigneur des Anges qui seul intervient. Nous n’avons plus le droit de dire : Tandis que je m’avance, un autre descend avant moi. Car l’univers entier se présentât-il, la grâce n’en serait pas pour cela épuisée, l’action divine n’en aurait pas moins toute son efficacité et n’en demeurerait pas moins toujours la même. Quoique les rayons du soleil nous éclairent chaque jour, ils ne se raréfient point, quoiqu’ils réjouissent bien des regards, ils ne perdent point leur splendeur ; ainsi (ou plutôt encore moins) l’action de l’Esprit-Saint n’est pas diminuée par le grand nombre de ceux en qui elle s’exerce. Ce qui arrivait à Bethsaïde avait pour but de préparer ceux qui auraient connaissance de cette vertu de l’eau pour guérir les maladies corporelles et qui seraient familiarisés avec ce spectacle, à croire sans peine que les maux de l’âme sont, eux aussi, susceptibles de guérison.
	}
	{}
	{}
	
\paragraph{Te Deum}
	
\gscore{Te_Deum_simplex}

\switchcolumn

\paragraph{Te Deum}

À toi, Dieu, notre louange ! Nous t’acclamons, tu es Seigneur !\\
À toi, Père éternel, l’hymne de l’univers.\\
Devant toi se prosternent les archanges, \\
Les anges et les esprits des cieux ;\\
Ils te rendent grâce ; ils adorent et ils chantent :\\
Saint, Saint, Saint, le Seigneur, Dieu de l’univers ;\\
Le ciel et la terre sont remplis de la majesté de ta gloire.\\
C’est toi que les Apôtres glorifient,\\
Toi que proclament les prophètes,\\
Toi dont témoignent les martyrs ;\\
C’est toi que par le monde entier, l’Église annonce et reconnaît.\\
Dieu, nous t’adorons : Père infiniment saint,\\
Fils éternel et bien-aimé, Esprit de puissance et de paix.\\
Christ, le Fils du Dieu vivant, le Seigneur de la gloire,\\
Tu n’as pas craint de prendre chair dans le corps d’une vierge\\
Pour libérer l’humanité captive.\\
Par ta victoire sur la mort, \\
Tu as ouvert à tout croyant les portes du Royaume ;\\
Tu règnes dans la gloire à la droite du Père ;\\
Nous croyons que tu viendras pour le jugement.\\
Nous t'en supplions donc, porte secours à tes fidèles,\\
Que tu as rachetés par ton sang :\\
Prends-les pour l'éternité avec tous les saints dans ta lumière.\\
Sauve ton peuple, Seigneur et bénis ton héritage ;\\
Sois leur guide et conduis-les sur le chemin d’éternité.\\
Chaque jour nous te bénissons,\\
Nous te louons à jamais dans les siècles des siècles.\\
Daigne, Seigneur, en ce jour, nous garder de tout péché.\\
Aie pitié de nous, Seigneur : aie pitié de nous.\\
Que ta miséricorde soit sur nous puisque tu es notre espoir.\\
Tu es, Seigneur, mon espérance : jamais, je ne serai déçu.

\switchcolumn*

\vv Dóminus vobíscum. \\
\rr Et cum spíritu tuo.

Orémus.\\
Deus, qui beátum Raphaélem Archángelum Tobíæ fámulo tuo cómitem dedísti in via: concéde nobis fámulis tuis; ut eiúsdem semper protegámur custódia et muniámur auxílio.
Per Dóminum nostrum Iesum Christum, Fílium tuum: qui tecum vivit et regnat in unitáte Spíritus Sancti, Deus, per ómnia sǽcula sæculórum. \\
\rr Amen.

\switchcolumn

\vv Le Seigneur soit avec vous. \\
\rr Et avec votre esprit.

Prions. \\
Dieu, qui as donné le bienheureux Archange Raphaël comme compagnon de route à ton serviteur Tobie, accorde-nous, à nous tes serviteurs, la grâce d’être toujours protégés et secourus par ce même Archange.
Par Notre Seigneur Jésus Christ, ton Fils, qui vit et règne avec toi et le Saint-Esprit, Dieu, maintenant et pour les siècles des siècles.\\
\rr Amen.

\switchcolumn*

\vv Dóminus vobíscum. \\
\rr Et cum spíritu tuo.

~~

\smallscore{ORBDa_marteo}

\switchcolumn

\vv Le Seigneur soit avec vous. \\
\rr Et avec votre esprit.

~~

\vv Bénissons le Seigneur. \\
\rr Nous rendons grâces à Dieu.

\switchcolumn*

~~

\vv Fidélium ánimæ per misericórdiam Dei requiéscant in pace. \\
\rr Amen.

\switchcolumn

~~

\vv Que par la miséricorde de Dieu, les âmes des fidèles trépassés reposent en paix. \\
\rr Amen.

\end{paracol}
\end{document}