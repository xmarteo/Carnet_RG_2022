\documentclass[twoside]{article}


\usepackage[paperwidth=150mm, paperheight=230mm]{geometry}
\usepackage{fontspec}
%\usepackage[latin1]{inputenc}
\usepackage[french]{babel}
\usepackage[strict]{changepage}
\usepackage{fancyhdr}
\usepackage{paracol}
\usepackage{tableof}
\usepackage{setspace}
\usepackage{alltt}
\usepackage{titlesec}
\usepackage{xcolor}
\usepackage{xstring}
\usepackage{parskip}
\usepackage{enumitem}
\usepackage{etoolbox}
\usepackage{needspace}

%%%%%%%%%%%%%%%%%%%%%%%%%%%%%%%%%%%%%%%%%%%%%%%%%%% Mise en page %%%%%%%%%%%%%%%%%%%%%%%%%%%%%%
% on numérote les nbp par page et non globalement
\usepackage[perpage]{footmisc}

% définition des en-têtes et pieds de page
\pagestyle{empty}
\fancyhead{}
\fancyfoot{}
\renewcommand{\headrulewidth}{0pt}
\setlength{\headheight}{0pt}

% la commande titres permet de changer les titres de gauche et de droite.
\newcommand{\titres}[2]{
	\renewcommand{\rightmark}{\textcolor{red}{\sc #2}}
	\renewcommand{\leftmark}{\textcolor{red}{\sc #1}}
}
\titres{}{}

% pas d'indentation
\setlength{\parindent}{0mm}

\geometry{
inner=20mm,
outer=20mm,
top=10mm,
bottom=25mm,
headsep=0mm,
}

\twosided[p]

%%%%%%%%%%%%%%%%%%%%%%%%%%%%%%%%%%%%%%%%%%%%%%%%% Options gregorio %%%%%%%%%%%%%%%%%%%%%%%%%

\usepackage[autocompile]{gregoriotex}
%\usepackage{gregoriotex}

\definecolor{gregoriocolor}{RGB}{215,65,29}

%% style général de gregorio :
% lignes rouges, commenter pour du noir
%\gresetlinecolor{gregoriocolor}

% texte <alt> (au-dessus de la portée) en rouge et en petit, avec réglage de sa position verticale
\grechangestyle{abovelinestext}{\color{gregoriocolor}\footnotesize}
\newcommand{\altraise}{-1mm}
\grechangedim{abovelinestextraise}{\altraise}{scalable}

% taille des initiales
\newcommand{\initialsize}[1]{
    \grechangestyle{initial}{\fontspec{ZallmanCaps}\fontsize{#1}{#1}\selectfont}
}
\newcommand{\defaultinitialsize}{32}
\initialsize{\defaultinitialsize}
% espace avant et après les initiales
\newcommand{\initialspace}[1]{
  \grechangedim{afterinitialshift}{#1}{scalable}
  \grechangedim{beforeinitialshift}{#1}{scalable}
}
\newcommand{\defaultinitialspace}{0cm}
\initialspace{\defaultinitialspace}


% on définit le système qui capture des headers pour générer des annotations
% cette commande sera appelée pour définir des abréviations ou autres substitutions
\newcommand{\resultat}{}
\newcommand{\abbrev}[3]{
  \IfSubStr{#1}{#2}{ \renewcommand{\resultat}{#3} }{}
}
\newcommand{\officepartannotation}[1]{
  \renewcommand{\resultat}{#1}
  \abbrev{#1}{ntro}{Intr.}
  \abbrev{#1}{re}{Resp.}
  \abbrev{#1}{espo}{Resp.}
  \abbrev{#1}{adu}{Gr.}
  \abbrev{#1}{ll}{All.}
  \abbrev{#1}{act}{Tract.}
  \abbrev{#1}{equen}{Seq.}
  \abbrev{#1}{ffert}{Off.}
  \abbrev{#1}{ommun}{Co.}
  \abbrev{#1}{an}{Ant.}
  \abbrev{#1}{ntiph}{Ant.}
  \abbrev{#1}{ntic}{Cant.}
  \abbrev{#1}{ymn}{Hy.}
  \abbrev{#1}{salm}{}
  \abbrev{#1}{Toni Communes}{}
  \abbrev{#1}{yrial}{}
  \greannotation{\resultat}
}
\newcommand{\modeannotation}[1]{
  \renewcommand{\resultat}{#1}
  \abbrev{#1}{1}{ {\sc i} }
  \abbrev{#1}{2}{ {\sc ii} }
  \abbrev{#1}{3}{ {\sc iii} }
  \abbrev{#1}{4}{ {\sc iv} }
  \abbrev{#1}{5}{ {\sc v} }
  \abbrev{#1}{6}{ {\sc vi} }
  \abbrev{#1}{7}{ {\sc vii} }
  \abbrev{#1}{8}{ {\sc viii} }
  \greannotation{\resultat}
}
\gresetheadercapture{office-part}{officepartannotation}{}
\gresetheadercapture{mode}{modeannotation}{string}

%%%%%%%%%%%%%%%%%%%%%%%%%%%%%%%%%%%%%%%%%%%%%% Graphisme %%%%%%%%%%%%%%%%%%%%%%%%%%%
% on définit l'échelle générale

\newcommand{\echelle}{1}

% on centre les titres et on ne les numérote pas
\titleformat{\section}[block]{\Large\filcenter\sc}{}{}{}
\titleformat{\subsection}[block]{\large\filcenter\sc}{}{}{}
\titleformat{\paragraph}[block]{\filcenter\sc}{}{}{}
\setcounter{secnumdepth}{0}
% on diminue l'espace avant les titres
\titlespacing*{\paragraph}{0pt}{1.8ex plus .4ex minus .4ex}{1.2ex plus .2ex minus .2ex}

% commandes versets, repons et croix
\newcommand{\vv}{\textcolor{gregoriocolor}{\fontspec[Scale=\echelle]{Charis SIL}℣.\hspace{3mm}}}
\newcommand{\rr}{\textcolor{gregoriocolor}{\fontspec[Scale=\echelle]{Charis SIL}℟.\hspace{3mm}}}
\newcommand{\cc}{\textcolor{gregoriocolor}{\fontspec[Scale=\echelle]{FreeSerif}\symbol{"2720}~}}
\renewcommand{\aa}{\textcolor{gregoriocolor}{\fontspec[Scale=\echelle]{Charis SIL}\Abar.\hspace{3mm}}}

% commandes diverses
\newcommand{\antiphona}{\textcolor{gregoriocolor}{\noindent Antiphona.\hspace{4mm}}}
\newcommand{\antienne}{\textcolor{gregoriocolor}{\noindent Antienne.\hspace{4mm}}}
% rubrique
\newcommand{\rubrique}[1]{\textcolor{gregoriocolor}{\emph{#1}}}
% pour afficher du texte noir roman au milieu d'une rubrique
\newcommand{\normaltext}[1]{{\normalfont\normalcolor #1}}

\newcommand{\saut}{\hspace{1cm}}
\newcommand{\capsaut}{\hspace{3mm}}
% pour affichier 1 en rouge et un peu d'espace
\newcommand{\un}{{\color{gregoriocolor} 1~~~}}


% abréviations
\newcommand{\tpalleluia}{\rubrique{(T.P.} \mbox{Allelúia.\rubrique{)}}}
\newcommand{\tpalleluiafr}{\rubrique{(T.P.} \mbox{Alléluia.\rubrique{)}}}

\newcommand{\tqomittitur}{{\small \rubrique{(In Tempore Quadragesimæ ommittitur} Allelúia.\rubrique{)}}}
\newcommand{\careme}{{\small \rubrique{(Pendant le Carême on omet l'}Alléluia.\rubrique{)}}}

% environnement hymne : alltt + normalfont + marges custom
\newenvironment{hymne}
  {
  \begin{adjustwidth}{1.6cm}{1mm}
  \begin{alltt}\normalfont
  }
  {
  \end{alltt}
  \end{adjustwidth}
  }
  
% la commande \u permet de souligner les inflexions
\let\u\textbf

% on définit la police par défaut
\setmainfont[Ligatures=TeX, Scale=\echelle]{Charis SIL}
%renderer=ICU a l'air de ne plus marcher...
%\setmainfont[Renderer=ICU, Ligatures=TeX, Scale=\echelle]{Charis SIL}
\setstretch{0.9}


%%%%%%%%%%%%%%%% Commandes de mise en forme %%%%%%%%%%%%%%%%

\newcommand{\lectioresponsorium}[8]{
	\needspace{3\baselineskip}
	\paragraph{#1}

	\vv Jube, domne, benedícere.\\
	\vv #3.\\
	\rr Amen.

	#5

	\vv Tu autem, Dómine, miserére nobis.\\
	\rr Deo grátias.

	\ifblank{#7}{}{\gregorioscore{gabc/#7}}

	\switchcolumn
	%\needspace{3\baselineskip}
	\paragraph{#2}

	\vv Veuillez, maître, bénir.\\
	\vv #4.\\
	\rr Amen.

	#6

	\vv Et toi Seigneur, prends pitié de nous.\\
	\rr Nous rendons grâces à Dieu.

	\ifblank{#8}{}{#8}

	\switchcolumn*

}

\newcommand{\versiculus}[2]{
	\vv #1.\\
	\rr #2.\\
}

\newcommand{\versiculusabsolutio}[6]{

	\paragraph{Versiculus et Absolutio}

	\versiculus{#1}{#2}
	\vv Pater noster... \rubrique{(secrètement)} Et ne nos indúcas in tentatiónem. \\
	\rr Sed líbera nos a malo. \\
	\vv #3. \rr Amen.

	\switchcolumn

	\paragraph{Versicule et Absolution}

	\versiculus{#4}{#5}
	\vv Notre Père... Et ne nous laisse pas entrer en tentation. \\
	\rr Mais délivre-nous du mal. \\
	\vv #6. \rr Amen.

	\switchcolumn*

}

\newcommand{\versetGloireAuPere}{
	\vv Gloire au Père, au Fils, et au Saint-Esprit.
}

\newcommand{\gscore}[1]{
	\gregorioscore{gabc/#1}
}

\newcommand{\smallscore}[1]{
	\gresetinitiallines{0}
	\gscore{#1}
	\gresetinitiallines{1}
}

\begin{document}

\null \newpage

\sloppy

\begin{center}\begin{doublespace}
{\fontspec[Scale=\echelle]{Futura Book}
\MakeUppercase{\Large Messe votive des saints anges}\\
selon l'usage réformé du rite romain
}
\end{doublespace}\end{center}

\rubrique{L'ordinaire de la messe en chant grégorien est donné à la messe du samedi, mémoire de saint Jean-Paul II, pape.}

\gscore{in_benedicite}

\rubrique{Ps. 102  : 20-21, 1-3.}~~~~~~
\emph{\rr Bénissez le Seigneur, tous ses anges : vous qui disposez de puissance, qui accomplissez sa parole, pour qu’on entende la voix de ses dispositions.\\
\vv \rubrique{\emph{1. }} Bénis le Seigneur, ô mon âme, bénis son nom très saint, tout mon être !\\
\vv \rubrique{\emph{2. }} Bénis le Seigneur, ô mon âme, n'oublie aucun de ses bienfaits !\\
\vv \rubrique{\emph{3. }} Car il pardonne toutes tes offenses et te guérit de toute maladie.}

\gscore{ky_kyrie_13_FO}

\paragraph{Collecte}

\begin{paracol}{2}
\vv Orémus.
\switchcolumn
\vv Prions le Seigneur.
\switchcolumn*
Deus, qui miro órdine
Angelórum ministéria hominúmque dispénsas,
concéde propítius,
ut, a quibus tibi ministrántibus in cælo semper assístitur,
ab his in terra vita nostra muniátur.
Per Dóminum nostrum Iesum Christum Fílium
tuum, qui tecum vivit et regnat in unitáte Spíritus Sancti, Deus, per ómnia
sǽcula sæculórum.\\
\rr Amen.
\switchcolumn
Seigneur Dieu,
avec une sagesse admirable,
tu assignes leurs fonctions aux anges et aux hommes;
nous t'en prions:
fais que notre vie soit protégée sur la terre
par ceux qui, dans le ciel,
servent toujours en ta présence.
Par Jésus Christ, ton Fils, notre Seigneur,
qui vit et règne avec toi dans l'unité du Saint-Esprit,
Dieu, pour les siècles des siècles.\\
\rr Amen.
\end{paracol}

\paragraph{Première lecture}

\begin{paracol}{2}
Lectio Epístolæ beati Páuli \\ Apóstoli ad Ephésios.
\switchcolumn
Lecture de la lettre de Saint Paul, Apôtre, aux Éphésiens.
\end{paracol}
\begin{alltt}\normalfont
    Frères,
        soyez entre vous pleins de générosité et de tendresse.
    Pardonnez-vous les uns aux autres,
        comme Dieu vous a pardonné dans le Christ.
    Oui, cherchez à imiter Dieu,
        puisque vous êtes ses enfants bien-aimés.
    Vivez dans l’amour,
        comme le Christ nous a aimés et s’est livré lui-même pour nous,
        s’offrant en sacrifice à Dieu, comme un parfum d’agréable odeur.
    Comme il convient aux fidèles,
        la débauche, l’impureté sous toutes ses formes et la soif de posséder
        sont des choses qu’on ne doit même plus évoquer chez vous ;
    pas davantage de propos grossiers, stupides ou scabreux
        – tout cela est déplacé –
        mais qu’il y ait plutôt des actions de grâce.
    Sachez-le bien : ni les débauchés, ni les dépravés, ni les profiteurs
        – qui sont de vrais idolâtres –
        ne reçoivent d’héritage dans le royaume du Christ et de Dieu ;
        ne laissez personne vous égarer par de vaines paroles.
    Tout cela attire la colère de Dieu sur ceux qui désobéissent.
        N’ayez donc rien de commun avec ces gens-là.
    Autrefois, vous étiez ténèbres ;
        maintenant, dans le Seigneur, vous êtes lumière ;
        conduisez-vous comme des enfants de lumière.
\end{alltt}

\smallscore{or04--verbum_domini_solesmes}

\emph{\vv Parole du Seigneur. ~~~~ \rr Nous rendons grâce à Dieu.}

\gscore{gr_laudate}

\rubrique{Ps. 148 : 1-2}\\
\emph{\rr Louez le Seigneur du haut des cieux, louez-le dans les hauteurs.\\
\vv Louez-le, tous ses anges : louez-le, toutes ses puissances.}\\
~\\
\gscore{al_angelus_domini}

\rubrique{Mat. 28 : 2}\\
\emph{Alléluia, alléluia. L’Ange du Seigneur descendit du ciel, et, s’approchant, il fit rouler la pierre, et il était assis dessus. Alléluia.}

\paragraph{Évangile}

\smallscore{or05--dominus_vobiscum_..._lectio_sancti_evangelii_(c)--solesmes}
\emph{\vv Le Seigneur soit avec vous. ~~~~\rr Et avec votre esprit.\\
\vv Évangile de Jésus-Christ selon saint Luc. \\
\rr Gloire à toi, Seigneur.}

\begin{alltt}\normalfont
    En ce temps-là,
    Jésus était en train d’enseigner dans une synagogue,
        le jour du sabbat.
    Voici qu’il y avait là une femme, possédée par un esprit
        qui la rendait infirme depuis dix-huit ans ;
        elle était toute courbée
        et absolument incapable de se redresser.
    Quand Jésus la vit, il l’interpella et lui dit :
        « Femme, te voici délivrée de ton infirmité. »
    Et il lui imposa les mains.
    À l’instant même elle redevint droite
        et rendait gloire à Dieu.

    Alors le chef de la synagogue, indigné
        de voir Jésus faire une guérison le jour du sabbat,
        prit la parole et dit à la foule :
    « Il y a six jours pour travailler ;
        venez donc vous faire guérir ces jours-là,
        et non pas le jour du sabbat. »
    Le Seigneur lui répliqua :
        « Hypocrites !
    Chacun de vous, le jour du sabbat,
        ne détache-t-il pas de la mangeoire son bœuf ou son âne
        pour le mener boire ?
    Alors cette femme, une fille d’Abraham,
        que Satan avait liée voici dix-huit ans,
        ne fallait-il pas la délivrer de ce lien le jour du sabbat ? »

    À ces paroles de Jésus,
        tous ses adversaires furent remplis de honte,
        et toute la foule était dans la joie
        à cause de toutes les actions éclatantes qu’il faisait.
\end{alltt}

\smallscore{or06--verbum_domini_(c)--solesmes}
\emph{\vv Parole du Seigneur. ~~~~ \rr Louange à toi, ô Christ.}

\paragraph{Offertoire}

%\gscore{of} TODO

\rubrique{Ap. 8 : 03-04}

\emph{L’Ange se tint près de l’autel du temple,
ayant un encensoir d’or à la main :
et il lui fut donné abondance d’encens :
et de la main de l’Ange, la fumée des parfums
s’éleva en présence du Seigneur.}

\paragraph{Prière sur les offrandes}
\begin{paracol}{2}

\vv Oráte, fratres:
ut meum ac vestrum sacrifícium
acceptábile fiat apud Deum Patrem omnipoténtem.

\rr Suscípiat Dóminus sacrifícium de mánibus tuis
ad laudem et glóriam nóminis sui,
ad utilitátem quoque nostram
totiúsque Ecclésiæ suæ sanctæ.

\switchcolumn

\vv Priez, frères et sœurs : que mon sacrifice, et le vôtre, soit agréable à Dieu le Père tout-puissant.

\rr Que le Seigneur reçoive de vos mains ce sacrifice à la louange et à la gloire de son nom, pour notre bien et celui de toute l’Église.

\switchcolumn*

Hóstias tibi, Dómine, laudis offérimus,
supplíciter deprecántes, ut eásdem,
angélico ministério in conspéctum tuæ maiestátis delátas,
et placátus accípias,
et ad salútem nostram proveníre concédas.
Per Christum Dóminum nostrum.\\
\rr Amen.

\switchcolumn
Le sacrifice de louange,
porté par les anges en présence de ta gloire,
nous te l'offrons, Seigneur,
avec nos humbles prières:
accueille-le favorablement,
pour qu'il nous obtienne le salut.
Par le Christ, notre Seigneur.\\
\rr Amen.

\end{paracol}

\paragraph{Préface des anges}

\smallscore{or07--dominus_vobiscum_..._sursum_corda_(b._tonus_sollemnis)--solesmes}

\emph{\vv Le Seigneur soit avec vous.~~~~\rr Et avec votre esprit.\\
\vv Élevons notre cœur.~~~~\rr Nous le tournons vers le Seigneur.\\
\vv Rendons grâce au Seigneur notre Dieu.~~~~\rr Cela est juste et bon.}

\begin{paracol}{2}
\switchcolumn
\switchcolumn*

Vere dignum et iustum est, æquum et salutáre,
nos tibi semper et ubíque grátias ágere:
Dómine, sancte Pater, omnípotens ætérne Deus:
Et in Archángelis Angelísque tuis tua præcónia non tacére,
quia ad excelléntiam tuam recúrrit et glóriam
quod angélica creatúra tibi probábilis honorétur:
et, cum illa sit amplo decóre digníssima,
tu quam sis imménsus et super ómnia præferéndus osténderis,
per Christum Dóminum nostrum.
Per quem multitúdo Angelórum tuam célebrat maiestátem,
quibus adorántes in exsultatióne coniúngimur,
una cum eis laudis voce clamántes:

\switchcolumn

Vraiment, il est juste et bon, pour ta gloire et notre salut,
de t'offrir notre action de grâce, toujours et en tout lieu,
Seigneur, Père très saint, Dieu éternel et tout-puissant.
Oui, il est bon de te chanter pour les archanges et les anges,
car c'est ta perfection et ta gloire que rejoint notre louange
lorsqu'elle honore ces créatures spirituelles,
et leur splendeur manifeste combien tu es grand
et surpasses tous les êtres,
par le Christ, notre Seigneur .
Par lui, la multitude des anges célèbre ta grandeur:
dans l'allégresse d'une même adoration,
laisse-nous joindre nos voix à leur louange,
pour chanter et proclamer:

\end{paracol}

\gscore{ky_sanctus_13_FO}

\gscore{ky_agnus_13}

\gscore{co_benedicite}
 
\rubrique{Luc 12 : 42 \& Ps. 131}\\
\emph{\rr Le serviteur fidèle et avisé que le Seigneur établit sur ses gens pour leur donner en temps voulu leur ration de froment.\\
\vv \rubrique{\emph{12a. }} Si tes fils gardent mon alliance, les volontés que je leur fais connaître,\\
\vv \rubrique{\emph{12b. }} Leurs fils, eux aussi, à tout jamais, siègeront sur le trône dressé pour toi.\\
\vv \rubrique{\emph{13. }} Car le Seigneur a fait choix de Sion ; elle est le séjour qu'il désire.\\
\vv \rubrique{\emph{15. }} Je bénirai, je bénirai ses récoltes pour rassasier de pain ses pauvres.\\
\vv \rubrique{\emph{16. }} Je vêtirai de gloire ses prêtres, et ses fidèles crieront, crieront de joie.
}

\paragraph{Postcommunion}
\begin{paracol}{2}
\vv Orémus.
\switchcolumn
\vv Prions le Seigneur.
\switchcolumn*
Pane cælésti refécti,
súpplices te, Dómine, deprecámur,
ut, eius fortitúdine roboráti,
sub Angelórum fidéli custódia,
fortes, salútis progrediámur in via.
Per Christum Dóminum nostrum.\\
\rr Amen.
\switchcolumn
Nourris par le pain du ciel,
nous te supplions, Seigneur:
puissions-nous, avec cette force neuve
et sous la fidèle protection des anges,
avancer courageusement dans la voie du salut.
Par le Christ, notre Seigneur.\\
\rr Amen.
\end{paracol}

\paragraph{Envoi}

\smallscore{or12--dominus_vobiscum_..._benedicat_(b)--solesmes}

\begin{paracol}{2}
\end{paracol}
\end{document}